\setchapterstyle{kao}
\setchapterpreamble[u]{\margintoc}
\chapter{Integration}
\labch{integration}

\section{Measurable functions}

If $\mathcal{N}$ is a $\sigma$-algebra on $Y$, $\{ f^{-1}(E): E \in \mathcal{N} \}$ is a $\sigma$-algebra on $X$.

\begin{definition}[Measurable function]
    If $(X, \mathcal{M})$ and $(Y, \mathcal{N})$ are measurable spaces, a mapping $f: X \to Y$ is called $(\mathcal{M}, \mathcal{N})$-measurable, or just measurable when $\mathcal{M}$ and $\mathcal{N}$ are understood, if $f^{-1}(E) \in \mathcal{M}$ for all $E \in \mathcal{N}$.
\end{definition}

The composition of measurable functions is measurable.
If $f: X \to Y$ is $(\mathcal{M}, \mathcal{N})$-measurable and $g: Y \to Z$ is $(\mathcal{N}, \mathcal{O})$-measurable, then $g \circ f$ is $(\mathcal{M}, \mathcal{O})$-measurable.

\begin{proposition}
    If $\mathcal{N}$ is generated by $\mathcal{E}$, then $f: X \to Y$ is $(\mathcal{M}, \mathcal{N})$-measurable iff $f^{-1}(E) \in \mathcal{M}$ for all $E \in \mathcal{E}$. 
\end{proposition}

\begin{proof}
    TODO
\end{proof}

\begin{definition}
    If  $(X, \mathcal{M})$ is a measurable space, a real- or complex-valued function $f$ on $X$ will be called $\mathcal{M}$-measurable or just measurable if it is $(\mathcal{M}, \mathcal{B}_{\R})$- or $(\mathcal{M}, \mathcal{B}_{\C})$-measurable.
\end{definition}

\begin{definition}[Lebesgue measurable]
    \labdef{Lebesgue_measurable}
    $f: \R \to \R$ is Lebesgue measurable if it is $(\mathcal{L}, \mathcal{B}_{\R})$-measurable; likewise for $f:\R \to \C$.
\end{definition}

\begin{definition}[Borel measurable]
    \labdef{Borel_measurable}
    $f: \R \to \R$ is Borel measurable if it is $(\mathcal{B}_{\R}, \mathcal{B}_{\R})$-measurable; likewise for $f:\R \to \C$.
\end{definition}

\begin{remark}
    If $f, g$ are Lebesgue measurable, it does not follow that $f \circ g$ is Lebesgue measurable, even if $g$ is assumed to be continuous.
\end{remark}

\begin{proposition}
    If $(X, \mathcal{M})$ is a measurable space and $f: X \to \R$, the following are equivalent:
    \begin{enumerate}
        \item $f$ is $\mathcal{M}$-measurable.
        \item $f^{-1}((a, \infty)) \in \mathcal{M}$ for all $a \in \R$.
        \item $f^{-1}([a, \infty)) \in \mathcal{M}$ for all $a \in \R$.
        \item $f^{-1}((-\infty, a)) \in \mathcal{M}$ for all $a \in \R$.
        \item $f^{-1}((-\infty, a]) \in \mathcal{M}$ for all $a \in \R$.
    \end{enumerate}
\end{proposition}

\begin{proof}
    TODO
\end{proof}

\begin{definition}
    If $(X, \mathcal{M})$ is a measurable space and $f$ is a function on $X$, and $E \in \mathcal{M}$, $f$ is measurable on $E$ if $f^{-1}(B) \cap E \in \mathcal{M}$ for all Borel sets $B$.
    Equivalently, $f|E$ is $\mathcal{M}_E$-measurable, where $\mathcal{M}_E = \{ F \cap E: F \in \mathcal{M} \}$.
\end{definition}

\begin{proposition}
    Let $(X, \mathcal{M})$ and $(Y_{\alpha}, \mathcal{N}_{\alpha})$ ($\alpha \in A$) be measurable spaces, $Y = \prod_{\alpha \in A} Y_{\alpha}$, $\mathcal{N} = \prod_{\alpha \in A} \mathcal{N}_{\alpha}$, and $\pi_{\alpha}: Y \to Y_{\alpha}$ the coordinate maps. Then $f: X \to Y$ is $(\mathcal{M}, \mathcal{N})$-measurable iff $f_{\alpha} = \pi_{\alpha} \circ f$ is $(\mathcal{M}, \mathcal{N}_{\alpha})$-measurable for all $\alpha \in A$.
\end{proposition}

\begin{proof}
    TODO
\end{proof}

\begin{proof}
    TODO
\end{proof}

\begin{corollary}
    A function $f: X \to \C$ is $\mathcal{M}$-measurable iff $\Re(f)$ and $\Im(f)$ are $\mathcal{M}$-measurable.
\end{corollary}

\begin{proposition}
    If $f,g$ are $\mathcal{M}$-measurable, then so are $f+g$ and $f g$.
\end{proposition}

\begin{proof}
    TODO
\end{proof}

\begin{proposition}
    If $\{f_j\}$ is a sequence of $\bar{\R}$-valued measurable functions on $(X, \mathcal{M})$, then the functions $\sup_j f_j(x)$, $\inf_j f_j(x)$, $\limsup_{j \to \infty} f_j(x)$, and $\liminf_{j \to \infty} f_j(x)$ are all measurable.
    If $f(x) = \lim_{j \to \infty}$ exists for every $x \in X$, then $f$ is measurable.
\end{proposition}

\begin{proof}
    TODO
\end{proof}

\begin{corollary}
    If $f, g: X \to \C$ are $\mathcal{M}$-measurable, then so are functions $\max(f, g)$ and $\min(f, g)$.
\end{corollary}

\begin{definition}
    If $f: X \to \bar{\R}$, we define the positive and negative parts of $f$ to be $f^{+}(x) = \max(f(x), 0)$ and $f^{-}(x) = \min(f(x), 0)$, respectively. 
\end{definition}

Clearly, $f = f^{+} - f^{-}$.
And if $f$ is measurable, so are $f^{+}$ and $f^{-}$.

\begin{definition}[Polar decomposition]
    If $f: X \to \C$, the polar decomposition of $f$ is:
    \begin{align}
        f = (\sgn f) |f|,
    \end{align}
    where 
    \begin{align}
        \sgn z = \begin{cases}
            z / |z|, \text{ if } z \ne 0, \\
            0, \text{ if } z = 0.
        \end{cases}
    \end{align}
\end{definition}

Again, if $f$ is measurable, so are $\sgn f$ and $|f|$.

\begin{definition}
    Suppose $(X, \mathcal{M})$ is a measurable space.
    If $E \subset X$, the characteristic function $\chi_E$ of $E$ (sometimes called the indicator function of $E$ and denoted by $\mathbf{1}_E$) is defined by
    \begin{align}
        \chi_E(X) = \begin{cases}
            1, \text{ if } x \in E, \\
            0, \text{ if } x \notin E.
        \end{cases}
    \end{align}
\end{definition}

\begin{theorem}
    Let $(X, \mathcal{M})$ be a measurable space.
    \begin{enumerate}
        \item If $f: X \to [0, \infty]$ is measurable, there is a sequence $\{ \phi_n \}$ of simple functions such that $0 \le \phi_1 \le \phi_2 \le \cdots \le f$, $\phi_n \to f$ pointwise, and $\phi_n \to f$ uniformly on any set on which $f$ is bounded.
        \item If $f: X \to \C$ is measurable, there is a sequence $\{ \phi_n \}$ of simple functions such that $0 \le |\phi_1| \le |\phi_2| \le \cdots \le |f|$, $\phi_n \to f$ pointwise, and $\phi_n \to f$ uniformly on any set on which $f$ is bounded.
    \end{enumerate}
\end{theorem}

\begin{proof}
    TODO
\end{proof}

\begin{proposition}
    The following implications are valid iff the measure $\mu$ is complete:
    \begin{enumerate}
        \item If $f$ is measurable and $f = g$ $\mu$-almost everywhere, then $g$ is measurable.
        \item If $f_n$ is measurable for $n \in \N$ and $f_n \to f$ $\mu$-almost everywhere, then $f$ is measurable.
    \end{enumerate}
\end{proposition}

\begin{proof}
    TODO
\end{proof}

\begin{proposition}
    Let $(X, \mathcal{M}, \mu)$ be a measurable space and $(X, \bar{\mathcal{M}}, \bar{\mu})$ be its completion.
    If $f$ is an $\bar{\mathcal{M}}$-measurable function on X, there is an $\mathcal{M}$-measurable function $g$ such that $f = g$ $\bar{\mu}$-almost everywhere.
\end{proposition}

\begin{proof}
    TODO
\end{proof}

\section{Integration of nonnegative functions}

\begin{definition}
    If $\phi$ is a simple function in $L^+$ with standard representation $\phi = \sum_{j=1}^{n} a_j \chi_{E_j}$, we define the integral of simple function $\phi$ with respect to $\mu$ by
    \begin{align}
        \int \phi \dd \mu = \sum_{j=1}^{n} a_j \mu(E_j).
    \end{align}
\end{definition}

Note that $\int \phi \dd \mu$ may equal $\infty$.
When there is no danger of confusion, we may also write $\int \phi$ for $\int \phi \dd \mu$.

To summarize, $\int_{A} \phi \dd \mu = \int_{A} \phi = \int_{A} \phi(x) \dd \mu(x) = \int \phi \chi_{A} \dd \mu$ and $\int \phi = \int_{X} \phi$.

\begin{proposition}[Properties for integrals of simple functions]
    Let $\phi$ and $\psi$ be simple function in $L^+$.
    \begin{enumerate}
        \item $\int c \phi = c \int \phi$ for $c > 0$.
        \item $\int (\phi + \psi) = \int \phi + \int \psi$.
        \item If $\phi \le \psi$, then $\int \phi \le \int \psi$.
        \item The map $A \mapsto \int_{A} \dd \mu$ is a measure on $\mathcal{M}$.
    \end{enumerate}
\end{proposition}

\begin{proof}
    TODO
\end{proof}

After defining integrals for simple functions and examining its crucial properties, we now extend the integral to all functions in $L^+$.

\begin{definition}
    For any function $f \in L^+$, we define its integral by
    \begin{align}
        \int f \dd \mu = \sup \left\{ \int \phi \dd \mu: 0 \le \phi \le f, \phi \text{ is simple} \right\}.
    \end{align}
\end{definition}

\begin{theorem}[The monotone convergence theorem]
    If $\{ f_n \}$ is a sequence in $L^+$ such that $f_j \le f_{j+1}$ for all $j$, and $f = \lim_{n \to \infty} f_n = \sup_{n} f_n$, then $\int f = \lim_{n \to \infty} \int f_n$.
\end{theorem}

\begin{theorem}
    If $\{ f_n \}$ is a finite or infinite sequence in $L^+$ and $f = \sum_{n} f_n$, then $\int f = \sum \int f_n$.
\end{theorem}

\begin{proposition}
    If $f \in L^+$, then $\int f = 0$ iff $f = 0$ almost everywhere.
\end{proposition}

This implies if $f, g \in L^+$, then $\int f = \int g$ iff $f = g$ almost everywhere.

\begin{corollary}
    If $\{ f_n \} \subset L^+$, $f \in L^+$, and $f_n(x)$ increases to $f(x)$ for almost everywhere $x$, then $\int f = \lim \int f_n$.
\end{corollary}

\begin{lemma}[Fatou's Lemma]
    If $\{ f_n \}$ is any sequence in $L^+$, then 
    \begin{align}
        \int (\liminf f_n) \le \liminf \int f_n.
    \end{align}
\end{lemma}

\begin{proof}
    
\end{proof}

\begin{corollary}
    If $\{ f_n \} \subset L^+$, $f \in L^+$, and $f_n(x) \to f$ almost everywhere, then $\int f \le \liminf \int f_n$.
\end{corollary}

\begin{proposition}
    If $f \in L^+$ and $\int f < \infty$, then $\{ x: f(x) = \infty \}$ is a null set and $\{ x: f(x) > 0 \}$ is $\sigma$-finite. 
\end{proposition}

\begin{proof}
    First, suppose $\{ x : f(x) = \infty \}$ is not a null set.
    So $\mu(\{ x : f(x) = \infty \}) > 0$. 
    Let us construct a simple function $\phi \in L^+$ by
    \begin{align}
        \phi(x) = \begin{cases}
            \infty, \text{ if } f(x) = \infty, \\
            0, \text{ if } f(x) < \infty.
        \end{cases}
    \end{align}
    Then $0 \le \phi \le f$, and $\int f \ge \int \phi = \infty \cdot \mu(\{ x : f(x) = \infty \}) = \infty$ which contradicts to $\int f < \infty$.
    Therefore, $\{ x : f(x) = \infty \}$ is a null set.

    Second, suppose $\{ x: f(x) > 0 \}$ is not $\sigma$-finite which means there exists $\mu(E_j) = \infty$ for some $j$ with $\{ x: f(x) > 0 \} = \bigcup_{j=1}^{\infty} E_j$ for whatever $\{ E_j \}$.
    Take $E_j = \{ x: f(x) > 1/j \}$ and $\mu(E_j) = \infty$.
    Let us again construct a simple function $\psi \in L^+$ by
    \begin{align}
        \psi(x) = \begin{cases}
            1/j, \text{ if } x \in E_j, \\
            0, \text{ if } x \notin E_j.
        \end{cases}
    \end{align} 
    Then $0 \le \psi \le f$, and $\int f \ge \int \psi = (1/j) \cdot \mu(\{ x : f(x) > 1/j \}) = \infty$ which contradicts to $\int f < \infty$.
    Therefore, $\{ x: f(x) > 0 \}$ is $\sigma$-finite.
\end{proof}

\section{Integration of complex functions}

We continue to work on a fixed measure space $(X, \mathcal{M}, \mu)$.


\begin{definition}
    We say that a real-valued function $f$ is integrable if $\int f^+$ and $\int f^-$ are both finite.
\end{definition}

\begin{proposition}
    The set of integrable real-valued functions on $X$ is a real vector space, and the integral is a linear function on it. 
\end{proposition}

\begin{proof}
    TODO
\end{proof}

\begin{proposition}
    If $f \in L'$, then $|\int f| \le \int |f|$.
\end{proposition}

\begin{proof}
    TODO
\end{proof}

\begin{proposition}
    If $f \int L'$, then $\{ x: f(x) \ne \infty \}$ is $\sigma$-finite.
\end{proposition}

\begin{proof}
    TODO
\end{proof}

\begin{proposition}
    If $f, g \in L'$, then $\int_E f = \int_E g$ for all $E in \mathcal{M}$ iff $\int |f-g| = 0$ iff $f=g$ almost everywhere.
\end{proposition}

\begin{proof}
    TODO
\end{proof}

This proposition shows that for the purposes of integration it makes no difference if we alter functions on null sets.

\begin{definition}
    $L^1$ is the set of equivalence classes of almost everywhere definited integrable functions on $X$.
\end{definition}

\begin{theorem}[The Dominated Convergence Theorem]
    \labthm{dominated_convergence}
    Let $\{ f_n \}$ be a sequence in $L^1$ such that $f_n \to f$ and there exists a nonnegative $g \in L^1$ such that $|f_n| \le g$ almost everywhere for all $n$. Then $f \in L^1$ and $\int f = \lim_{n \to infty} \int f_n$.
\end{theorem}

\begin{proof}
    TODO
\end{proof}

\begin{theorem}
    Suppose $\{ f_j \}$ is a sequence in $L^1$ such that $\sum_{j=1}^{\infty} \int |f_j| < \infty$.
    Then $\sum_{j=1}^{\infty} f_j$ converges almost everywhere to a function in $L^1$, and $\int \sum_{j=1}^{\infty} f_j = \sum_{j=1}^{\infty} \int f_j$.
\end{theorem}

\begin{proof}
    TODO
\end{proof}

\begin{theorem}
    If $f \in L^{1} (\mu)$ and $\epsilon > 0$, there is an integrable simple function $\phi = \sum a_j \chi_{E_j}$ such that $\int |f - \phi| \dd \mu < \epsilon$.
    If $\mu$ is a Lebesgue-Stieltjes measure on $\R$, the sets $E_j$ in the definition of $\phi$ can be taken to be finite unions of open intervals.
    Moreover, there is a continuous function $g$ that vanishes outside a bounded interval such that $\int |f - g| \dd \mu < \epsilon$.
\end{theorem}

The first assertion states that the integrable simple functions are dense in $L^1$ in the $L^1$ metric.

The next theorem gives a criterion for the validity of interchanging a limit or a derivative with an integral. 

\begin{theorem}
    Suppose that $f: X \times [a, b] \to \C$ with $-\infty < a < b < \infty$ and that $f(\cdot, t): X \to \C$ is integrable for each $t \in [a, b]$.
    Let $F(t) = \int_X f(x,t) \dd \mu(x)$.
    \begin{enumerate}
        \item Suppose that there exists $g\in L^1(\mu)$ such that $|f(x,t)| \le g(x)$ for all $x, t$. If $\lim_{t \to t_0} f(x, t) = f(x, t_0)$ for every $x$, then $\lim_{t \to t_0} F(t) = F(t_0)$; in particular, if $f(x, \cdot)$ is continuous for each $x$, $F$ is continuous.
        \item Suppose that $\pp f / \pp t$ exists and there is an integrable function $g \in L^1(\mu)$ such that $|(\pp f / \pp t)(x, t)| \le g(x)$ for all $x, t$. Then $F$ is differentiable and $F'(x) = \int (\pp f / \pp t)(x,t) \dd \mu(x)$.
    \end{enumerate} 
\end{theorem}

\begin{proof}
    For the first part, let $f_n(x) = f(x, t_n)$ where $\{ t_n \}$ is any sequence in $[a, b]$ converging to $t_0$.
    Then ${f_n}$ is a sequence in $L^1$ as $f(\cdot, t): X \to \C$ is integrable for each $t \in [a, b]$.
    Additionally, we have $f_0(x) = f(x, t_0)$.
    If $\lim_{t \to t_0} f(x,t) = f(x, t_0)$ for every $x$ then $f_n \to f_0$.
    And the existence of $g$ shows that $|f_n| \le g$ for all $n$.
    Then by \refthm{dominated_convergence}, $f_0 \in L^1$ and $\int f_0 = \lim_{n \to \infty} \int f_n$ where $\int f_0 = \int_{X} f(x, t_0) \dd \mu(x) = F(t_0)$ and $\int f_n = \int_{X} f(x, t_n) \dd \mu(x) = F(t_n)$.
    Thus $\lim_{t \to t_0} F(t) = F(t_0)$.

    For the second part, observe that $(\pp f / \pp t)(x,t_0) = \lim_{n \to \infty} h_n(x)$ where $h_n(x) = \frac{f(x, t_n) - f(x, t_0)}{t_n - t_0}$.
    Again, $\{ t_n \}$ is any sequence in $[a, b]$ converging to $t_0$.
    It follows that $\pp f / \pp t$ is measurable\sidenote{Why measurable?}.
    And by the Mean Value Theorem\sidenote{If $f(x)$ is continuous on the closed interval $[a,b]$ and differentiable on the open interval $(a,b)$, then there is a number $c$ such that $a < c < b$ and $f'(c) = \frac{f(b) - f(a)}{b-a}$}, $|h_n(x)| \le \sup_{t \in [a, b]} |(\pp f / \pp t)(x, t)| \le g(x)$.
    Again, apply \refthm{dominated_convergence} to give
    \begin{align}
        F'(t_0) &= \lim_{n \to \infty} \frac{F(t_n) - F(t_0)}{t_n - t_0} \\
        &= \lim_{n \to \infty} \frac{\int_X f(x,t_n) \dd \mu(x) - \int_X f(x,t_0) \dd \mu(x)}{t_n - t_0} \\
        &= \lim_{n \to \infty} \int h_n(x) \dd \mu(x) \\
        &= \int \lim_{n \to \infty} h_n(x) \dd \mu(x) \\
        &= \int (\pp f / \pp t)(x,t) \dd \mu(x)
    \end{align}
\end{proof}

\section{Modes of convergence}

\begin{definition}[Cauchy in measure]
    A sequence $\{ f_n \}$ of measurable complex-valued functions on $(X, \mathcal{M}, \mu)$ is Cauchy in measure if for every $\epsilon$, then $\mu(\{ x: |f_n(x) - f_m(x)| \ge \epsilon \}) \to 0$ as $n, m \to \infty$.
\end{definition}

\begin{definition}[Convergence in measure]
    A sequence $\{ f_n \}$ of measurable complex-valued functions on $(X, \mathcal{M}, \mu)$ converges in measure to $f$ if for every $\epsilon$, then $\mu(\{ x: |f_n(x) - f(x)| \ge \epsilon \}) \to 0$ as $n \to \infty$.
\end{definition}

\begin{example}
    We will examine the following examples:
    \begin{enumerate}
        \item $f_n = n^{-1} \chi_{(0, n)}$
        \item $f_n = \chi_{(n, n+1)}$ 
        \item $f_n = n \chi_{[0, 1/n]}$
        \item $f_n = \chi_{[j/2^k, (j+1)/2^k]}$ where $n = 2^k + j$ with $0 \le j < 2^k$. \eg, $f_1 = \chi_{[0, 1]}$, $f_2 = \chi_{[0, 1/2]}$, $f_3 = \chi_{[1/2, 1]}$, $f_4 = \chi_{[0, 1/4]}$, $f_5 = \chi_{[1/4, 1/2]}$, $f_6 = \chi_{[1/2, 3/4]}$, $f_7 = \chi_{[3/4, 1]}$, \etc
    \end{enumerate}
\end{example}


\begin{proposition}
    If $f_n \to f$ in measure, then $\{ f_n \}$ is Cauchy in measure. 
\end{proposition}

\begin{proof}
    If $\{ f_n \}$ converges in mesaure to $f$ for every $\epsilon$, then $\mu(\{ x: |f_n(x) - f(x)| \ge \epsilon \}) \to 0$ as $n \to \infty$.
    Since $|f_n(x) - f_m(x)| \ge \epsilon$ implies at least one of $|f_n(x) - f(x)| \ge \epsilon'$ and $|f_m(x) - f(x)| \ge \epsilon - \epsilon'$ with $0 < \epsilon' < \epsilon$ must hold, 
    \begin{align}
        \{ x: |f_n(x) - f_m(x)| \ge \epsilon \} & \subset \{ x: |f_n(x) - f(x)| \ge \epsilon' \} \\
        & \cup \{ x: |f_m(x) - f(x)| \ge \epsilon - \epsilon' \}. \nonumber
    \end{align}
    Consequently, for every $\epsilon$
    \begin{align}
        \mu(\{ x: |f_n(x) - f_m(x)| \ge \epsilon \}) & \le \mu(\{ x: |f_n(x) - f(x)| \ge \epsilon' \}) \\
        & + \mu(\{ x: |f_m(x) - f(x)| \ge \epsilon - \epsilon' \}) \to 0, \nonumber
    \end{align}
    as $n, m \to \infty$.
    Therefore, $\{ f_n \}$ is Cauchy in measure.
\end{proof}

\begin{theorem}
    Suppose that $\{ f_n \}$ is Cauchy in measure.
    Then there is a measurable function $f$ such that $f_n \to f$ in measure, and there is a subsequence $\{ f_{n_j} \}$ that converges to $f$ almost everywhere.
    Moreover, if also $f_n \to g$ in measure, then $g = f$ almost everywhere.
\end{theorem}

\begin{theorem}[Egoroff's Theorem]
    \labthm{Egoroff}
    Suppose that $\mu(X) < \infty$, and $\{ f_n \}$ and $f$ are measurable complex-valued functions on $X$ such that $f_n \to f$ almost everywhere.
    Then for every $\epsilon > 0$, there exists $E \subset X$ such that $\mu(E) < \epsilon$ and $f_n \to f$ uniformly on $E^c$.
\end{theorem}

The type of convergence involved in the conclusion of \refthm{Egoroff} is sometimes called almost uniform convergence. 

\begin{definition}
    A sequence of functions $\{ f_n \}$ is almost uniform convergent, if there exists $E \subset X$ such that $\mu(E) < \epsilon$ and $f_n \to f$ uniformly on $E^c$.
\end{definition}

\begin{table*}[h!]
    \caption{Examples of modes of convergence}
    \begin{tabular}{lcccccc}
        \toprule 
                               & \multicolumn{6}{c}{Modes of convergence} \\
        \cmidrule(lr){2-7}
                               &         &           & almost     &          & in      & almost \\
        $f_n$                  & uniform & pointwise & everywhere & in $L^1$ & measure & uniform \\
        \midrule
        $n^{-1} \chi_{(0, n)}$ & Y       & Y         & Y          & N        & Y       &   \\
        $\chi_{(n, n+1)}$      & N       & Y         & Y          & N        & N       &   \\
        $n \chi_{[0, 1/n]}$    & N       & N         & Y          & N        & Y       &   \\
        $\chi_{[j/2^k, (j+1)/2^k]}$ & N  & N         & N          & Y        & Y       &   \\
        \bottomrule
    \end{tabular}
\end{table*}

\section{Product measures}

\begin{proposition}

    \begin{enumerate}
        \item If $E \in \mathcal{M} \times \mathcal{N}$, then $E_x \in \mathcal{N}$ for all $x \in X$ and $E^y \in \mathcal{M}$ for all $y \in Y$.
        \item If $f$ is $\mathcal{M} \otimes \mathcal{N}$-measurable, then $f_x$ is $\mathcal{N}$-measurable for all $x \in X$ and $f^y$ is $\mathcal{M}$-measurable for all $y \in Y$. 
    \end{enumerate}
\end{proposition}

\begin{definition}[Monotone Class]
    A monotone class on a space $X$ is a subset $\mathcal{C}$ of $\mathcal{P}(X)$ that is closed under countable increasing unions and countable decreasing intersections.
    For any $\mathcal{E} \subset \mathcal{P}(X)$, there is a unique smallest monotone class containing $\mathcal{E}$, called the monotone class generated by $\mathcal{E}$.
\end{definition}

By definition, every $\sigma$-algebra is a monotone class.
Since the intersection of any family of monotone classes is a monotone class, there is a unique smallest monotone class containing $\mathcal{E}$, that is the intersection of all monotone class containing $\mathcal{E}$.

\begin{lemma}[The Monotone Class Lemma]
    If $\mathcal{A}$ is an algebra of subsets of $X$, then the monotone class $\mathcal{C}$ generated by $\mathcal{A}$ coincides with the $\sigma$-algebra $\mathcal{M}$ generated by $\mathcal{A}$
\end{lemma}

\begin{theorem}
    
\end{theorem}

\begin{theorem}[The Fubini-Tonelli Theorem]
    Suppose that $(X, \mathcal{M}, \mu)$ and $(Y, \mathcal{N}, \nu)$ are $\sigma$-finite measure spaces.
    \begin{enumerate}
        \item (Tonelli) If $f \in L^+(X \times Y)$, then the functions $g(x) = \int f_x \dd \nu$ and $h(y) = \int f^y \dd \mu$ are in $L^+(X)$ and $L^+(Y)$, respectively, and
        \begin{align}
            \int f \dd (\mu \times \nu) &= \int \left[ \int f(x, y) \dd \nu(y) \right] \dd \mu(x) \labeq{Tonelli} \\
            &= \int \left[ \int f(x, y) \dd \mu(x) \right] \dd \nu(y). \nonumber
        \end{align}
        \item (Fubini) If $f \in L^1(\mu \times \nu)$, then $f_x \in L^1(\nu)$ for almost everywhere $x \in X$, $f^y \in L^1(\mu)$ for almost everywhere $y \in Y$, the almost everywhere defined functions $g(x) = \int f_x \dd \nu$ and $h(y) = \int f^y \dd \mu$ are in $L^1(\mu)$ and $L^1(\nu)$, respectively, and \refeq{Tonelli} holds.
    \end{enumerate}
\end{theorem}

\begin{theorem}[The Fubini-Tonelli Theorem for complete measures]
    
\end{theorem}

\section{The n-dimensional Lebesgue integral}

\section{Integration in polar coordinates}

