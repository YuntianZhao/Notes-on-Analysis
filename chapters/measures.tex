\setchapterstyle{kao}
\setchapterpreamble[u]{\margintoc}
\chapter{Measures}
\labch{measures}

\section{\texorpdfstring{$\sigma$}{sigma}-algebras}

\begin{definition}[Algebra]
    \labdef{algebra}
    Let $X$ be a nonempty set.
    An algebra of sets on $X$ is a nonempty collection $\mathcal{A}$ of subsets of $X$ that is closed under finite unions and complements.
\end{definition}

\begin{definition}[\texorpdfstring{$\sigma$}{sigma}-algebra]
    \labdef{sigma_algebra}
    A $\sigma$-algebra is an algebra that is closed under countable unions.
\end{definition}

Since $\bigcap_{j} E_j = (\bigcup_{j} E_j^c)^c$, algebras (respectively, $\sigma$-algebra) are also closed under finite (respectively, countable) intersections.

\begin{example}
    If $X$ is any set, $\mathcal{P}(X)$ and $\{ \emptyset, X \}$ are $\sigma$-algebra.
\end{example}

\begin{example}
    If $X$ is uncountable, then
    \begin{align}
        \mathcal{A} = \{ E \subset X : E \text{ is countable or } E^{c} \text{ is countable} \}
    \end{align}
    is a $\sigma$-algebra, called the $\sigma$-algebra of countable or co-countable sets.
\end{example}

\begin{definition}
    \labdef{generated_sigma_algebra}
    If $\mathcal{E}$ is any subset of $\mathcal{P}(X)$, there is a unique smallest $\sigma$-algebra $\mathcal{M}(\mathcal{E})$ containing $\mathcal{E}$, namely, the intersection of all $\sigma$-algebras containing $\mathcal{E}$.
    $\mathcal{M}(\mathcal{E})$ is called the $\sigma$-algebra generated by $\mathcal{E}$.
\end{definition}

\begin{lemma}
    \lablemma{sigma_algebra_is_smallest}
    If $\mathcal{E} \subset \mathcal{M}(\mathcal{F})$, then $\mathcal{M}(\mathcal{E}) \subset \mathcal{M}(\mathcal{F})$.
\end{lemma}

\begin{proof}
    Because $\mathcal{M}(\mathcal{F})$ is a $\sigma$-algebra containing $\mathcal{E}$, it contains $\mathcal{M}(\mathcal{E})$.
\end{proof}

\begin{definition}[Borel \texorpdfstring{$\sigma$}{sigma}-algebra]
    If $X$ is any metric space, or more generally any topological space, the $\sigma$-algebra generated by the family of open sets in $X$ (or equivalently the family of closed sets in $X$) is called Borel $\sigma$-algebra on $X$ and is denoted by $\mathcal{B}_{X}$.
    Its members are called Borel sets.
\end{definition}

$\mathcal{B}_{X}$ includes open sets, closed sets, countable intersections of open sets (countable unions of open sets are open), countable unions of closed sets (countable intersections of closed sets are closed), and so forth. 

\begin{proposition}
    \labprop{generate_Borel_sigma_algebra}
    $\mathcal{B}_{\R}$ is generated by each of the followings:
    \begin{enumerate}
        \item the open intervals: $\mathcal{E}_1 = \{ (a, b): a < b \}$,
        \item the closed intervals: $\mathcal{E}_2 = \{ [a, b]: a < b \}$,
        \item the half-open intervals: $\mathcal{E}_3 = \{ (a, b]: a < b \}$ or $\mathcal{E}_4 = \{ [a, b): a < b \}$,
        \item the open rays: $\mathcal{E}_5 = \{ (a, \infty): a \in \R \}$ or $\mathcal{E}_6 = \{ (-\infty, a): a \in \R \}$,
        \item the closed rays: $\mathcal{E}_7 = \{ [a, \infty): a \in \R \}$ or $\mathcal{E}_8 = \{ (-\infty, a]: a \in \R \}$.
    \end{enumerate}
\end{proposition}

\begin{definition}[Product \texorpdfstring{$\sigma$}{sigma}-algebra]
    \labdef{product_sigma_algebra}
    Let $\{ X_{\alpha} \}_{\alpha \in A}$ be an indexed collection nonempty sets, $X = \prod_{\alpha \in A} X_{\alpha}$, and $\pi_{\alpha}: X \to X_{\alpha}$ the coordinated maps.
    If $\mathcal{M}_{\alpha}$ is a $\sigma$-algebra on $X_{\alpha}$ for each $\alpha$, the product $\sigma$-algebra on $X$ is the $\sigma$-algebra generated by
    \begin{align}
        \left\{ \pi_{\alpha}^{-1} (E_\alpha) : E_{\alpha} \in \mathcal{M}_{\alpha}, \alpha \in A \right\}.
    \end{align}
    We denote this $\sigma$-algebra by $\bigotimes_{\alpha \in A} \mathcal{M}_{\alpha}$.
\end{definition}

In other words, $\bigotimes_{\alpha \in A} \mathcal{M}_{\alpha} = \mathcal{M} \left( \left\{ \pi_{\alpha}^{-1} (E_\alpha) : E_{\alpha} \in \mathcal{M}_{\alpha}, \alpha \in A \right\} \right)$.

\begin{proposition}
    \labprop{countable_A_product_sigma_algebra}
    If $A$ is countable, then $\bigotimes_{\alpha \in A} \mathcal{M}_{\alpha}$ is the $\sigma$-algebra generated by $\left\{ \prod_{\alpha \in A} E_{\alpha} : E_{\alpha} \in \mathcal{M}_{\alpha} \right\}$.
\end{proposition}

\begin{proof}
    Let $\mathcal{E} = \left\{ \pi_{\alpha}^{-1} (E_\alpha) : E_{\alpha} \in \mathcal{M}_{\alpha}, \alpha \in A \right\}$ and $\mathcal{F} = \left\{ \prod_{\alpha \in A} E_{\alpha} : E_{\alpha} \in \mathcal{M}_{\alpha} \right\}$.
    We need to prove that $\mathcal{M}(\mathcal{E}) = \mathcal{M}(\mathcal{F})$.

    For any $\alpha \in A$, $\pi_{\alpha}^{-1} (E_{\alpha}) = \prod_{\beta \in A} E_{\beta}$ where $E_\beta = X_{\beta}$ for $\beta \ne \alpha$.
    Then $\mathcal{E} \subset \mathcal{F} \subset \mathcal{M}(\mathcal{F})$.
    By \reflemma{sigma_algebra_is_smallest}, $\mathcal{M} (\mathcal{E}) \subset \mathcal{M} (\mathcal{F})$.

    On the other hand, $\prod_{\alpha \in A} E_{\alpha} = \bigcap_{\alpha \in A} \pi_{\alpha}^{-1} (E_\alpha)$ where $\pi_{\alpha}^{-1} (E_\alpha) \in \mathcal{E}$.
    Any intersection of elements contained in $\mathcal{E}$ is included in $\mathcal{M}(\mathcal{E})$.
    Thus, $\mathcal{F} \subset \mathcal{M}(\mathcal{E})$.
    Again by \reflemma{sigma_algebra_is_smallest}, $\mathcal{M} (\mathcal{E}) \supset \mathcal{M} (\mathcal{F})$.
\end{proof}

\begin{proposition}
    \labprop{generate_product_sigma_algebra}
    Suppose that $\mathcal{M}_{\alpha}$ is generated by $\mathcal{E}_{\alpha}$, $\alpha \in A$.
    Then $\bigotimes_{\alpha \in A} \mathcal{M}_{\alpha}$ is generated by $\mathcal{F}_1 = \left\{ \pi_{\alpha}^{-1} (E_\alpha) : E_{\alpha} \in \mathcal{E}_{\alpha}, \alpha \in A \right\}$.
    If $A$ is countable and $X_\alpha \in \mathcal{E}_{\alpha}$ for all $\alpha$, then $\bigotimes_{\alpha \in A} \mathcal{M}_{\alpha}$ is generated by $\mathcal{F}_2 = \left\{ \prod_{\alpha \in A} E_{\alpha} : E_{\alpha} \in \mathcal{E}_{\alpha} \right\}$.
\end{proposition}

\begin{proof}
    As $\mathcal{F}_1 \subset \left\{ \pi_{\alpha}^{-1} (E_\alpha) : E_{\alpha} \in \mathcal{M}_{\alpha}, \alpha \in A \right\}$, by \reflemma{sigma_algebra_is_smallest}, $\mathcal{M}(\mathcal{F}_1) \subset \bigotimes_{\alpha \in A} \mathcal{M}_{\alpha} = \mathcal{M} \left( \left\{ \pi_{\alpha}^{-1} (E_\alpha) : E_{\alpha} \in \mathcal{M}_{\alpha}, \alpha \in A \right\} \right)$

    For each $\alpha$, $\left\{ E \subset X_\alpha: \pi_{\alpha}^{-1}(E) \in \mathcal{M}(\mathcal{F}_1) \right\}$ is a $\sigma$-algebra on $X_\alpha$ that contains $\mathcal{E}_\alpha$ and hence $\mathcal{M}_\alpha$.
    In other words, $\pi_{\alpha}^{-1}(E) \in \mathcal{M}(\mathcal{F}_1)$ for all $E \in \mathcal{M}_{\alpha}$, $\alpha \in A$, and hence $\bigotimes_{\alpha \in A} \mathcal{M}_{\alpha} \subset \mathcal{M}(\mathcal{F}_1)$.
    Therefore, $\bigotimes_{\alpha \in A} \mathcal{M}_{\alpha} = \mathcal{M}(\mathcal{F}_1)$.

    Recall $\bigotimes_{\alpha \in A} \mathcal{M}_{\alpha} = \mathcal{M}(\mathcal{F})$ where $\mathcal{F} = \left\{ \prod_{\alpha \in A} E_{\alpha} : E_{\alpha} \in \mathcal{M}_{\alpha} \right\}$ as in \refprop{countable_A_product_sigma_algebra}.
    Since $\mathcal{F}_2 \subset \mathcal{F} \subset \mathcal{M}(\mathcal{F})$, $\mathcal{M}(\mathcal{F}_2) \subset \mathcal{M}(\mathcal{F}) = \bigotimes_{\alpha \in A} \mathcal{M}_{\alpha} $ by \reflemma{sigma_algebra_is_smallest}.

    For each $\alpha$, $\left\{ E \subset X_\alpha: \prod_{\alpha \in A} E_\alpha \in \mathcal{M}(\mathcal{F}_2) \right\}$ is a $\sigma$-algebra on $X_\alpha$ that contains $\mathcal{E}_\alpha$ and hence $\mathcal{M}_\alpha$.
    In other words, $\prod_{\alpha \in A} E_\alpha \in \mathcal{M}(\mathcal{F}_2)$ for all $E_{\alpha} \in \mathcal{M}_{\alpha}$, and hence $\bigotimes_{\alpha \in A} \mathcal{M}_{\alpha} \subset \mathcal{M}(\mathcal{F}_2)$.
    Therefore, $\bigotimes_{\alpha \in A} \mathcal{M}_{\alpha} = \mathcal{M}(\mathcal{F}_2)$ if $A$ is countable.
\end{proof}

\begin{proposition}
    \labprop{product_Borel_sigma_algebra}
    Let $X_1, \dots, X_n$ be metric spaces and let $X = \prod_{j=1}^{n} X_j$ equipped with the product metric.
    Then $\bigotimes_{j=1}^{n} \mathcal{B}_{X_j} \subset \mathcal{B}_{X}$.
    If the $X_j$'s are separable, then $\bigotimes_{j=1}^{n} \mathcal{B}_{X_j} = \mathcal{B}_{X}$.
\end{proposition}

\begin{proof}

\end{proof}



\begin{corollary}
    $\bigotimes_{j=1}^{n} \mathcal{B}_{\R} = \mathcal{B}_{\R^n}$.
\end{corollary}

\begin{proof}

\end{proof}

\begin{definition}[Elementary Family]
    An elementary family is a collection $\mathcal{E}$ of subsets of $X$ such that
    \begin{enumerate}
        \item $\emptyset \in \mathcal{E}$;
        \item if $E, F \in \mathcal{E}$, then $E \cap F \in \mathcal{E}$;
        \item if $E \in \mathcal{E}$, then $E^c$ is a finite disjoint union of members of $\mathcal{E}$.
    \end{enumerate}
\end{definition}

\begin{proposition}
    If $\mathcal{E}$ is an elementary family, the collection $\mathcal{A}$ of finite disjoint unions of members of $\mathcal{E}$ is an algebra.
\end{proposition}

\begin{proof}
    If $A, B \in \mathcal{E}$, and $B^c = \bigcup_{j=1}^{J} C_j$ where $C_j \in \mathcal{E}$ and are disjoint, then $A \setminus B = \bigcup_{j=1}^{J} (A \cap C_j)$ where $(A \cap C_j) \in \mathcal{E}$ and are disjoint.
    So, $(A \setminus B) \in \mathcal{A}$.
    And then $A \cup B = (A \setminus B) \cup B$.
    Thus, $(A \cup B) \in \mathcal{A}$.
    It follows by induction that if $A_1, \dots, A_n \in \mathcal{E}$, then $\bigcup_{j=1}^{n} A_j \in \mathcal{A}$.
    So, $\mathcal{A}$ is closed under finite unions.

    To see $\mathcal{A}$ is closed under complements, suppose $A_1, \dots, A_n \in \mathcal{E}$ and $A_m^c = \bigcup_{j=1}^{J_m} B_{m}^{j}$ with $B_{m}^{j}$ are disjoint members of $\mathcal{E}$.
    Then
    \begin{align}
        \left( \bigcup_{m=1}^{n} A_m \right)^c &= \bigcap_{m=1}^{n} \left( \bigcup_{j=1}^{J_m} B_{m}^{j} \right) \\
        &= \bigcup_{(j_1,\dots,j_n) \in \prod_{m=1}^{n} J_{m}} \left( \bigcap_{m=1}^{n} B_{m}^{j_m} \right).
    \end{align}
    Since $\bigcap_{m=1}^{n} B_{m}^{j_m} \in \mathcal{E}$ and are disjoint, $\bigcup_{(j_1,\dots,j_n) \in \prod_{m=1}^{n} J_{m}} \left( \bigcap_{m=1}^{n} B_{m}^{j_m} \right) \in \mathcal{A}$.
    Together, we showed that $\mathcal{A}$ is closed under finite unions and complements.
\end{proof}

\section{Measures}

\begin{definition}[Measure]
    Let $X$ be a set equipped with a $\sigma$-algebra $\mathcal{M}$.
    A measure on $\mathcal{M}$ (or on $(X, \mathcal{M})$, or simply on $X$ if $\mathcal{M}$ is understood) is a function $\mu: \mathcal{M} \to [0, \infty]$ such that
    \begin{enumerate}
        \item $\mu(\emptyset) = 0$;
        \item if $\{ E_j \}_{j=1}^{\infty}$ is a sequence of disjoint sets in $\mathcal{M}$, then $\mu(\bigcup_{j=1}^{\infty} E_j) = \sum _{j=1}^{\infty} \mu(E_j)$.
    \end{enumerate}
\end{definition}

\begin{definition}[Finite measure]
    Let $(X, \mathcal{M}, \mu)$ be a measure space.
    If $\mu(X) < \infty$, $\mu$ is called finite.
\end{definition}

\begin{definition}[$\sigma$-finite measure]
    Let $(X, \mathcal{M}, \mu)$ be a measure space.
    If $X = \bigcup_{j=1}^{\infty} E_j$ where $E_j \in \mathcal{M}$ and $\mu(E_j) < \infty$ for all $j$, $\mu$ is called $\sigma$-finite.
\end{definition}

\begin{definition}[Semifinite measure]
    Let $(X, \mathcal{M}, \mu)$ be a measure space.
    If for each $E \in \mathcal{M}$ with $\mu(E) = \infty$ there exists $F \in \mathcal{M}$ with $F \subset E$ and $0 < \mu(F) < \infty$, $\mu$ is called semifinite.
\end{definition}

\begin{example}
    Let $x$ be an infinite set and $\mathcal{M} = \mathcal{P}(X)$.
    Define $\mu(E) = 0$ if $E$ is finite, $\mu(E) = \infty$ if $E$ is infinite.
    Then, $\mu$ is a finitely additive measure but not a measure.
\end{example}

\begin{theorem}
    \labthm{properties_measure}
    Let $(X, \mathcal{M}, \mu)$ be a measure space.
    \begin{enumerate}
        \item If $E, F \in \mathcal{M}$ and $E \subset F$, then $\mu(E) \le \mu(F)$.
        \item If $\{ E_{j} \}_{j=1}^{\infty} \subset \mathcal{M}$, then $\mu(\bigcup_{j=1}^{\infty} E_j) \le \sum _{j=1}^{\infty} \mu(E_j)$.
        \item If $\{ E_{j} \}_{j=1}^{\infty} \subset \mathcal{M}$ and $E_1 \subset E_2 \subset \cdots$, then $\mu(\bigcup_{j=1}^{\infty} E_j) = \lim_{j\to \infty} \mu(E_j)$.
        \item If $\{ E_{j} \}_{j=1}^{\infty} \subset \mathcal{M}$, $E_1 \supset E_2 \supset \cdots$, and $\mu(E_1) < \infty$, then $\mu(\bigcap_{j=1}^{\infty} E_j) = \lim_{j\to \infty} \mu(E_j)$.
    \end{enumerate}
\end{theorem}

\begin{proof}
    If $E \subset F$, then $\mu(F) = \mu(E) + \mu(F \setminus E) \ge \mu(E)$.

    Let $F_1 = E_1$, $F_j = E_j \setminus (\bigcup_{k=1}^{j-1} E_k)$.
    Then ${F_j}_{j=1}^{\infty}$ is disjoint and $\bigcup_{j=1}^{\infty} F_j = \bigcup_{j=1}^{\infty} E_j$.
    \begin{align}
        \mu \left( \bigcup_{j=1}^{\infty} E_j \right) = \mu \left( \bigcup_{j=1}^{\infty} F_j \right) = \sum_{j=1}^{\infty} \mu(F_j) \le \sum_{j=1}^{\infty} \mu(E_j).
    \end{align}

    Setting $E_0 = \emptyset$, we have
    \begin{align}
        \mu \left( \bigcup_{j=1}^{\infty} E_j \right) & = \mu \left( \bigcup_{j=1}^{\infty} E_j \setminus E_{j-1} \right) = \sum_{j=1}^{\infty} \mu(E_j \setminus E_{j-1}) \\
                                         & = \lim_{n \to \infty} \sum_{j=1}^{n} \mu(E_j \setminus E_{j-1}) = \lim_{n \to \infty} \mu(E_n).
    \end{align}

    Let $F_j = E_1 \setminus E_j$, then $F_1 \subset F_2 \subset \cdots$ and $\mu(\bigcup_{j=1}^{\infty} F_j) = \lim_{j\to \infty} \mu(F_j)$.
    Also notice that $\mu(E_1) = \mu(F_j) + \mu(E_j)$ and $\bigcup_{j=1}^{\infty} F_j = E_1 \setminus \left( \bigcap_{j=1}^{\infty} E_j \right)$.
    \begin{align}
        \mu(E_1) & = \mu \left( \bigcap_{j=1}^{\infty} E_j \right) + \mu \left(\bigcup_{j=1}^{\infty} F_j \right) \\
                 & = \mu \left( \bigcap_{j=1}^{\infty} E_j \right) + \lim_{j\to \infty} \mu(F_j)               \\
                 & = \mu \left( \bigcap_{j=1}^{\infty} E_j \right) + \lim_{j\to \infty} [\mu(E_1) - \mu(E_j)].
    \end{align}
    Since $\mu(E_1) < \infty$, we may subtract it from both sides to yield the desired result.
\end{proof}

\begin{definition}
    \labdef{null_set}
    If $(X, \mathcal{M}, \mu)$ is a measure space, a set $E \in \mathcal{M}$ such that $\mu(E) = 0$ is called a null set.
\end{definition}

\begin{definition}[Almost everywhere]
    \labdef{almost_everywhere}
    If a statement about points $x \in X$ is true except for $x$ in some null set, we say it is true almost everywhere (abbreviated a.e.) or for almost every $x$.
\end{definition}

\begin{definition}
    \labdef{complete_measure}
    A measure whose domain includes all subsets of null sets is called complete.
\end{definition}

\begin{theorem}
    \labthm{completion_of_measure}
    Suppose $(X, \mathcal{M}, \mu)$ is a measure space.
    Let $\mathcal{N} = \{ N \in \mathcal{M}: \mu(N) = 0 \}$ and $\Bar{\mathcal{M}} = \{ E \cup F: E \in \mathcal{M} \text{ and } F \subset N \text{ for some } N \in \mathcal{N} \}$.
    Then $\bar{\mathcal{M}}$ is a $\sigma$-algebra, and there is a unique extension $\bar{\mu}$ of $\mu$ to a complete measure on $\bar{\mathcal{M}}$.
\end{theorem}

\begin{proof}
    Since $\mathcal{M}$ and $\mathcal{N}$ are closed under countable unions, so is $\bar{\mathcal{M}}$.
    Next, we need to show that $\bar{\mathcal{M}}$ is closed under complements.
    If $E \cup F \in \bar{\mathcal{M}}$ where $E \in \mathcal{M}$ and $F \subset N \in \mathcal{N}$, we can assume that $E \cap N = \emptyset$.
    Then $E \cup F = (E \cup N) \cap (N^c \cup F)$, so $(E \cup F)^c = (E \cup N)^c \cup (N \setminus F)$ where $(E \cup N)^c \in \mathcal{M}$ and $(N \setminus F) \subset N$.
    Therefore, $(E \cup F)^c \in \bar{\mathcal{M}}$, and $\bar{\mathcal{M}}$ is a $\sigma$-algebra.

    TODO: Show that there is a unique extension $\bar{\mu}$ of $\mu$ to a complete measure on $\bar{\mathcal{M}}$.
\end{proof}

\begin{proposition}
    Every $\sigma$-finite measure is semifinite.
\end{proposition}

\begin{proof}
    Let $(X, \mathcal{M}, \mu)$ be a measure space.
    Suppose a $\sigma$-finite measure $\mu$ is not semifinite.
    So there exists $E \in \mathcal{M}$ such taht $\mu(E) = \infty$, and for any $F \in \mathcal{M}$ with $F \subset E$, we have $mu(F) \in \{ 0, \infty \}$.

    Since $\mu$ is $\sigma$-finite, there are $\{ E_j \}_{j=1}^{\infty} \subset \mathcal{M}$ such that $X = \bigcup_{j=1}^{\infty} E_j$ and $\mu(E_j) < \infty$, for any $j$.
    Then we have $E = \bigcup_{j=1}^{\infty} (E \cap E_j)$ and $\mu(E) \le \sum_{j=1}^{\infty} \mu(E \cap E_j)$ by subadditivity.
    Because $\mu(E \cap E_j) \le \mu(E_j) < \infty$ and $\mu(E \cap E_j) \in \{ 0, \infty \}$ as $(E \cap E_j) \subset E$, $\mu(E \cap E_j) = 0$.
    But then $\mu(E) = 0$, which leads to a contradiction.
\end{proof}

\begin{proposition}
    If $\mu_1, \dots, \mu_n$ are measures on $(X, \mathcal{M})$ and $a_1, \dots, a_n \in [0, \infty)$, then $\mu = \sum_{k=1}^{n} a_k \mu_k$ is a measure on $(X, \mathcal{M})$.
\end{proposition}

\begin{proposition}
    If $(X, \mathcal{M}, \mu)$ is a measure space and $E \in \mathcal{M}$, define $\mu_E(A) = \mu(A \cap E)$ for $A \in \mathcal{M}$.
    Then $\mu_E$ is a measure.
\end{proposition}

\begin{proposition}
    If $(X, \mathcal{M}, \mu)$ is a measure space and $\{ E_j \}_{j=1}^{\infty} \subset \mathcal{M}$, then $\mu(\liminf E_j) \le \liminf \mu(E_j)$.
    Also, $\mu(\limsup E_j) \ge \limsup \mu(E_j)$ provided that $\mu(\bigcup_{j=1}^{\infty} E_j) < \infty$.
\end{proposition}

\begin{proposition}
    If $\mu$ is a semifinite measure and $\mu(E) = \infty$, for any $C > 0$ there exists $F \subset E$ with $C < \mu(F) < \infty$.
\end{proposition}

\section{Outer measures}

\begin{definition}[Outer measure]
    \labdef{outer_measure}
    An outer measure on a nonempty set $X$ is a function $\mu^*: \mathcal{P}(X) \to [0, \infty]$ that satisfies
    \begin{enumerate}
        \item $\mu^*(\emptyset) = 0$;
        \item $\mu^*(A) \le \mu^*(B)$ if $A \subset B$;
        \item $\mu^*(\bigcup_{j=1}^{\infty} A_j) \le \sum_{j=1}^{\infty} \mu^*(A_j)$.
    \end{enumerate}
\end{definition}

The most common way to obtain outer measures is to start with a family $\mathcal{E}$ of elementary sets on which a notion of measure is defined (\ie, rectanles in the plane) and then to approximate arbitrary sets from the outside by countable unions of members of $\mathcal{E}$.

\begin{proposition}
    \labthm{covering_outer_measure}
    Let $\mathcal{E} \subset \mathcal{P}(X)$ and $\rho : \mathcal{E} \to [0, \infty]$ be such that $\emptyset \in \mathcal{E}$, $X \in \mathcal{E}$, and $\rho(\emptyset) = 0$.
    For any $A \subset X$, define
    \begin{align}
        \mu^*(A) = \inf \left\{ \sum_{j=1}^{\infty} \rho(E_j): E_j \in \mathcal{E} \text{ and } A \subset \bigcup _{j=1}^{\infty} E_j \right\}.
    \end{align}
    Then, $\mu^*$ is an outer measure.
\end{proposition}

\begin{proof}
    For any $A \subset X$, there exists $\{ E_j \}_{j=1}^{\infty} \subset \mathcal{E}$ such that $A \subset \bigcup_{j=1}^{\infty} E_j$ (take $E_j = X$ for all $j$) so the definition of $\mu^*$ makes sense.
    Obviously $\mu^*(\emptyset) = 0$ (take $E_j = \emptyset$ for all $j$), and $\mu^*(A) \le \mu^*(B)$ for $A \subset B$ because the set over which the infimum is taken in the definition of $\mu^*(A)$ includes the corresponding set in the definition of $\mu^*(B)$.
    To prove the countable subadditivity, suppose $\{ A_j \}_{j=1}^{\infty} \subset \mathcal{P}(X)$ and $\epsilon > 0$. For each $j$, there exists $\{ E_{j}^{k} \}_{k=1}^{\infty} \subset \mathcal{E}$ such that $A_j \subset \bigcup_{k=1}^{\infty} E_{j}^{k}$ and $\sum_{k=1}^{\infty} \rho(E_{j}^{k}) \le \mu^*(A) + \frac{\epsilon}{2^j}$. 
    But if $A = \bigcup_{j=1}^{\infty} A_j$, we have $A \subset \bigcup_{j,k} E_{j}^{k}$ and $\sum_{j,k} \rho(E_{j}^{k}) \le \sum_{j} \mu^*(A_j) + \epsilon$.
    Therefore $\mu^*(A) \le \mu^*(A_j) + \epsilon$.
    Since $\epsilon$ is arbitrary, we are done.
\end{proof}

\begin{definition}
    If $\mu^*$ is an outer measure on $X$, a set $A \subset X$ is called $\mu^*$-measurable if $\mu^*(E) = \mu^*(E \cap A) + \mu^*(E \cap A^c)$ for all $E \subset X$.
\end{definition}

The inequality $\mu^*(E) \le \mu^*(E \cap A) + \mu^*(E \cap A^c)$ holds for any $A$ and $E$ because of the subadditivity in \refdef{outer_measure}.
So to prove that $A$ is $\mu^*$-measurable, it suffices to prove the reverse inequality.
The latter is trivial if $\mu^*(E) = \infty$, so we see that $A$ is is $\mu^*$-measurable iff $\mu^*(E) \ge \mu^*(E \cap A) + \mu^*(E \cap A^c)$ for all $E \subset X$ such that $\mu^*(E) < \infty$.

\begin{theorem}[Caratheodory's Theorem]
    \labthm{Caratheodory}
    If $\mu^*$ is an outer measure on $X$, the collection $\mathcal{M}$ of $\mu^*$-measurable sets is a $\sigma$-algebra, and the restriction of $\mu^*$ to $\mathcal{M}$ is a complete measure.
\end{theorem}

\begin{proof}
    First, we observe that $\mathcal{M}$ is closed under complements since the definition of $\mu^*$-measurablility of $A$ is symmetric in $A$ and $A^c$.
    Next, if $A, B \in \mathcal{M}$ and $E \subset X$,
    \begin{align}
        \mu^*(E) &= \mu^*(E \cap A) + \mu^*(E \cap A^c) \\
        &= \mu^*(E \cap A \cap B) + \mu^*(E \cap A \cap B^c) + \mu^*(E \cap A^c \cap B) + \mu^*(E \cap A^c \cap B^c).
    \end{align}
    But $A\cup B = (A \cap B) \cup (A \cap B^c) \cup (A^c \cap B)$, so by subadditivity,
    \begin{align}
        \mu^*(E \cap A \cap B) + \mu^*(E \cap A \cap B^c) + \mu^*(E \cap A^c \cap B) \ge \mu^*(E \cap (A \cup B)),
    \end{align}
    and hence 
    \begin{align}
        \mu^*(E) \ge \mu^*(E \cap (A \cup B)) + \mu^*(E \cap (A \cup B)^c).
    \end{align}
    It follows that $A \cup B \in \mathcal{M}$, so $\mathcal{M}$ is an algebra.
    Moreover, if $A, B \in \mathcal{M}$ and $A \cap B = \emptyset$, 
    \begin{align}
        \mu^*(A \cup B) = \mu^*((A \cup B) \cap A) + \mu^*((A \cup B) \cap A^c) = \mu^*(A) + \mu^*(B),
    \end{align}
    so $\mu^*$ is finitely additive on $\mathcal{M}$.

    To show $\mathcal{M}$ is a $\sigma$-algebra, it will suffice to show that $\mathcal{M}$ is closed under countable disjoint unions.
    If $\mathcal{A}_{j=1}^{\infty}$ is a sequence of disjoint sets in $\mathcal{M}$, let $B_n = \bigcup_{j=1}^{n} A_j$ and $B = \bigcup_{j=1}^{\infty} A_j$.
    Then for any $E \subset X$,
    \begin{align}
        \mu^*(E) &= \mu^*(E \cap B_n \cap A_n) + \mu^*(E \cap B_n \cap A_n^c) \\
        &= \mu^*(E \cap A_n) + \mu^*(E \cap B_{n-1}),
    \end{align}
    so a simple induction shows that $\mu^*(E \cap B_n) = \sum_{j=1}^{n} \mu^*(E \cap A_j)$.
    Therefore,
    \begin{align}
        \mu^*(E) &= \mu^*(E \cap B_n) + \mu^*(E \cap B_n^c) \\
        &\ge \left[ \sum_{j=1}^{n} \mu^*(E \cap A_j) \right] + \mu^*(E \cap B_n^c),
    \end{align}
    and lettting $n \to \infty$, we obtain
    \begin{align}
        \mu^*(E) &\ge \left[ \sum_{j=1}^{\infty} \mu^*(E \cap A_j) \right] + \mu^*(E \cap B^c) \\
        &\ge \mu^*\left( \bigcup_{j=1}^{\infty} (E \cap A_j) \right) + \mu^*(E \cap B^c) \\
        &= \mu^*(E \cap B) + \mu^*(E \cap B^c) \\
        &\ge \mu^*(E).
    \end{align}
    All the inequalities in this last calculation are thus equalities.
    If follows that $B \in \mathcal{M}$ and taking $E = B$ to obtain $\mu^*(B) = \sum_{j=1}^{\infty} \mu^*(A_j)$, so $\mu^*$ is countably additive on $\mathcal{M}$.

    Finally, if $\mu^*(A) = 0$, for any $E \subset X$ we have
    \begin{align}
        \mu^*(E) \le \mu^*(E \cap A) + \mu^*(E \cap A^c) = \mu^*(E \cap A^c) \le \mu^*(E),
    \end{align}
    so that $A \in \mathcal{M}$.
    Therefore, $\mu^*|\mathcal{M}$ is a complete measure.
\end{proof}

Our first applications of \refthm{Caratheodory} will be in the context of extending measures from algebras to $\sigma$-algebras.

\begin{definition}[Premeasure]
    \labdef{premeasure}
    If $\mathcal{A} \subset \mathcal{P}(X)$ is an algebra, a function $\mu_0: \mathcal{A} \to [0, \infty]$ will be called a premeasure if
    \begin{enumerate}
        \item $\mu_0(\emptyset) = 0$;
        \item if $\{ A_j \}_{j=1}^{\infty}$ is a sequence of disjoint sets in $\mathcal{A}$ such that $\bigcup_{j=1}^{\infty} A_j \in \mathcal{A}$, then $\mu_0(\bigcup_{j=1}^{\infty} A_j) = \sum _{j=1}^{\infty} \mu_0(A_j)$.
    \end{enumerate}
\end{definition}

In particular, a premeasure is finitely additive since one can take $A_j = \emptyset$ for $j$ large.
The notions of finite and $\sigma$-finite premeasure are defined just as for measures.
If $\mu_0$ is a premeasure on $\mathcal{A} \subset \mathcal{P}(X)$, it induces an outer measure on $X$ in accordance with \refthm{covering_outer_measure}, namely

\begin{align}
    \mu^*(E) = \inf \left\{ \sum_{j=1}^{\infty} \mu_0(A_j) : A_j \in \mathcal{A}, E \subset \bigcup_{j=1}^{\infty} A_j \right\} \labeq{premeasure_outer_measure}
\end{align}

\begin{proposition}
    \labprop{premeasure_outer_measure}
    If $\mu_0$ is a premeasure on $\mathcal{A}$ and $\mu^*$ is defined by \refeq{premeasure_outer_measure}, then
    \begin{enumerate}
        \item $\mu^*|\mathcal{A} = \mu_0$;
        \item every set in $\mathcal{A}$ is $\mu^*$-measurable.
    \end{enumerate}
\end{proposition}

\begin{proof}
    Suppose $E \in \mathcal{A}$. 
    If $E \subset \bigcup_{j=1}^{\infty} A_j$ with $A_j \in \mathcal{A}$, then let $B_n = E \cap (A_n \setminus \bigcup_{j=1}^{n-1} A_j)$.
    Then the $B_n$ are disjoint members of $\mathcal{A}$ whose union is $E$, so $\mu_0(E) = \sum_{j=1}^{\infty} \mu_0(B_j) \le \sum_{j=1}^{\infty} \mu_0(A_j)$.
    It follows that $\mu_0(E) \le \mu^*(E)$, and the reverse inequality is obvious since $E \subset \bigcup_{j=1}^{\infty} A_j$ where $A_1 = E$ and $A_j = \emptyset$ for $j > 1$.

    If $A \in \mathcal{A}$, $E \subset X$, and $\epsilon > 0$, there exists a sequence $\{ B_j \}_{j=1}^{\infty} \subset \mathcal{A}$ with $E \subset \bigcup_{j=1}^{\infty} B_j$ and $\sum_{j=1}^{\infty} \mu_0(B_j) \le \mu^*(E) + \epsilon$.
    Since $\mu_0$ is additive on $\mathcal{A}$,
    \begin{align}
        mu^*(E) + \epsilon &\ge \sum_{j=1}^{\infty} \mu_0(B_j) \\
        &= \sum_{j=1}^{\infty} \mu_0(B_j \cap A) + \sum_{j=1}^{\infty} \mu_0(B_j \cap A^c) \\
        &\ge \mu^*(E \cap A) + \mu^*(E \cap A^c).
    \end{align}
    Since $\epsilon$ is arbitrary, $A$ is $\mu^*$-measurable for every $A \in \mathcal{A}$.
\end{proof}

\begin{theorem}
    \labthm{premeasure_measure}
    Let $\mathcal{A} \subset \mathcal{P}(X)$ be an algebra, $\mu_0$ a premeasure on $\mathcal{A}$, and $\mathcal{M}$ the $\sigma$-algebra generated by $\mathcal{A}$.
    There exists a measure $\mu$ on $\mathcal{M}$ whose restriction to $\mathcal{A}$ is $\mu_0$ --- namely, $\mu = \mu^* | \mathcal{M}$ where $\mu^*$ is given by \refeq{premeasure_outer_measure}.
    If $\nu$ is another measure on $\mathcal{M}$ that extends $\mu_0$, then $\nu(E) \le \mu(E)$ for all $E \in \mathcal{M}$, with equality when $\mu(E) < \infty$.
    If $\mu_0$ is $\sigma$-finite, then $\mu$ is the unique extension of $\mu_0$ to a measure on $\mathcal{M}$.
\end{theorem}

\begin{proof}
    The first assertion follows from \refthm{Caratheodory} and \refprop{premeasure_outer_measure} since the $\sigma$-algebra of $\mu^*$-measurable sets includes $\mathcal{A}$ and hence $\mathcal{M}$.

    As for the second assertion, if $E \in \mathcal{M}$ and $E \subset \bigcup_{j} A_j$ where $A_j \in \mathcal{A}$, then $\nu(E) \le \sum_{j=1}^{\infty} \nu(A_j) = \sum_{j=1}^{\infty} \mu_0(A_j)$, hence $\nu(E) \le \mu(E)$.
    Also, if we set $A = \bigcup_{j} A_j$, we have
    \begin{align}
        \nu(A) = \lim_{n \to \infty} \nu\left( \bigcup_{j=1}^{n} A_j \right) = \lim_{n \to \infty} \mu\left( \bigcup_{j=1}^{n} A_j \right) = \mu(A).
    \end{align}
    If $\mu(E) < \infty$, we can choose the $\{A_j\}$ so that $\mu(A) < \mu(E) + \epsilon$, hence $\mu(A \setminus E) < \epsilon$, and 
    \begin{align}
        \mu(E) \le \mu(A) = \nu(A) = \nu(E) + \nu(A \setminus E) \le \nu(E) + \mu(A \setminus E) \le \nu(E) + \epsilon.
    \end{align}
    Since $\epsilon$ is arbitrary, $\nu(E) = \mu(E)$ when $\mu(E) < \infty$.

    Finally, suppose $X = \bigcup_{j=1}^{\infty} A_j$ with $\mu_0(A_j) < \infty$, where we can assume that $A_j$ are disjoint.
    Then for any $E \in \mathcal{M}$,
    \begin{align}
        \mu(E) = \sum_{j=1}^{\infty} \mu(E \cap A_j) = \sum_{j=1}^{\infty} \nu(E \cap A_j) = \nu(E),
    \end{align}
    so $\nu = \mu$.
\end{proof}

In this section, we have explored the relationship between measures, outer measures, and premeasures.



\begin{proposition}
    Let $(X, \mathcal{M}, \mu)$ be a measure space, $\mu^*$ the outer measure induced by $\mu$ according to \refeq{premeasure_outer_measure}, $\mathcal{M}^*$ the $\sigma$-algebra of $\mu^*$-measurable sets, and $\bar{\mu} = \mu^*|\mathcal{M}^*$.
    \begin{enumerate}
        \item If $\mu$ is $\sigma$-finite, then $\bar{\mu}$ is the completion of $\mu$.
        \item In general, $\bar{\mu}$ is the saturation of the completion of $\mu$.
    \end{enumerate}
\end{proposition}

\begin{proof}
    TODO
\end{proof}

\section{Borel measures on the real line}

\begin{definition}[Distribution Function]
    \labdef{distribution_function}
    suppose that $\mu$ is a finite Borel measure on $\R$., and let $F(x) = \mu((-\infty, x])$, which is sometimes called the distribution function of $\mu$.
\end{definition}


In this section, we will construct a definitive theory for measuring subsets of $\R$ based on the idea that the measure of an interval is its length.
We begin with a more general construction that yields a large family of measures on $\R$ whose domain is the Borel $\sigma$-algebra $\mathcal{B}_{\R}$; such measures are called Borel measures on $\R$.

\begin{proposition}
    \labprop{F_to_premeasure}
    Let $F:\R \to \R$ be increasing and right continuous.
    If $(a_j, b_j]$ are disjoint h-intervals, let
    \begin{align}
        \mu_0 \left(\bigcup_{j=1}^{n} (a_j, b_j] \right) = \sum_{j=1}^{n} [F(b_j) - F(a_j)],
    \end{align}
    and let $\mu_0(\emptyset) = 0$.
    Then $\mu_0$ is a premeasure on the algebra $\mathcal{A}$.
\end{proposition}

\begin{proof}
    First, we must check htat $\mu_0$ is well defined, since elements of $\mathcal{A}$ can be represented in more than one way as disjoint unions of h-intervals.
    If $\{ (a_j, b_j] \}_{j=1}^{\infty}$ are disjoint and $\bigcup_{j=1}^{\infty} (a_j, b_j] = (a, b]$, then, after relabeling the index $j$, we must have $a = a_1 < b_1 = a_2 < b_2 = \dots < b_n = b$, so $\sum_{j=1}^{\infty} [F(b_j) - F(a_j)] = F(b) - F(a)$.
    More generally, if $\{ I_i \}_{i=1}^{n}$ and $\{ J_j \}_{j=1}^{m}$ are finite sequences of disjoint h-intervals such that $\bigcup_{i=1}^{n} I_i = \bigcup_{j=1}^{m} J_j$, this reasoning shows that $\sum_{i=1}^{n} \mu_0(I_i) = \sum_{i=1}^{n} \sum_{j=1}^{m} \mu_0(I_i \cap J_j) = \sum_{j=1}^{m} J_j$.
    Thus $\mu_0$ is well defined, and it is finitely additive by construction.

    It remains to show that if $\{I_j\}_{j=1}^{\infty}$ is a sequence of disjoint h-intervals with $\bigcup_{j=1}^{\infty} I_j \in \mathcal{A}$, then $\mu_0(\bigcup_{j=1}^{\infty} I_j) = \sum_{j=1}^{\infty} \mu_0(I_j)$.
    Since $\bigcup_{j=1}^{\infty} I_j$ is a finite union of h-intervals, the sequence $\{I_j\}_{j=1}^{\infty}$ can be partitioned into finitely many subsequences such that the union of the intervals in eqch subsequence is a single h-interval.
    By considering each subsequence separately and using the fintie additivity of $\mu_0$, we may assume that $\bigcup_{j=1}^{\infty} I_j$ is an h-intervals $I = (a, b]$.
    In this case, we have
    \begin{align}
        \mu_0(I) = \mu_0\left(\bigcup_{j=1}^{n} I_j\right) + \mu_0\left(I \setminus \bigcup_{j=1}^{\infty} I_j\right) \ge \mu_0\left(\bigcup_{j=1}^{n} I_j\right) = \sum_{j=1}^{n} \mu_0(I_j).
    \end{align}
    Letting $n \to \infty$, we obtain $\mu_0(I) \ge \sum_{j=1}^{\infty} \mu_0(I_j)$.

    To prove the reverse inequality, let us suppose first that $a$ and $b$ are finite, and fix $\epsilon > 0$. 
    Since $F$ is right continuous, there exists $\delta > 0$ such that $F(a+\delta) - F(a) < \epsilon$.
    And if $I_j = (a_j, b_j]$, for each $j$, there $\delta_j > 0$ such that $F(b_j+\delta_j) - F(b_j) < \frac{\epsilon}{2^j}$.
    The open intervals $(a_j, b_j + \delta_j)$ cover the compact set $[a+\delta, b]$, so there is a finite subcover.
    By discarding any $(a_j, b_j + \delta_j)$ that is contained in a larger one and relabeling the index $j$, we may assume that the interval $\{(a_j, b_j+\delta_j)\}_{j=1}^{N}$ cover $[a+\delta, b]$, and $b_j + \delta_j \in (a_{j=1}, b_{j+1}+\delta_{j+1})$ for $j = 1, \dots, N-1$.
    But then 
    \begin{align}
        \mu_0(I) & = F(b) - F(a) \\
        & < F(b) - F(a+\delta) + \epsilon \\
        & \le F(b_N + \delta_N) - F(a_1) + \epsilon \\
        & = F(b_N + \delta_N) - F(a_N) + \sum_{j=1}^{N-1} [F(a_{j+1}) - F(a_j)] + \epsilon \\
        & < F(b_N + \delta_N) - F(a_N) + \sum_{j=1}^{N-1} [F(b_j+\delta_j) - F(a_j)] + \epsilon \\
        & < \sum_{j=1}^{N} [F(b_j) + \frac{\epsilon}{2^j} - F(a_j)] + \epsilon \\
        & < \sum_{j=1}^{\infty} \mu(I_j) + 2 \epsilon.
    \end{align}
    Since $\epsilon$ is arbitrary, we are done when $a$ and $b$ are finite.

    If $a = -\infty$, for any $M < \infty$, the intervals $(a_j, b_j + \delta_j)$ cover $[-M, b]$, so the same reasoning gives $F(b) - F(-M) \le \sum_{j=1}^{\infty} \mu(I_j) + 2 \epsilon$, whereas if $b = \infty$, for any $M < \infty$, we likewise obtain $F(M) - F(a)\le \sum_{j=1}^{\infty} \mu(I_j) + 2 \epsilon$.
    The desired result then follows by lettting $\epsilon \to 0$ and $M \to \infty$.
\end{proof}

\begin{theorem}
    \labthm{F_to_Borel_measure}
    If $F: \R \to \R$ is any increasing, right continuous function, there is a unique Borel measure $\mu_F$ on $\R$ such that $\mu_F ((a,b]) = F(b) - F(A)$ for all $a, b$.
    If $G$ is another such function, we have $\mu_F = \mu_G$ iff $F-G$ is constant.
    Conversely, if $\mu$ is Borel measure on $\R$ that is finite on all bounded Borel sets and we define
    \begin{align}
        F(x) = \begin{cases}
            \mu((0, x]), & \text{ if } x > 0, \\
            0, & \text{ if } x = 0,\\
            -\mu((-x, 0]), & \text{ if } x < 0, \\
        \end{cases}
    \end{align}
    then $F$ is increasing and right continuous, and $\mu = \mu_F$.
\end{theorem}

\begin{proof}
    Each $F$ can induce a premeasure on $\mathcal{A}$ by \refprop{F_to_premeasure} where $\mathcal{A}$ contains h-intervals.
    And the induced premeasure is $\sigma$-finite since $\R = \bigcup_{j=-\infty}^{\infty}(j, j+1]$.
    The $\sigma$-algebra generated by $\mathcal{A}$ is $\mathcal{B}_{\R}$ (see \refprop{generate_Borel_sigma_algebra}).
    By \refthm{premeasure_measure}, there exists a unique\sidenote{The uniqueness is because that the induced premeasure is $\sigma$-finite.} extension of the induced premeasure on $\mathcal{B}_{\R}$.

    Obviously, if $F - G$ is constant, $F$ and $G$ induce the same premeasure and hence the same measure.
    Conversely, if $\mu_F = \mu_G$, then for $x > 0$, $F(x) - F(0) = \mu_F((0, x]) = \mu_G((0, x]) = G(x) - G(0)$.
    And likewise for $x < 0$. 
    So, for any $x$, $F(x) - G(x) = F(0) - G(0)$ is a constant.
    
    By monotonity of $\mu$, $F$ is increasing.
    By Continuity of $\mu$, $F$ is right continuous.
    It is evident that $\mu=\mu_F$ on $\mathcal{A}$, and hence $\mu=\mu_F$ on $\mathcal{B}_{\R}$ by the uniqueness of \refthm{premeasure_measure}.
\end{proof}

Third, \refthm{Caratheodory} gives, for each increasing and right continuous $F$, not only the Borel measure $\mu_F$ but a complete measure whose domain includes $\mathcal{B}_{\R}$.
In fact, $\bar{\mu}_F$ is just the completion of $\mu_F$, and its domain is strictly larger than $\mathcal{B}_{\R}$. 
We shall usually denote this complete measure also by $\mu_F$.
It is called the Lebesgue-Stieltjes measure associated to $F$.

Lebesgue-Stieltjes measures enjoy some useful regularity properties that we now investigate.
In this discussion we fix a complete Lebesgue-Stieltjes measure $\mu$ on $\R$ associated to the increasing, right continuous function $F$, and we denote by $\mathcal{M}_{\mu}$ the domain of $\mu$.
Thus, for any $E \in \mathcal{M}_{\mu}$, 
\begin{align}
    \mu(E) &= \inf \left\{ \sum_{j=1}^{\infty} [F(b_j) - F(a_j)]: E \subset \bigcup_{j=1}^{\infty} (a_j, b_j] \right\} \\
    &= \inf \left\{ \sum_{j=1}^{\infty} \mu((a_j, b_j]): E \subset \bigcup_{j=1}^{\infty} (a_j, b_j] \right\}.
\end{align}

We first observe that we can replace h-intervals by open intervals.

\begin{lemma}
    \lablemma{measure_open_covering}
    For any $E \in \mathcal{M}_{\mu}$, 
    \begin{align}
        \mu(E) = \inf \left\{ \sum_{j=1}^{\infty} \mu((a_j, b_j)): E \subset \bigcup_{j=1}^{\infty} (a_j, b_j) \right\}.
    \end{align}
\end{lemma}

\begin{proof}
    Let $\nu(E) = \inf \left\{ \sum_{j=1}^{\infty} \mu ((a_j, b_j)): E \subset \bigcup_{j=1}^{\infty} (a_j, b_j) \right\}$.
    And suppose $E \subset \bigcup_{j} (a_j, b_j)$.
    $(a_j, b_j)$ is a countable disjoint union of h-intervals $\{ I_{j}^{k} \}_{k=1}^{\infty}$.
    Specifically, $I_{j}^{k} = (c_{j}^{k}, c_{j}^{k+1}]$ where $\{ c_{j}^{k} \}_{k=1}^{\infty}$ is any sequence such that $c_{j}^{1} = a_j$ and $c_{j}^{k}$ increases to $b_j$ as $k \to \infty$.
    Thus $E \subset \bigcup_{j,k} I_{j}^{k}$, so
    \begin{align}
        \sum_{j=1}^{\infty} \mu ((a_j, b_j)) = \sum_{j=1}^{\infty} \sum_{k=1}^{\infty} \mu(I_{j}^{k}) \ge \mu(E),
    \end{align}
    and hence $\nu(E) \ge \mu(E)$.

    On the other hand, given $\epsilon > 0$, there exists $\{ (a_j, b_j] \}_{j=1}^{\infty}$ with $E \subset \bigcup_{j} (a_j, b_j]$ and $\sum_{j=1}^{\infty} \mu((a_j, b_j]) \le \mu(E) + \epsilon$.
    And for each $j$, there exists $\delta_j > 0$ such that $F(b_j + \delta_j) - F(b_j) < \frac{\epsilon}{2^j}$ since $F$ is right continuous.
    Then $E \subset \bigcup_{j} (a_j, b_j + \delta_j)$ and 
    \begin{align}
        \sum_{j=1}^{\infty} \mu((a_j, b_j + \delta_j)) \le \sum_{j=1}^{\infty} \mu((a_j, b_j]) + \epsilon \le \mu(E) + 2 \epsilon,
    \end{align}
    so that $\nu(E) \le \mu(E)$.
\end{proof}

\begin{theorem}
    \labthm{measure_open_compact}
    If $E \in \mathcal{M}_{\mu}$, 
    \begin{align}
        \mu(E) &= \inf \left\{ \mu(U): U \supset E \text{ and } U \text{ is open} \right\} \\
        &= \sup \left\{ \mu(K): K \subset E \text{ and } K \text{ is compact} \right\}.
    \end{align}
\end{theorem}

\begin{proof}
    By \reflemma{measure_open_covering}, for any $\epsilon > 0$, there exists intervals $(a_j, b_j)$ such that $E \subset \bigcup_{j=1}^{\infty} (a_j, b_j)$ and $\sum_{j=1}^{\infty} \mu((a_j, b_j)) \le \mu(E) + \epsilon$.
    If $U = \bigcup_{j=1}^{\infty} (a_j, b_j)$ then $U$ is open, $U \supset E$, and $\mu(U) \le \sum_{j=1}^{\infty} \mu((a_j, b_j)) \le \mu(E) + \epsilon$.
    On the other hand, $\mu(U) \ge \mu(E)$ whenever $U \supset E$, so the first equality is valid.

    For the second one, suppose first that $E$ is bounded.
    If $E$ is closed, then $E$ is compact and and th equality is obvious.
    Otherwise, given $\epsilon > 0$, we can choose an open $U \supset \bar{E} \setminus E$ such that $\mu(E) \le \mu(\bar{E} \setminus E) + \epsilon$.
    Let $K = \bar{E} \setminus U$.
    Then $K$ is compact, $K \subset E$, and
    \begin{align}
        \mu(K) & = \mu(E) - \mu(E \cap U) \\
        & = \mu(E) - [\mu(U) - \mu(U \setminus E)] \\
        & \ge \mu(E) - \mu(U) + \mu(\bar{E} \setminus E) \\
        & \ge \mu(E) + \epsilon.
    \end{align}
    If $E$ is unbounded, let $E_j = E \cap (j, j+1]$ are disjoint and bounded.
    By the preceeding argument, for any $\epsilon > 0$, there exists compact $K_j \subset E_j$ with $\mu(K_j) \ge \mu(E_j) - \frac{\epsilon}{2^{|j|}}$.
    Let $H_n = \bigcup_{j=-n}^{n} K_j$ where $K_j$ are disjoint.
    Then $H_n$ is compact, $H_n \subset E$ and 
    \begin{align}
        \mu(H_n) = \sum_{j=-n}^{n} \mu(K_j) \ge \sum_{j=-n}^{n} \left[ \mu(E_j) - \frac{\epsilon}{2^{|j|}} \right] = \mu(\bigcup_{j=-n}^{n} E_j) - 3\epsilon.   
    \end{align}
    Since $\mu(E) = \lim_{n \to \infty} \mu(\bigcup_{j=-n}^{n} E_j)$ by \refthm{properties_measure}, the result follows.
\end{proof}

\begin{theorem}
    \labthm{differ_by_0_measure_set}
    If $E \subset \R$, the followings are equivalent.
    \begin{enumerate}
        \item $E \in \mathcal{M}_{\mu}$.
        \item $E = V \setminus N_1$ where $V$ is a $G_{\delta}$ set and $\mu(N_1) = 0$.
        \item $E = H \cup N_2$ where $H$ is a $F_{\sigma}$ set and $\mu(N_2) = 0$.
    \end{enumerate}
\end{theorem}

\begin{proof}
    Obviously, 2 and 3 each imply 1 since $\mu$ is complete on $\mathcal{M}_{\mu}$.
\end{proof}

The significance of is that all Borel sets (or more generally all sets in $\mathcal{M}_{\mu}$) are of a reasonably simple form modulo sets of measure zero. 
This contrasts markedly with the machinations necessary to construct the Borel sets from the open sets when null sets are not excepted.
Another version of the idea that general measurable sets can be approximated by simple sets is contained in the following proposition.

\begin{proposition}
    \labprop{finite_open_interval}
    If $E \in \mathcal{M}_{\mu}$ and $\mu(E) < \infty$, then for every $\epsilon > 0$, there is a set $A$ that is finite union of open intervals such that $\mu(E \triangle A) < \epsilon$.
\end{proposition}

\begin{proof}
    By \refthm{measure_open_compact}, given $\epsilon > 0$, there exists a open set $U \supset E$ such that $\mu(U) < \mu(E) + \frac{\epsilon}{2}$.
    Thus $\mu(U \setminus E) = \mu(U) - \mu(E) < \frac{\epsilon}{2}$.

    As every open sets in $\R$ can be written as a countable union of disjoint open intervals\sidenote{See \refprop{open_set_in_R_countable_disjoint_union_of_open_intervals}.}, we have $U = \bigcup_{j=1}^{\infty} U_j$ where $U_j$ are disjoint, and 
    \begin{align}
        \sum_{j=1}^{\infty} \mu(U_j) = \mu(U) < \mu(E) + \frac{\epsilon}{2} < \infty.
    \end{align}
    The series $\sum_{j=1}^{\infty} \mu(U_j)$ converges, so there exists $N \in \N$ such that $\sum_{j=N+1}^{\infty} \mu(U_j) < \frac{\epsilon}{2}$.
    Let $A = \bigcup_{j=1}^{N} U_j \subset U$, then $\mu(U \setminus A) = \sum_{j=N+1}^{\infty} \mu(U_j) < \frac{\epsilon}{2}$, and
    \begin{align}
        \mu(E \triangle A) = \mu(E \setminus A) + \mu(A \setminus E) \le \mu(U \setminus A) + \mu(U \setminus E) < \epsilon.
    \end{align}
\end{proof}

We now examine the most important measure on $\R$, namely, Lebesgue measure: 
This is the complete measure $\mu_F$ associated to the function $F(x) = x$, for which the measure of an interval is simply its length. 
We shall denote it by $m$. 
The domain of $m$ is called the class of Lebesgue measurable sets, and we shall denote it by $\mathcal{L}$. 
We shall also refer to the restriction of $m$ to $\mathcal{B}_{\R}$ as Lebesgue measure.



Among the most most significant properties of Lebesgue measure are its invariance under translations and simple behavior under dilations.

\begin{definition}
    If $E \subset \R$ and $s, r \in \R$, we define
    \begin{align}
        E+s &= \left\{ x + s: x \in E \right\}, \\
        rE &= \left\{ r x : x \in E \right\}.
    \end{align}
\end{definition}

\begin{theorem}
    If $E \in \mathcal{L}$, then $E+s \in \mathcal{L}$ and $rE \in \mathcal{L}$ for all $s, r \in \R$.
    Moreover, $m(E+s) = m(E)$ and $m(r E) = |r| m(E)$.
\end{theorem}

The Lebesgue null sets include not only all countable sets but many sets having the cardinality of the continuum.
We now present the standard example, the Cantor set, which is also of interest for other reasons.

\begin{definition}[Cantor Set]
    \labdef{Cantor_set}
    The Cantor set $C$ is the set of all $x \in [0, 1]$ that have a base-3 expansion $x = \sum_{j=1}^{\infty} a_j 3^{-j}$ with $a_j \ne 1$ for all $j$.
\end{definition}

Thus $C$ is obtained from $[0, 1]$ by removing the open middle third $(\frac{1}{3}, \frac{2}{3})$, then removing the open middle thirds $(\frac{1}{9}, \frac{2}{9})$ and $(\frac{7}{9}, \frac{8}{9})$ of the two remaining intervals, and so forth. 
The basic properties of $C$ are summarized as follows.

\begin{proposition}
    Let $C$ be the Cantor set.
    \begin{enumerate}
        \item $C$ is compact, nowhere dense, and totally disconnected (\ie, the only connected subsets of C are single points). Moreover, $C$ has no isolated points.
        \item $m(C) = 0$.
        \item $\card(C) = \mathfrak{c}$.
    \end{enumerate}
\end{proposition}

\begin{proof}
    TODO: Show that $C$ is compact, nowhere dense, and totally disconnected (\ie, the only connected subsets of C are single points). Moreover, $C$ has no isolated points.

    $C$ is obtained from $[0, 1]$ by removing one interval of length $\frac{1}{3}$, two intervals of length $\frac{1}{9}$, and so forth.
    Thus,
    \begin{align}
        m(C) = 1 - \sum_{j=0}^{\infty} \frac{2^j}{3^{j+1}} = 1 - \frac{1}{3} \cdot \frac{1}{1 - \frac{2}{3}} = 0.
    \end{align}

    Lastly, suppose $x \in C$, so that  $x = \sum_{j=1}^{\infty} a_j 3^{-j}$ with $a_j = 0$ or $2$ for all $j$.
    Let $f(x) = \sum_{j=0}^{\infty} b_j 2^{-j}$ where $b_j = a_j/2$.
    The series defining $f(x)$ is the base-2 expansion of number in $[0, 1]$, and any number in $[0, 1]$ can be obtained in this way.
    Hence $f$ maps $C$ onto $[0, 1]$ and is bijective, so $\card(C) = \mathfrak{c}$.
\end{proof}