\setchapterstyle{kao}
\setchapterpreamble[u]{\margintoc}
\chapter{Measures}
\labch{measures}

\section{\texorpdfstring{$\sigma$}{sigma}-algebras}

\begin{definition}[Algebra]
Let $X$ be a nonempty set. 
An algebra of sets on $X$ is a nonempty collection $\mathcal{A}$ of subsets of $X$ that is closed under finite unions and complements.
\end{definition}

\begin{definition}[\texorpdfstring{$\sigma$}{sigma}-algebra]
A $\sigma$-algebra is an algebra that is closed under countable unions.
\end{definition}

Since $\bigcap_{j} E_j = (\bigcup_{j} E_j^c)^c$, algebras (respectively, $\sigma$-algebra) are also closed under finite (respectively, countable) intersections.

\begin{example}
If $X$ is any set, $\mathcal{P}(X)$ and $\{ \emptyset, X \}$ are $\sigma$-algebra.
\end{example}

\begin{example}
If $X$ is uncountable, then 
\begin{align}
\mathcal{A} = \{ E \subset X : E \text{ is countable or } E^{c} \text{ is countable} \}
\end{align}
is a $\sigma$-algebra, called the $\sigma$-algebra of countable or co-countable sets.
\end{example}

\begin{definition}
If $\mathcal{E}$ is any subset of $\mathcal{P}(X)$, there is a unique smallest $\sigma$-algebra $\mathcal{M}(\mathcal{E})$ containing $\mathcal{E}$, namely, the intersection of all $\sigma$-algebras containing $\mathcal{E}$. 
$\mathcal{M}(\mathcal{E})$ is called the $\sigma$-algebra generated by $\mathcal{E}$.
\end{definition}

\begin{lemma}
If $\mathcal{E} \subset \mathcal{M}(\mathcal{F})$, then $\mathcal{M}(\mathcal{E}) \subset \mathcal{M}(\mathcal{F})$.
\end{lemma}

\begin{proof}
Because $\mathcal{M}(\mathcal{F})$ is a $\sigma$-algebra containing $\mathcal{E}$, it contains $\mathcal{M}(\mathcal{E})$.
\end{proof}

\begin{definition}[Borel \texorpdfstring{$\sigma$}{sigma}-algebra]
If $X$ is any metric space, or more generally any topological space, the $\sigma$-algebra generated by the family of open sets in $X$ (or equivalently the family of closed sets in $X$) is called Borel $\sigma$-algebra on $X$ and is denoted by $\mathcal{B}_{X}$.
Its members are called Borel sets.
\end{definition}

\begin{definition}
Let $\{ X_{\alpha} \}_{\alpha \in A}$ be an indexed collection nonempty sets, $X = \prod_{\alpha \in A} X_{\alpha}$, and $\pi_{\alpha}: X \to X_{\alpha}$ the coordinated maps.
If $\mathcal{M}_{\alpha}$ is a $\sigma$-algebra on $X_{\alpha}$ for each $\alpha$, the product $\sigma$-algebra on $X$ is the $\sigma$-algebra generated by 
\begin{align}
\{ \pi_{\alpha}^{-1} (E_\alpha) : E_{\alpha} \in \mathcal{M}_{\alpha}, \alpha \in A \}.
\end{align}
We denote this $\sigma$-algebra by $\bigotimes_{\alpha \in A} \mathcal{M}_{\alpha}$.
\end{definition}

\begin{proposition}
If $A$ is countable, then $\bigotimes_{\alpha \in A} \mathcal{M}_{\alpha}$ is the $\sigma$-algebra generated by $\{ \prod_{\alpha \in A} E_{\alpha} : E_{\alpha} \in \mathcal{M}_{\alpha} \}$.
\end{proposition}

\begin{proof}
    
\end{proof}

\begin{proposition}
Suppose that $\mathcal{M}_{\alpha}$ is generated by $\mathcal{E}_{\alpha}$, $\alpha \in A$.
Then $\bigotimes_{\alpha \in A} \mathcal{M}_{\alpha}$ is generated by $\mathcal{F}_1 = \{ \pi_{\alpha}^{-1} (E_\alpha) : E_{\alpha} \in \mathcal{E}_{\alpha}, \alpha \in A \}$.
If $A$ is countable and $X_\alpha \in \mathcal{E}_{\alpha}$ for all $\alpha$, then $\bigotimes_{\alpha \in A} \mathcal{M}_{\alpha}$ is generated by $\mathcal{F}_2 = \{ \prod_{\alpha \in A} E_{\alpha} : E_{\alpha} \in \mathcal{E}_{\alpha}  \}$.
\end{proposition}

\begin{proof}
    
\end{proof}

\begin{proposition}
Let $X_1, \dots, X_n$ be metric spaces and let $X = \prod_{j=1}{n} X_j$ equipped with the product metric.
Then $\bigotimes_{j=1}^{n} \mathcal{B}_{X_j} \subset \mathcal{B}_{X}$.
If the $X_j$'s are separable, then $\bigotimes_{j=1}^{n} \mathcal{B}_{X_j} = \mathcal{B}_{X}$.
\end{proposition}

\begin{proof}
    
\end{proof}

\begin{corollary}
$\bigotimes_{j=1}^{n} \mathcal{B}_{\R} = \mathcal{B}_{\R^n}$.
\end{corollary}

\begin{proof}
    
\end{proof}

\begin{definition}[Elementary family]
An elementary family is a collection $\mathcal{E}$ of subsets of $X$ such that
\begin{enumerate}
    \item $\emptyset \in \mathcal{E}$;
    \item if $E, F \in \mathcal{E}$, then $E \cap F \in \mathcal{E}$;
    \item if $E \in \mathcal{E}$, then $E^c$ is a finite disjoint union of members of $\mathcal{E}$.
\end{enumerate}
\end{definition}

\begin{proposition}
If $\mathcal{E}$ is an elementary family, the collection $\mathcal{A}$ of finite disjoint unions of members of $\mathcal{E}$ is an algebra.
\end{proposition}

\section{Measures}

\begin{definition}[Measure]
Let $X$ be a set equipped with a $\sigma$-algebra $\mathcal{M}$.
A measure on $\mathcal{M}$ (or on $(X, \mathcal{M})$, or simply on $X$ if $\mathcal{M}$ is understood) is a function $\mu: \mathcal{M} \to [0, \infty]$ such that
\begin{enumerate}
    \item $\mu(\emptyset) = 0$;
    \item if $\{ E_j \}_{j=1}^{\infty}$ is a sequence of disjoint sets in $\mathcal{M}$, then $\mu(\bigcup_{j=1}^{\infty} E_j) = \sum _{j=1}^{\infty} \mu(E_j)$.
\end{enumerate}
\end{definition}

\begin{definition}[Finite measure]
Let $(X, \mathcal{M}, \mu)$ be a measure space.
If $\mu(X) < \infty$, $\mu$ is called finite.
\end{definition}

\begin{definition}[$\sigma$-finite measure]
Let $(X, \mathcal{M}, \mu)$ be a measure space.
If $X = \bigcup_{j=1}^{\infty} E_j$ where $E_j \in \mathcal{M}$ and $\mu(E_j) < \infty$ for all $j$, $\mu$ is called $\sigma$-finite.
\end{definition}

\begin{definition}[Semifinite measure]
Let $(X, \mathcal{M}, \mu)$ be a measure space.
If for each $E \in \mathcal{M}$ with $\mu(E) = \infty$ there exists $F \in \mathcal{M}$ with $F \subset E$ and $0 < \mu(F) < \infty$, $\mu$ is called semifinite.
\end{definition}

\begin{example}
Let $x$ be an infinite set and $\mathcal{M} = \mathcal{P}(X)$.
Define $\mu(E) = 0$ if $E$ is finite, $\mu(E) = \infty$ if $E$ is infinite.
Then, $\mu$ is a finitely additive measure but not a measure.
\end{example}

\begin{theorem}
Let $(X, \mathcal{M}, \mu)$ be a measure space.
\begin{enumerate}
    \item If $E, F \in \mathcal{M}$ and $E \subset F$, then $\mu(E) \le \mu(F)$.
    \item If $\{ E_{j} \}_{j=1}^{\infty} \subset \mathcal{M}$, then $\mu(\bigcup_{j=1}^{\infty} E_j) \le \sum _{j=1}^{\infty} \mu(E_j)$.
    \item If $\{ E_{j} \}_{j=1}^{\infty} \subset \mathcal{M}$ and $E_1 \subset E_2 \subset \cdots$, then $\mu(\bigcup_{j=1}^{\infty} E_j) = \lim_{j\to \infty} \mu(E_j)$.
    \item If $\{ E_{j} \}_{j=1}^{\infty} \subset \mathcal{M}$, $E_1 \supset E_2 \supset \cdots$, and $\mu(E_1) < \infty$, then $\mu(\bigcap_{j=1}^{\infty} E_j) = \lim_{j\to \infty} \mu(E_j)$.
\end{enumerate}
\end{theorem}

\begin{proof}
If $E \subset F$, then $\mu(F) = \mu(E) + \mu(F \setminus E) \ge \mu(E)$.

Let $F_1 = E_1$, $F_j = E_j \setminus (\bigcup_{k=1}^{j-1} E_k)$.
Then ${F_j}_{j=1}^{\infty}$ is disjoint and $\bigcup_{j=1}^{\infty} F_j = \bigcup_{j=1}^{\infty} E_j$.
\begin{align}
\mu( \bigcup_{j=1}^{\infty} E_j) = \mu( \bigcup_{j=1}^{\infty} F_j ) = \sum_{j=1}^{\infty} \mu(F_j) \le \sum_{j=1}^{\infty} \mu(E_j).
\end{align}

Setting $E_0 = \emptyset$, we have
\begin{align}
\mu( \bigcup_{j=1}^{\infty} E_j) &= \mu( \bigcup_{j=1}^{\infty} E_j \setminus E_{j-1}) = \sum_{j=1}^{\infty} \mu(E_j \setminus E_{j-1}) \\
&= \lim_{n \to \infty} \sum_{j=1}^{n} \mu(E_j \setminus E_{j-1}) = \lim_{n \to \infty} \mu(E_n).
\end{align}

Let $F_j = E_1 \setminus E_j$, then $F_1 \subset F_2 \subset \cdots$ and $\mu(\bigcup_{j=1}^{\infty} F_j) = \lim_{j\to \infty} \mu(F_j)$.
Also notice that $\mu(E_1) = \mu(F_j) + \mu(E_j)$ and $\bigcup_{j=1}^{\infty} F_j = E_1 \setminus \left( \bigcap_{j=1}^{\infty} E_j \right)$.
\begin{align}
\mu(E_1) &= \mu( \bigcap_{j=1}^{\infty} E_j ) + \mu(\bigcup_{j=1}^{\infty} F_j) \\
&= \mu( \bigcap_{j=1}^{\infty} E_j ) + \lim_{j\to \infty} \mu(F_j) \\
&= \mu( \bigcap_{j=1}^{\infty} E_j ) + \lim_{j\to \infty} [\mu(E_1) - \mu(E_j)].
\end{align}
Since $\mu(E_1) < \infty$, we may subtract it from both sides to yield the desired result. 
\end{proof}

\begin{definition}
If $(X, \mathcal{M}, \mu)$ is a measure space, a set $E \in \mathcal{M}$ such that $\mu(E) = 0$ is called a null set.
\end{definition}

\begin{definition}[Almost everywhere]
If a statement about points $x \in X$ is true except for $x$ in some null set, we say it is true almost everywhere (abbreviated a.e.) or for almost every $x$.
\end{definition}

\begin{definition}
A measure whose domain includes all subsets of null sets is called complete.
\end{definition}

\begin{theorem}
Suppose $(X, \mathcal{M}, \mu)$ is a measure space.
Let $\mathcal{N} = \{ N \in \mathcal{M}: \mu(N) = 0 \}$ and $\Bar{\mathcal{M}} = \{ E \cup F: E \in \mathcal{M} \text{ and } F \subset N \text{ for some } N \in \mathcal{N} \}$.
Then $\bar{\mathcal{M}}$ is a $\sigma$-algebra, and there is a unique extension $\bar{\mu}$ of $\mu$ to a complete measure on $\bar{\mathcal{M}}$.
\end{theorem}

\begin{proof}
Since $\mathcal{M}$ and $\mathcal{N}$ are closed under countable unions, so is $\bar{\mathcal{M}}$.
Next, we need to show that $\bar{\mathcal{M}}$ is closed under complements.
If $E \cup F \in \bar{\mathcal{M}}$ where $E \in \mathcal{M}$ and $F \subset N \in \mathcal{N}$, we can assume that $E \cap N = \emptyset$.
Then $E \cup F = (E \cup N) \cap (N^c \cup F)$, so $(E \cup F)^c = (E \cup N)^c \cup (N \setminus F)$ where $(E \cup N)^c \in \mathcal{M}$ and $(N \setminus F) \subset N$.
Therefore, $(E \cup F)^c \in \bar{\mathcal{M}}$, and $\bar{\mathcal{M}}$ is a $\sigma$-algebra.

TODO: Show that there is a unique extension $\bar{\mu}$ of $\mu$ to a complete measure on $\bar{\mathcal{M}}$.
\end{proof}

\begin{proposition}
Every $\sigma$-finite measure is semifinite.
\end{proposition}

\begin{proof}
Let $(X, \mathcal{M}, \mu)$ be a measure space.
Suppose a $\sigma$-finite measure $\mu$ is not semifinite.
So there exists $E \in \mathcal{M}$ such taht $\mu(E) = \infty$, and for any $F \in \mathcal{M}$ with $F \subset E$, we have $mu(F) \in \{ 0, \infty \}$.

Since $\mu$ is $\sigma$-finite, there are $\{ E_j \}_{j=1}^{\infty} \subset \mathcal{M}$ such that $X = \bigcup_{j=1}^{\infty} E_j$ and $\mu(E_j) < \infty$, for any $j$.
Then we have $E = \bigcup_{j=1}^{\infty} (E \cap E_j)$ and $\mu(E) \le \sum_{j=1}^{\infty} \mu(E \cap E_j)$ by subadditivity.
Because $\mu(E \cap E_j) \le \mu(E_j) < \infty$ and $\mu(E \cap E_j) \in \{ 0, \infty \}$ as $(E \cap E_j) \subset E$, $\mu(E \cap E_j) = 0$.
But then $\mu(E) = 0$, which leads to a contradiction.
\end{proof}

\begin{proposition}
If $\mu_1, \dots, \mu_n$ are measures on $(X, \mathcal{M})$ and $a_1, \dots, a_n \in [0, \infty)$, then $\mu = \sum_{k=1}^{n} a_k \mu_k$ is a measure on $(X, \mathcal{M})$.
\end{proposition}

\begin{proposition}
If $(X, \mathcal{M}, \mu)$ is a measure space and $E \in \mathcal{M}$, define $\mu_E(A) = \mu(A \cap E)$ for $A \in \mathcal{M}$.
Then $\mu_E$ is a measure. 
\end{proposition}

\begin{proposition}
If $(X, \mathcal{M}, \mu)$ is a measure space and $\{ E_j \}_{j=1}^{\infty} \subset \mathcal{M}$, then $\mu(\liminf E_j) \le \liminf \mu(E_j)$.
Also, $\mu(\limsup E_j) \ge \limsup \mu(E_j)$ provided that $\mu(\bigcup_{j=1}^{\infty} E_j) < \infty$.
\end{proposition}

\begin{proposition}
If $\mu$ is a semifinite measure and $\mu(E) = \infty$, for any $C > 0$ there exists $F \subset E$ with $C < \mu(F) < \infty$.
\end{proposition}

\section{Outer measures}

\begin{definition}[Outer measure]
\labdef{outer_measure}
An outer measure on a nonempty set $X$ is a function $\mu^*: \mathcal{P}(X) \to [0, \infty]$ that satisfies
\begin{enumerate}
    \item $\mu^*(\emptyset) = 0$;
    \item $\mu^*(A) \le \mu^*(B)$ if $A \subset B$;
    \item $\mu^*(\bigcup_{j=1}^{\infty} A_j) \le \sum_{j=1}^{\infty} \mu^*(A_j)$.
\end{enumerate}
\end{definition}

\begin{proposition}
Let $\mathcal{E} \subset \mathcal{P}(X)$ and $\rho : \mathcal{E} \to [0, \infty]$ be such that $\emptyset \in \mathcal{E}$, $X \in \mathcal{E}$, and $\rho(\emptyset) = 0$.
For any $A \subset X$, define
\begin{align}
\mu^*(A) = \inf \left\{ \sum_{j=1}^{\infty} \rho(E_j): E_j \in \mathcal{E} \text{ and } A \subset \bigcup _{j=1}^{\infty} E_j \right\}.
\end{align}
Then, $\mu^*$ is an outer measure.
\end{proposition}

\begin{proof}

\end{proof}

\begin{definition}
If $\mu^*$ is an outer measure on $X$, a set $A \subset X$ is called $\mu^*$-measurable if $\mu^*(E) = \mu^*(E \cap A) + \mu^*(E \cap A^c)$ for all $E \subset X$.
\end{definition}

The inequality $\mu^*(E) \le \mu^*(E \cap A) + \mu^*(E \cap A^c)$ holds for any $A$ and $E$ because of the subadditivity in \refdef{outer_measure}.
So to prove that $A$ is $\mu^*$-measurable, it suffices to prove the reverse inequality. 
The latter is trivial if $\mu^*(E) = \infty$, so we see that $A$ is is $\mu^*$-measurable iff $\mu^*(E) \ge \mu^*(E \cap A) + \mu^*(E \cap A^c)$ for all $E \subset X$ such that $\mu^*(E) < \infty$.

\begin{theorem}[Caratheodory's Theorem]
If $\mu^*$ is an outer measure on $X$, the collection $\mathcal{M}$ of $\mu^*$-measurable sets is a $\sigma$-algebra, and the restriction of $\mu^*$ to $\mathcal{M}$ is a complete measure.
\end{theorem}

\begin{proof}
TODO

Finally, if $\mu^*(A) = 0$, for any $E \subset X$ we have
\begin{align}
\mu^*(E) \le \mu^*(E \cap A) + \mu^*(E \cap A^c) = \mu^*(E \cap A^c) \le \mu^*(E),
\end{align}
so that $A \in \mathcal{M}$.
Therefore, $\mu^*|\mathcal{M}$ is a complete measure.
\end{proof}

\begin{definition}[Premeasure]
If $\mathcal{A} \subset \mathcal{P}(X)$ is an algebra, a function $\mu_0: \mathcal{A} \to [0, \infty]$ will be called a premeasure if
\begin{enumerate}
    \item $\mu_0(\emptyset) = 0$;
    \item if $\{ A_j \}_{j=1}^{\infty}$ is a sequence of disjoint sets in $\mathcal{A}$ such that $\bigcup_{j=1}^{\infty} A_j \in \mathcal{A}$, then $\mu_0(\bigcup_{j=1}^{\infty} A_j) = \sum _{j=1}^{\infty} \mu_0(A_j)$.
\end{enumerate}
\end{definition}

\begin{align}
\mu^*(E) = \inf \left\{ \sum_{j=1}^{\infty} \mu_0(A_j) : A_j \in \mathcal{A}, E \subset \bigcup_{j=1}^{\infty} A_j \right\}
\end{align}

\begin{proposition}
If $\mu_0$ is a premeasure on $\mathcal{A}$ and $\mu^*$ is defined by, then
\begin{enumerate}
    \item $\mu^*|\mathcal{A} = \mu_0$;
    \item every set in $\mathcal{A}$ is $\mu^*$-measurable.
\end{enumerate}
\end{proposition}

\begin{proof}
    
\end{proof}

\begin{theorem}

\end{theorem}

\begin{proof}
    
\end{proof}

\section{Borel measures on the real line}

In this section, we will construct a definitive theory for measuring subsets of $\R$ based on the idea that the measure of an interval is its length.
We begin with a more general construction that yields a large family of measures on $\R$ whose domain is the Borel $\sigma$-algebra $\mathcal{B}_{\R}$; such measures are called Borel measures on $\R$.

\begin{proposition}
Let $F:\R \to \R$ be increasing and right continuous.
If $(a_j, b_j]$ are disjoint h-intervals, let
\begin{align}
\mu_0 \left(\bigcup_{j=1}^{n} (a_j, b_j] \right) = \sum_{j=1}^{n} [F(b_j) - F(a_j)],
\end{align}
and let $\mu_0(\emptyset) = 0$.
Then $\mu_0$ is a premeasure on the algebra $\mathcal{A}$.
\end{proposition}

\begin{proof}

\end{proof}

