\setchapterstyle{kao}
\setchapterpreamble[u]{\margintoc}
\chapter{Measures}
\labch{measures}

\section{\texorpdfstring{$\sigma$}{sigma}-algebras}

\begin{definition}[Algebra]
    \labdef{algebra}
    Let $X$ be a nonempty set.
    An algebra of sets on $X$ is a nonempty collection $\mathcal{A}$ of subsets of $X$ that is closed under finite unions and complements.
\end{definition}

\begin{definition}[\texorpdfstring{$\sigma$}{sigma}-algebra]
    \labdef{sigma_algebra}
    A $\sigma$-algebra is an algebra that is closed under countable unions.
\end{definition}

Since $\bigcap_{j} E_j = (\bigcup_{j} E_j^c)^c$, algebras (respectively, $\sigma$-algebra) are also closed under finite (respectively, countable) intersections.

\begin{example}
    If $X$ is any set, $\mathcal{P}(X)$ and $\{ \emptyset, X \}$ are $\sigma$-algebra.
\end{example}

\begin{example}
    If $X$ is uncountable, then
    \begin{align}
        \mathcal{A} = \{ E \subset X : E \text{ is countable or } E^{c} \text{ is countable} \}
    \end{align}
    is a $\sigma$-algebra, called the $\sigma$-algebra of countable or co-countable sets.
\end{example}

\begin{definition}
    \labdef{generated_sigma_algebra}
    If $\mathcal{E}$ is any subset of $\mathcal{P}(X)$, there is a unique smallest $\sigma$-algebra $\mathcal{M}(\mathcal{E})$ containing $\mathcal{E}$, namely, the intersection of all $\sigma$-algebras containing $\mathcal{E}$.
    $\mathcal{M}(\mathcal{E})$ is called the $\sigma$-algebra generated by $\mathcal{E}$.
\end{definition}

\begin{lemma}
    \lablemma{sigma_algebra_is_smallest}
    If $\mathcal{E} \subset \mathcal{M}(\mathcal{F})$, then $\mathcal{M}(\mathcal{E}) \subset \mathcal{M}(\mathcal{F})$.
\end{lemma}

\begin{proof}
    Because $\mathcal{M}(\mathcal{F})$ is a $\sigma$-algebra containing $\mathcal{E}$, it contains $\mathcal{M}(\mathcal{E})$.
\end{proof}

\begin{definition}[Borel \texorpdfstring{$\sigma$}{sigma}-algebra]
    If $X$ is any metric space, or more generally any topological space, the $\sigma$-algebra generated by the family of open sets in $X$ (or equivalently the family of closed sets in $X$) is called Borel $\sigma$-algebra on $X$ and is denoted by $\mathcal{B}_{X}$.
    Its members are called Borel sets.
\end{definition}

$\mathcal{B}_{X}$ includes open sets, closed sets, countable intersections of open sets (countable unions of open sets are open), countable unions of closed sets (countable intersections of closed sets are closed), and so forth. 

\begin{proposition}
    $\mathcal{B}_{\R}$ is generated by each of the followings:
    \begin{enumerate}
        \item the open intervals: $\mathcal{E}_1 = \{ (a, b): a < b \}$,
        \item the closed intervals: $\mathcal{E}_2 = \{ [a, b]: a < b \}$,
        \item the half-open intervals: $\mathcal{E}_3 = \{ (a, b]: a < b \}$ or $\mathcal{E}_4 = \{ [a, b): a < b \}$,
        \item the open rays: $\mathcal{E}_5 = \{ (a, \infty): a \in \R \}$ or $\mathcal{E}_6 = \{ (-\infty, a): a \in \R \}$,
        \item the closed rays: $\mathcal{E}_7 = \{ [a, \infty): a \in \R \}$ or $\mathcal{E}_8 = \{ (-\infty, a]: a \in \R \}$.
    \end{enumerate}
\end{proposition}

\begin{definition}[Product \texorpdfstring{$\sigma$}{sigma}-algebra]
    \labdef{product_sigma_algebra}
    Let $\{ X_{\alpha} \}_{\alpha \in A}$ be an indexed collection nonempty sets, $X = \prod_{\alpha \in A} X_{\alpha}$, and $\pi_{\alpha}: X \to X_{\alpha}$ the coordinated maps.
    If $\mathcal{M}_{\alpha}$ is a $\sigma$-algebra on $X_{\alpha}$ for each $\alpha$, the product $\sigma$-algebra on $X$ is the $\sigma$-algebra generated by
    \begin{align}
        \left\{ \pi_{\alpha}^{-1} (E_\alpha) : E_{\alpha} \in \mathcal{M}_{\alpha}, \alpha \in A \right\}.
    \end{align}
    We denote this $\sigma$-algebra by $\bigotimes_{\alpha \in A} \mathcal{M}_{\alpha}$.
\end{definition}

In other words, $\bigotimes_{\alpha \in A} \mathcal{M}_{\alpha} = \mathcal{M} \left( \left\{ \pi_{\alpha}^{-1} (E_\alpha) : E_{\alpha} \in \mathcal{M}_{\alpha}, \alpha \in A \right\} \right)$.

\begin{proposition}
    \labprop{countable_A_product_sigma_algebra}
    If $A$ is countable, then $\bigotimes_{\alpha \in A} \mathcal{M}_{\alpha}$ is the $\sigma$-algebra generated by $\left\{ \prod_{\alpha \in A} E_{\alpha} : E_{\alpha} \in \mathcal{M}_{\alpha} \right\}$.
\end{proposition}

\begin{proof}
    Let $\mathcal{E} = \left\{ \pi_{\alpha}^{-1} (E_\alpha) : E_{\alpha} \in \mathcal{M}_{\alpha}, \alpha \in A \right\}$ and $\mathcal{F} = \left\{ \prod_{\alpha \in A} E_{\alpha} : E_{\alpha} \in \mathcal{M}_{\alpha} \right\}$.
    We need to prove that $\mathcal{M}(\mathcal{E}) = \mathcal{M}(\mathcal{F})$.

    For any $\alpha \in A$, $\pi_{\alpha}^{-1} (E_{\alpha}) = \prod_{\beta \in A} E_{\beta}$ where $E_\beta = X_{\beta}$ for $\beta \ne \alpha$.
    Then $\mathcal{E} \subset \mathcal{F} \subset \mathcal{M}(\mathcal{F})$.
    By \reflemma{sigma_algebra_is_smallest}, $\mathcal{M} (\mathcal{E}) \subset \mathcal{M} (\mathcal{F})$.

    On the other hand, $\prod_{\alpha \in A} E_{\alpha} = \bigcap_{\alpha \in A} \pi_{\alpha}^{-1} (E_\alpha)$ where $\pi_{\alpha}^{-1} (E_\alpha) \in \mathcal{E}$.
    Any intersection of elements contained in $\mathcal{E}$ is included in $\mathcal{M}(\mathcal{E})$.
    Thus, $\mathcal{F} \subset \mathcal{M}(\mathcal{E})$.
    Again by \reflemma{sigma_algebra_is_smallest}, $\mathcal{M} (\mathcal{E}) \supset \mathcal{M} (\mathcal{F})$.
\end{proof}

\begin{proposition}
    \labprop{generate_product_sigma_algebra}
    Suppose that $\mathcal{M}_{\alpha}$ is generated by $\mathcal{E}_{\alpha}$, $\alpha \in A$.
    Then $\bigotimes_{\alpha \in A} \mathcal{M}_{\alpha}$ is generated by $\mathcal{F}_1 = \left\{ \pi_{\alpha}^{-1} (E_\alpha) : E_{\alpha} \in \mathcal{E}_{\alpha}, \alpha \in A \right\}$.
    If $A$ is countable and $X_\alpha \in \mathcal{E}_{\alpha}$ for all $\alpha$, then $\bigotimes_{\alpha \in A} \mathcal{M}_{\alpha}$ is generated by $\mathcal{F}_2 = \left\{ \prod_{\alpha \in A} E_{\alpha} : E_{\alpha} \in \mathcal{E}_{\alpha} \right\}$.
\end{proposition}

\begin{proof}
    As $\mathcal{F}_1 \subset \left\{ \pi_{\alpha}^{-1} (E_\alpha) : E_{\alpha} \in \mathcal{M}_{\alpha}, \alpha \in A \right\}$, by \reflemma{sigma_algebra_is_smallest}, $\mathcal{M}(\mathcal{F}_1) \subset \bigotimes_{\alpha \in A} \mathcal{M}_{\alpha} = \mathcal{M} \left( \left\{ \pi_{\alpha}^{-1} (E_\alpha) : E_{\alpha} \in \mathcal{M}_{\alpha}, \alpha \in A \right\} \right)$

    For each $\alpha$, $\left\{ E \subset X_\alpha: \pi_{\alpha}^{-1}(E) \in \mathcal{M}(\mathcal{F}_1) \right\}$ is a $\sigma$-algebra on $X_\alpha$ that contains $\mathcal{E}_\alpha$ and hence $\mathcal{M}_\alpha$.
    In other words, $\pi_{\alpha}^{-1}(E) \in \mathcal{M}(\mathcal{F}_1)$ for all $E \in \mathcal{M}_{\alpha}$, $\alpha \in A$, and hence $\bigotimes_{\alpha \in A} \mathcal{M}_{\alpha} \subset \mathcal{M}(\mathcal{F}_1)$.
    Therefore, $\bigotimes_{\alpha \in A} \mathcal{M}_{\alpha} = \mathcal{M}(\mathcal{F}_1)$.

    Recall $\bigotimes_{\alpha \in A} \mathcal{M}_{\alpha} = \mathcal{M}(\mathcal{F})$ where $\mathcal{F} = \left\{ \prod_{\alpha \in A} E_{\alpha} : E_{\alpha} \in \mathcal{M}_{\alpha} \right\}$ as in \refprop{countable_A_product_sigma_algebra}.
    Since $\mathcal{F}_2 \subset \mathcal{F} \subset \mathcal{M}(\mathcal{F})$, $\mathcal{M}(\mathcal{F}_2) \subset \mathcal{M}(\mathcal{F}) = \bigotimes_{\alpha \in A} \mathcal{M}_{\alpha} $ by \reflemma{sigma_algebra_is_smallest}.

    For each $\alpha$, $\left\{ E \subset X_\alpha: \prod_{\alpha \in A} E_\alpha \in \mathcal{M}(\mathcal{F}_2) \right\}$ is a $\sigma$-algebra on $X_\alpha$ that contains $\mathcal{E}_\alpha$ and hence $\mathcal{M}_\alpha$.
    In other words, $\prod_{\alpha \in A} E_\alpha \in \mathcal{M}(\mathcal{F}_2)$ for all $E_{\alpha} \in \mathcal{M}_{\alpha}$, and hence $\bigotimes_{\alpha \in A} \mathcal{M}_{\alpha} \subset \mathcal{M}(\mathcal{F}_2)$.
    Therefore, $\bigotimes_{\alpha \in A} \mathcal{M}_{\alpha} = \mathcal{M}(\mathcal{F}_2)$ if $A$ is countable.
\end{proof}

\begin{proposition}
    Let $X_1, \dots, X_n$ be metric spaces and let $X = \prod_{j=1}^{n} X_j$ equipped with the product metric.
    Then $\bigotimes_{j=1}^{n} \mathcal{B}_{X_j} \subset \mathcal{B}_{X}$.
    If the $X_j$'s are separable, then $\bigotimes_{j=1}^{n} \mathcal{B}_{X_j} = \mathcal{B}_{X}$.
\end{proposition}

\begin{proof}

\end{proof}



\begin{corollary}
    $\bigotimes_{j=1}^{n} \mathcal{B}_{\R} = \mathcal{B}_{\R^n}$.
\end{corollary}

\begin{proof}

\end{proof}

\begin{definition}[Elementary Family]
    An elementary family is a collection $\mathcal{E}$ of subsets of $X$ such that
    \begin{enumerate}
        \item $\emptyset \in \mathcal{E}$;
        \item if $E, F \in \mathcal{E}$, then $E \cap F \in \mathcal{E}$;
        \item if $E \in \mathcal{E}$, then $E^c$ is a finite disjoint union of members of $\mathcal{E}$.
    \end{enumerate}
\end{definition}

\begin{proposition}
    If $\mathcal{E}$ is an elementary family, the collection $\mathcal{A}$ of finite disjoint unions of members of $\mathcal{E}$ is an algebra.
\end{proposition}

\begin{proof}
    If $A, B \in \mathcal{E}$, and $B^c = \bigcup_{j=1}^{J} C_j$ where $C_j \in \mathcal{E}$ and are disjoint, then $A \setminus B = \bigcup_{j=1}^{J} (A \cap C_j)$ where $(A \cap C_j) \in \mathcal{E}$ and are disjoint.
    So, $(A \setminus B) \in \mathcal{A}$.
    And then $A \cup B = (A \setminus B) \cup B$.
    Thus, $(A \cup B) \in \mathcal{A}$.
    It follows by induction that if $A_1, \dots, A_n \in \mathcal{E}$, then $\bigcup_{j=1}^{n} A_j \in \mathcal{A}$.
    So, $\mathcal{A}$ is closed under finite unions.

    To see $\mathcal{A}$ is closed under complements, suppose $A_1, \dots, A_n \in \mathcal{E}$ and $A_m^c = \bigcup_{j=1}^{J_m} B_{m}^{j}$ with $B_{m}^{j}$ are disjoint members of $\mathcal{E}$.
    Then
    \begin{align}
        \left( \bigcup_{m=1}^{n} A_m \right)^c &= \bigcap_{m=1}^{n} \left( \bigcup_{j=1}^{J_m} B_{m}^{j} \right) \\
        &= \bigcup_{(j_1,\dots,j_n) \in \prod_{m=1}^{n} J_{m}} \left( \bigcap_{m=1}^{n} B_{m}^{j_m} \right).
    \end{align}
    Since $\bigcap_{m=1}^{n} B_{m}^{j_m} \in \mathcal{E}$ and are disjoint, $\bigcup_{(j_1,\dots,j_n) \in \prod_{m=1}^{n} J_{m}} \left( \bigcap_{m=1}^{n} B_{m}^{j_m} \right) \in \mathcal{A}$.
    Together, we showed that $\mathcal{A}$ is closed under finite unions and complements.
\end{proof}

\section{Measures}

\begin{definition}[Measure]
    Let $X$ be a set equipped with a $\sigma$-algebra $\mathcal{M}$.
    A measure on $\mathcal{M}$ (or on $(X, \mathcal{M})$, or simply on $X$ if $\mathcal{M}$ is understood) is a function $\mu: \mathcal{M} \to [0, \infty]$ such that
    \begin{enumerate}
        \item $\mu(\emptyset) = 0$;
        \item if $\{ E_j \}_{j=1}^{\infty}$ is a sequence of disjoint sets in $\mathcal{M}$, then $\mu(\bigcup_{j=1}^{\infty} E_j) = \sum _{j=1}^{\infty} \mu(E_j)$.
    \end{enumerate}
\end{definition}

\begin{definition}[Finite measure]
    Let $(X, \mathcal{M}, \mu)$ be a measure space.
    If $\mu(X) < \infty$, $\mu$ is called finite.
\end{definition}

\begin{definition}[$\sigma$-finite measure]
    Let $(X, \mathcal{M}, \mu)$ be a measure space.
    If $X = \bigcup_{j=1}^{\infty} E_j$ where $E_j \in \mathcal{M}$ and $\mu(E_j) < \infty$ for all $j$, $\mu$ is called $\sigma$-finite.
\end{definition}

\begin{definition}[Semifinite measure]
    Let $(X, \mathcal{M}, \mu)$ be a measure space.
    If for each $E \in \mathcal{M}$ with $\mu(E) = \infty$ there exists $F \in \mathcal{M}$ with $F \subset E$ and $0 < \mu(F) < \infty$, $\mu$ is called semifinite.
\end{definition}

\begin{example}
    Let $x$ be an infinite set and $\mathcal{M} = \mathcal{P}(X)$.
    Define $\mu(E) = 0$ if $E$ is finite, $\mu(E) = \infty$ if $E$ is infinite.
    Then, $\mu$ is a finitely additive measure but not a measure.
\end{example}

\begin{theorem}
    Let $(X, \mathcal{M}, \mu)$ be a measure space.
    \begin{enumerate}
        \item If $E, F \in \mathcal{M}$ and $E \subset F$, then $\mu(E) \le \mu(F)$.
        \item If $\{ E_{j} \}_{j=1}^{\infty} \subset \mathcal{M}$, then $\mu(\bigcup_{j=1}^{\infty} E_j) \le \sum _{j=1}^{\infty} \mu(E_j)$.
        \item If $\{ E_{j} \}_{j=1}^{\infty} \subset \mathcal{M}$ and $E_1 \subset E_2 \subset \cdots$, then $\mu(\bigcup_{j=1}^{\infty} E_j) = \lim_{j\to \infty} \mu(E_j)$.
        \item If $\{ E_{j} \}_{j=1}^{\infty} \subset \mathcal{M}$, $E_1 \supset E_2 \supset \cdots$, and $\mu(E_1) < \infty$, then $\mu(\bigcap_{j=1}^{\infty} E_j) = \lim_{j\to \infty} \mu(E_j)$.
    \end{enumerate}
\end{theorem}

\begin{proof}
    If $E \subset F$, then $\mu(F) = \mu(E) + \mu(F \setminus E) \ge \mu(E)$.

    Let $F_1 = E_1$, $F_j = E_j \setminus (\bigcup_{k=1}^{j-1} E_k)$.
    Then ${F_j}_{j=1}^{\infty}$ is disjoint and $\bigcup_{j=1}^{\infty} F_j = \bigcup_{j=1}^{\infty} E_j$.
    \begin{align}
        \mu( \bigcup_{j=1}^{\infty} E_j) = \mu( \bigcup_{j=1}^{\infty} F_j ) = \sum_{j=1}^{\infty} \mu(F_j) \le \sum_{j=1}^{\infty} \mu(E_j).
    \end{align}

    Setting $E_0 = \emptyset$, we have
    \begin{align}
        \mu( \bigcup_{j=1}^{\infty} E_j) & = \mu( \bigcup_{j=1}^{\infty} E_j \setminus E_{j-1}) = \sum_{j=1}^{\infty} \mu(E_j \setminus E_{j-1}) \\
                                         & = \lim_{n \to \infty} \sum_{j=1}^{n} \mu(E_j \setminus E_{j-1}) = \lim_{n \to \infty} \mu(E_n).
    \end{align}

    Let $F_j = E_1 \setminus E_j$, then $F_1 \subset F_2 \subset \cdots$ and $\mu(\bigcup_{j=1}^{\infty} F_j) = \lim_{j\to \infty} \mu(F_j)$.
    Also notice that $\mu(E_1) = \mu(F_j) + \mu(E_j)$ and $\bigcup_{j=1}^{\infty} F_j = E_1 \setminus \left( \bigcap_{j=1}^{\infty} E_j \right)$.
    \begin{align}
        \mu(E_1) & = \mu( \bigcap_{j=1}^{\infty} E_j ) + \mu(\bigcup_{j=1}^{\infty} F_j)           \\
                 & = \mu( \bigcap_{j=1}^{\infty} E_j ) + \lim_{j\to \infty} \mu(F_j)               \\
                 & = \mu( \bigcap_{j=1}^{\infty} E_j ) + \lim_{j\to \infty} [\mu(E_1) - \mu(E_j)].
    \end{align}
    Since $\mu(E_1) < \infty$, we may subtract it from both sides to yield the desired result.
\end{proof}

\begin{definition}
    If $(X, \mathcal{M}, \mu)$ is a measure space, a set $E \in \mathcal{M}$ such that $\mu(E) = 0$ is called a null set.
\end{definition}

\begin{definition}[Almost everywhere]
    If a statement about points $x \in X$ is true except for $x$ in some null set, we say it is true almost everywhere (abbreviated a.e.) or for almost every $x$.
\end{definition}

\begin{definition}
    A measure whose domain includes all subsets of null sets is called complete.
\end{definition}

\begin{theorem}
    Suppose $(X, \mathcal{M}, \mu)$ is a measure space.
    Let $\mathcal{N} = \{ N \in \mathcal{M}: \mu(N) = 0 \}$ and $\Bar{\mathcal{M}} = \{ E \cup F: E \in \mathcal{M} \text{ and } F \subset N \text{ for some } N \in \mathcal{N} \}$.
    Then $\bar{\mathcal{M}}$ is a $\sigma$-algebra, and there is a unique extension $\bar{\mu}$ of $\mu$ to a complete measure on $\bar{\mathcal{M}}$.
\end{theorem}

\begin{proof}
    Since $\mathcal{M}$ and $\mathcal{N}$ are closed under countable unions, so is $\bar{\mathcal{M}}$.
    Next, we need to show that $\bar{\mathcal{M}}$ is closed under complements.
    If $E \cup F \in \bar{\mathcal{M}}$ where $E \in \mathcal{M}$ and $F \subset N \in \mathcal{N}$, we can assume that $E \cap N = \emptyset$.
    Then $E \cup F = (E \cup N) \cap (N^c \cup F)$, so $(E \cup F)^c = (E \cup N)^c \cup (N \setminus F)$ where $(E \cup N)^c \in \mathcal{M}$ and $(N \setminus F) \subset N$.
    Therefore, $(E \cup F)^c \in \bar{\mathcal{M}}$, and $\bar{\mathcal{M}}$ is a $\sigma$-algebra.

    TODO: Show that there is a unique extension $\bar{\mu}$ of $\mu$ to a complete measure on $\bar{\mathcal{M}}$.
\end{proof}

\begin{proposition}
    Every $\sigma$-finite measure is semifinite.
\end{proposition}

\begin{proof}
    Let $(X, \mathcal{M}, \mu)$ be a measure space.
    Suppose a $\sigma$-finite measure $\mu$ is not semifinite.
    So there exists $E \in \mathcal{M}$ such taht $\mu(E) = \infty$, and for any $F \in \mathcal{M}$ with $F \subset E$, we have $mu(F) \in \{ 0, \infty \}$.

    Since $\mu$ is $\sigma$-finite, there are $\{ E_j \}_{j=1}^{\infty} \subset \mathcal{M}$ such that $X = \bigcup_{j=1}^{\infty} E_j$ and $\mu(E_j) < \infty$, for any $j$.
    Then we have $E = \bigcup_{j=1}^{\infty} (E \cap E_j)$ and $\mu(E) \le \sum_{j=1}^{\infty} \mu(E \cap E_j)$ by subadditivity.
    Because $\mu(E \cap E_j) \le \mu(E_j) < \infty$ and $\mu(E \cap E_j) \in \{ 0, \infty \}$ as $(E \cap E_j) \subset E$, $\mu(E \cap E_j) = 0$.
    But then $\mu(E) = 0$, which leads to a contradiction.
\end{proof}

\begin{proposition}
    If $\mu_1, \dots, \mu_n$ are measures on $(X, \mathcal{M})$ and $a_1, \dots, a_n \in [0, \infty)$, then $\mu = \sum_{k=1}^{n} a_k \mu_k$ is a measure on $(X, \mathcal{M})$.
\end{proposition}

\begin{proposition}
    If $(X, \mathcal{M}, \mu)$ is a measure space and $E \in \mathcal{M}$, define $\mu_E(A) = \mu(A \cap E)$ for $A \in \mathcal{M}$.
    Then $\mu_E$ is a measure.
\end{proposition}

\begin{proposition}
    If $(X, \mathcal{M}, \mu)$ is a measure space and $\{ E_j \}_{j=1}^{\infty} \subset \mathcal{M}$, then $\mu(\liminf E_j) \le \liminf \mu(E_j)$.
    Also, $\mu(\limsup E_j) \ge \limsup \mu(E_j)$ provided that $\mu(\bigcup_{j=1}^{\infty} E_j) < \infty$.
\end{proposition}

\begin{proposition}
    If $\mu$ is a semifinite measure and $\mu(E) = \infty$, for any $C > 0$ there exists $F \subset E$ with $C < \mu(F) < \infty$.
\end{proposition}

\section{Outer measures}

\begin{definition}[Outer measure]
    \labdef{outer_measure}
    An outer measure on a nonempty set $X$ is a function $\mu^*: \mathcal{P}(X) \to [0, \infty]$ that satisfies
    \begin{enumerate}
        \item $\mu^*(\emptyset) = 0$;
        \item $\mu^*(A) \le \mu^*(B)$ if $A \subset B$;
        \item $\mu^*(\bigcup_{j=1}^{\infty} A_j) \le \sum_{j=1}^{\infty} \mu^*(A_j)$.
    \end{enumerate}
\end{definition}

The most common way to obtain outer measures is to start with a family $\mathcal{E}$ of elementary sets on which a notion of measure is defined (\ie, rectanles in the plane) and then to approximate arbitrary sets from the outside by countable unions of members of $\mathcal{E}$.

\begin{proposition}
    \labthm{covering_outer_measure}
    Let $\mathcal{E} \subset \mathcal{P}(X)$ and $\rho : \mathcal{E} \to [0, \infty]$ be such that $\emptyset \in \mathcal{E}$, $X \in \mathcal{E}$, and $\rho(\emptyset) = 0$.
    For any $A \subset X$, define
    \begin{align}
        \mu^*(A) = \inf \left\{ \sum_{j=1}^{\infty} \rho(E_j): E_j \in \mathcal{E} \text{ and } A \subset \bigcup _{j=1}^{\infty} E_j \right\}.
    \end{align}
    Then, $\mu^*$ is an outer measure.
\end{proposition}

\begin{proof}
    For any $A \subset X$, there exists $\{ E_j \}_{j=1}^{\infty} \subset \mathcal{E}$ such that $A \subset \bigcup_{j=1}^{\infty} E_j$ (take $E_j = X$ for all $j$) so the definition of $\mu^*$ makes sense.
    Obviously $\mu^*(\emptyset) = 0$ (take $E_j = \emptyset$ for all $j$), and $\mu^*(A) \le \mu^*(B)$ for $A \subset B$ because the set over which the infimum is taken in the definition of $\mu^*(A)$ includes the corresponding set in the definition of $\mu^*(B)$.
    To prove the countable subadditivity, suppose $\{ A_j \}_{j=1}^{\infty} \subset \mathcal{P}(X)$ and $\epsilon > 0$. For each $j$, there exists $\{ E_{j}^{k} \}_{k=1}^{\infty} \subset \mathcal{E}$ such that $A_j \subset \bigcup_{k=1}^{\infty} E_{j}^{k}$ and $\sum_{k=1}^{\infty} \rho(E_{j}^{k}) \le \mu^*(A) + \frac{\epsilon}{2^j}$. 
    But if $A = \bigcup_{j=1}^{\infty} A_j$, we have $A \subset \bigcup_{j,k} E_{j}^{k}$ and $\sum_{j,k} \rho(E_{j}^{k}) \le \sum_{j} \mu^*(A_j) + \epsilon$.
    Therefore $\mu^*(A) \le \mu^*(A_j) + \epsilon$.
    Sicne $\epsilon$ is arbitrary, we are done.
\end{proof}

\begin{definition}
    If $\mu^*$ is an outer measure on $X$, a set $A \subset X$ is called $\mu^*$-measurable if $\mu^*(E) = \mu^*(E \cap A) + \mu^*(E \cap A^c)$ for all $E \subset X$.
\end{definition}

The inequality $\mu^*(E) \le \mu^*(E \cap A) + \mu^*(E \cap A^c)$ holds for any $A$ and $E$ because of the subadditivity in \refdef{outer_measure}.
So to prove that $A$ is $\mu^*$-measurable, it suffices to prove the reverse inequality.
The latter is trivial if $\mu^*(E) = \infty$, so we see that $A$ is is $\mu^*$-measurable iff $\mu^*(E) \ge \mu^*(E \cap A) + \mu^*(E \cap A^c)$ for all $E \subset X$ such that $\mu^*(E) < \infty$.

\begin{theorem}[Caratheodory's Theorem]
    \labthm{Caratheodory}
    If $\mu^*$ is an outer measure on $X$, the collection $\mathcal{M}$ of $\mu^*$-measurable sets is a $\sigma$-algebra, and the restriction of $\mu^*$ to $\mathcal{M}$ is a complete measure.
\end{theorem}

\begin{proof}
    First, we observe that $\mathcal{M}$ is closed under complements since the definition of $\mu^*$-measurablility of $A$ is symmetric in $A$ and $A^c$.
    Next, if $A, B \in \mathcal{M}$ and $E \subset X$,
    \begin{align}
        \mu^*(E) &= \mu^*(E \cap A) + \mu^*(E \cap A^c) \\
        &= \mu^*(E \cap A \cap B) + \mu^*(E \cap A \cap B^c) + \mu^*(E \cap A^c \cap B) + \mu^*(E \cap A^c \cap B^c).
    \end{align}
    But $A\cup B = (A \cap B) \cup (A \cap B^c) \cup (A^c \cap B)$, so by subadditivity,
    \begin{align}
        \mu^*(E \cap A \cap B) + \mu^*(E \cap A \cap B^c) + \mu^*(E \cap A^c \cap B) \ge \mu^*(E \cap (A \cup B)),
    \end{align}
    and hence 
    \begin{align}
        \mu^*(E) \ge \mu^*(E \cap (A \cup B)) + \mu^*(E \cap (A \cup B)^c).
    \end{align}
    It follows that $A \cup B \in \mathcal{M}$, so $\mathcal{M}$ is an algebra.
    Moreover, if $A, B \in \mathcal{M}$ and $A \cap B = \emptyset$, 
    \begin{align}
        \mu^*(A \cup B) = \mu^*((A \cup B) \cap A) + \mu^*((A \cup B) \cap A^c) = \mu^*(A) + \mu^*(B),
    \end{align}
    so $\mu^*$ is finitely additive on $\mathcal{M}$.

    To show $\mathcal{M}$ is a $\sigma$-algebra, it will suffice to show that $\mathcal{M}$ is closed under countable disjoint unions.
    If $\mathcal{A}_{j=1}^{\infty}$ is a sequence of disjoint sets in $\mathcal{M}$, let $B_n = \bigcup_{j=1}^{n} A_j$ and $B = \bigcup_{j=1}^{\infty} A_j$.
    Then for any $E \subset X$,
    \begin{align}
        \mu^*(E) &= \mu^*(E \cap B_n \cap A_n) + \mu^*(E \cap B_n \cap A_n^c) \\
        &= \mu^*(E \cap A_n) + \mu^*(E \cap B_{n-1}),
    \end{align}
    so a simple induction shows that $\mu^*(E \cap B_n) = \sum_{j=1}^{n} \mu^*(E \cap A_j)$.
    Therefore,
    \begin{align}
        \mu^*(E) &= \mu^*(E \cap B_n) + \mu^*(E \cap B_n^c) \\
        &\ge \left[ \sum_{j=1}^{n} \mu^*(E \cap A_j) \right] + \mu^*(E \cap B_n^c),
    \end{align}
    and lettting $n \to \infty$, we obtain
    \begin{align}
        \mu^*(E) &\ge \left[ \sum_{j=1}^{\infty} \mu^*(E \cap A_j) \right] + \mu^*(E \cap B^c) \\
        &\ge \mu^*\left( \bigcup_{j=1}^{\infty} (E \cap A_j) \right) + \mu^*(E \cap B^c) \\
        &= \mu^*(E \cap B) + \mu^*(E \cap B^c) \\
        &\ge \mu^*(E).
    \end{align}
    All the inequalities in this last calculation are thus equalities.
    If follows that $B \in \mathcal{M}$ and taking $E = B$ to obtain $\mu^*(B) = \sum_{j=1}^{\infty} \mu^*(A_j)$, so $\mu^*$ is countably additive on $\mathcal{M}$.

    Finally, if $\mu^*(A) = 0$, for any $E \subset X$ we have
    \begin{align}
        \mu^*(E) \le \mu^*(E \cap A) + \mu^*(E \cap A^c) = \mu^*(E \cap A^c) \le \mu^*(E),
    \end{align}
    so that $A \in \mathcal{M}$.
    Therefore, $\mu^*|\mathcal{M}$ is a complete measure.
\end{proof}

Our first applications of \refthm{Caratheodory} will be in the context of extending measures from algebras to $\sigma$-algebras.

\begin{definition}[Premeasure]
    \labdef{premeasure}
    If $\mathcal{A} \subset \mathcal{P}(X)$ is an algebra, a function $\mu_0: \mathcal{A} \to [0, \infty]$ will be called a premeasure if
    \begin{enumerate}
        \item $\mu_0(\emptyset) = 0$;
        \item if $\{ A_j \}_{j=1}^{\infty}$ is a sequence of disjoint sets in $\mathcal{A}$ such that $\bigcup_{j=1}^{\infty} A_j \in \mathcal{A}$, then $\mu_0(\bigcup_{j=1}^{\infty} A_j) = \sum _{j=1}^{\infty} \mu_0(A_j)$.
    \end{enumerate}
\end{definition}

In particular, a premeasure is finitely additive since one can take $A_j = \emptyset$ for $j$ large.
The notions of finite and $\sigma$-finite premeasure are defined just as for measures.
If $\mu_0$ is a premeasure on $\mathcal{A} \subset \mathcal{P}(X)$, it induces an outer measure on $X$ in accordance with \refthm{covering_outer_measure}, namely

\begin{align}
    \mu^*(E) = \inf \left\{ \sum_{j=1}^{\infty} \mu_0(A_j) : A_j \in \mathcal{A}, E \subset \bigcup_{j=1}^{\infty} A_j \right\} \labeq{premeasure_outer_measure}
\end{align}

\begin{proposition}
    \labprop{premeasure_outer_measure}
    If $\mu_0$ is a premeasure on $\mathcal{A}$ and $\mu^*$ is defined by \refeq{premeasure_outer_measure}, then
    \begin{enumerate}
        \item $\mu^*|\mathcal{A} = \mu_0$;
        \item every set in $\mathcal{A}$ is $\mu^*$-measurable.
    \end{enumerate}
\end{proposition}

\begin{proof}
    Suppose $E \in \mathcal{A}$. 
    If $E \subset \bigcup_{j=1}^{\infty} A_j$ with $A_j \in \mathcal{A}$, then let $B_n = E \cap (A_n \setminus \bigcup_{j=1}^{n-1} A_j)$.
    Then the $B_n$ are disjoint members of $\mathcal{A}$ whose union is $E$, so $\mu_0(E) = \sum_{j=1}^{\infty} \mu_0(B_j) \le \sum_{j=1}^{\infty} \mu_0(A_j)$.
    It follows that $\mu_0(E) \le \mu^*(E)$, and the reverse inequality is obvious since $E \subset \bigcup_{j=1}^{\infty} A_j$ where $A_1 = E$ and $A_j = \emptyset$ for $j > 1$.

    If $A \in \mathcal{A}$, $E \subset X$, and $\epsilon > 0$, there exists a sequence $\{ B_j \}_{j=1}^{\infty} \subset \mathcal{A}$ with $E \subset \bigcup_{j=1}^{\infty} B_j$ and $\sum_{j=1}^{\infty} \mu_0(B_j) \le \mu^*(E) + \epsilon$.
    Since $\mu_0$ is additive on $\mathcal{A}$,
    \begin{align}
        mu^*(E) + \epsilon &\ge \sum_{j=1}^{\infty} \mu_0(B_j) \\
        &= \sum_{j=1}^{\infty} \mu_0(B_j \cap A) + \sum_{j=1}^{\infty} \mu_0(B_j \cap A^c) \\
        &\ge \mu^*(E \cap A) + \mu^*(E \cap A^c).
    \end{align}
    Since $\epsilon$ is arbitrary, $A$ is $\mu^*$-measurable for every $A \in \mathcal{A}$.
\end{proof}

\begin{theorem}
    \labthm{premeasure_measure}
    Let $\mathcal{A} \subset \mathcal{P}(X)$ be an algebra, $\mu_0$ a premeasure on $\mathcal{A}$, and $\mathcal{M}$ the $\sigma$-algebra generated by $\mathcal{A}$.
    There exists a measure $\mu$ on $\mathcal{M}$ whose restriction to $\mathcal{A}$ is $\mu_0$ --- namely, $\mu = \mu^* | \mathcal{M}$ where $\mu^*$ is given by \refeq{premeasure_outer_measure}.
    If $\nu$ is another measure on $\mathcal{M}$ that extends $\mu_0$, then $\nu(E) \le \mu(E)$ for all $E in \mathcal{M}$, with equality when $\mu(E) < \infty$.
    If $\mu_0$ is $\sigma$-finite, then $\mu$ is the unique extension of $\mu_0$ to a measure on $\mathcal{M}$.
\end{theorem}

\begin{proof}
    The first assertion follows from \refthm{Caratheodory} and \refprop{premeasure_outer_measure} since the $\sigma$-algebra of $\mu^*$-measurable sets includes $\mathcal{A}$ and hence $\mathcal{M}$.

    As for the second assertion, if $E \in \mathcal{M}$ and $E \subset \bigcup_{j} A_j$ where $A_j \in \mathcal{A}$, then $\nu(E) \le \sum_{j=1}^{\infty} \nu(A_j) = \sum_{j=1}^{\infty} \mu_0(A_j)$, hence $\nu(E) \le \mu(E)$.
    Also, if we set $A = \bigcup_{j} A_j$, we have
    \begin{align}
        \nu(A) = \lim_{n \to \infty} \nu\left( \bigcup_{j=1}^{n} A_j \right) = \lim_{n \to \infty} \mu\left( \bigcup_{j=1}^{n} A_j \right) = \mu(A).
    \end{align}
    If $\mu(E) < \infty$, we can choose the $\{A_j\}$ so that $\mu(A) < \mu(E) + \epsilon$, hence $\mu(A \setminus E) < \epsilon$, and 
    \begin{align}
        \mu(E) \le \mu(A) = \nu(A) = \nu(E) + \nu(A \setminus E) \le \nu(E) + \mu(A \setminus E) \le \nu(E) + \epsilon.
    \end{align}
    Since $\epsilon$ is arbitrary, $\nu(E) = \mu(E)$ when $\mu(E) < \infty$.

    Finally, suppose $X = \bigcup_{j=1}^{\infty} A_j$ with $\mu_0(A_j) < \infty$, where we can assume that $A_j$ are disjoint.
    Then for any $E \in \mathcal{M}$,
    \begin{align}
        \mu(E) = \sum_{j=1}^{\infty} \mu(E \cap A_j) = \sum_{j=1}^{\infty} \nu(E \cap A_j) = \nu(E),
    \end{align}
    so $\nu = \mu$.
\end{proof}

In this section, we have explored the relationship between measures, outer measures, and premeasures.

\section{Borel measures on the real line}

In this section, we will construct a definitive theory for measuring subsets of $\R$ based on the idea that the measure of an interval is its length.
We begin with a more general construction that yields a large family of measures on $\R$ whose domain is the Borel $\sigma$-algebra $\mathcal{B}_{\R}$; such measures are called Borel measures on $\R$.

\begin{proposition}
    Let $F:\R \to \R$ be increasing and right continuous.
    If $(a_j, b_j]$ are disjoint h-intervals, let
    \begin{align}
        \mu_0 \left(\bigcup_{j=1}^{n} (a_j, b_j] \right) = \sum_{j=1}^{n} [F(b_j) - F(a_j)],
    \end{align}
    and let $\mu_0(\emptyset) = 0$.
    Then $\mu_0$ is a premeasure on the algebra $\mathcal{A}$.
\end{proposition}

\begin{proof}

\end{proof}

\begin{theorem}
    If $F: \R \to \R$ is any increasing, right continuous function, there is a unique Borel measure $\mu_F$ on $\R$ such that $\mu_F ((a,b]) = F(b) - F(A)$ for all $a, b$.
    If $G$ is another such function, we have $\mu_F = \mu_G$ iff $F-G$ is constant.
    Conversely, if $\mu$ is Borel measure on $\R$ that is finite on all bounded Borel sets and we define
    \begin{align}
        F(x) = \begin{cases}
            \mu((0, x]), \text{ if } x > 0, \\
            0, \text{ if } x = 0,\\
            -\mu((-x, 0]), \text{ if } x < 0, \\
        \end{cases}
    \end{align}
    then $F$ is increasing and right continuous, and $\mu = \mu_F$.
\end{theorem}

\begin{proof}
    
\end{proof}

\begin{lemma}
    For any $E \in \mathcal{M}_{\mu}$, 
    \begin{align}
        \mu(E) = \inf \left\{ \sum_{j=1}^{\infty} \mu((a_j, b_j)): E \subset \bigcup_{j=1}^{\infty} (a_j, b_j) \right\}.
    \end{align}
\end{lemma}

\begin{proof}
    
\end{proof}
