\setchapterstyle{kao}

\setchapterpreamble[u]{\margintoc}
\chapter{Prologue}
\labch{prologue}

\section{The language of set theory}

\begin{definition}[Number systems]
    Our notation of the fundamental number systems is as follows: \\
    $\N$ the set of positive integers (not including zero) \\
    $\Z$ the set of integers \\
    $\Q$ the set of rational numbers \\
    $\R$ the set of real numbers \\
    $\C$ the set of complex numbers
\end{definition}

The family of all subsets of a set $X$ is denoted by:
\begin{align}
    \mathcal{P}(x) = \{ E : E \subset X \}
\end{align}

\begin{align}
    \bigcup_{E \in \mathcal{E}} = \{ x : x\in E \text{ for some } E \in \mathcal{E} \} \\
    \bigcap_{E \in \mathcal{E}} = \{ x : x\in E \text{ for all } E \in \mathcal{E} \}
\end{align}

\begin{align}
    \mathcal{E} = \{ E_{\alpha} : \alpha \in A \} = \{E_{\alpha}\}_{\alpha \in A}
\end{align}

If $E_{\alpha} \cap E_{\beta} = \empty$ whenever $\alpha \ne \beta$, the sets $E_{\alpha}$ are called disjoint.

The terms ``disjoint collection of sets'' and ``collection of disjoint sets'' are used interchangeably,

\begin{align}
    \limsup E_n = \bigcap_{k=1}^{\infty} \bigcup_{n=k}^{\infty} E_n \\
    \liminf E_n = \bigcup_{k=1}^{\infty} \bigcap_{n=k}^{\infty} E_n
\end{align}

\begin{align}
    E \setminus F = \{ x : x \in E \text{ and } x \notin F \}
\end{align}

\begin{align}
    E \triangle F = (E \setminus F) \cup (F \setminus E)
\end{align}

\begin{align}
    E^c = X \setminus E
\end{align}

deMorgan's Laws:
\begin{align}
    (\bigcup_{\alpha \in A} E_{\alpha})^{c} = \bigcap_{\alpha \in A} E_{\alpha}^{c} \\
    (\bigcap_{\alpha \in A} E_{\alpha})^{c} = \bigcup_{\alpha \in A} E_{\alpha}^{c}
\end{align}

If $X$ and $Y$ are sets, their Cartesian product $X \times Y$ is the set of all ordered pairs $(x, y)$ such that $x \in X$ and $y \in Y$.

A relation from $X$ to $Y$ is a subset of $X \times Y$.
If $X=Y$, we speak of a relation on $X$.

If $f: X \to Y$ and $g: Y \to Z$ are mappings, we denote by $g \circ f$ their composition $g \circ f: X \to Z$, $g \circ f (x) = g(f(x))$.

\begin{align}
    f(D) = \{ f(x) : x \in D \} \\
    f^{-1}(E) = \{ x : f(x) \in E \}
\end{align}

The map $f^{-1}: \mathcal{P}(Y) \to \mathcal{P}(X)$ commutes with unions, intersections, and complements:
\begin{align}
    f^{-1} (\bigcup_{\alpha \in A} E_{\alpha}) = \bigcup_{\alpha \in A} f^{-1}(E_{\alpha}) \\
    f^{-1} (\bigcap_{\alpha \in A} E_{\alpha}) = \bigcap_{\alpha \in A} f^{-1}(E_{\alpha}) \\
    f^{-1} (E^{c}) = (f^{-1}(E))^{c}
\end{align}

The direct image mapping $f: \mathcal{P}(X) \to \mathcal{P}(Y)$ commutes with unions, but in general not with intersections or complements.

If $f: X \to Y$ is a mapping, $X$ is the domain of $f$, and $f(X)$ is the range of $f$.
$f$ is injective if $f(x_1) = f(x_2)$ only when $x_1 = x_2$, surjective if $f(X) = Y$, and bijective if it is both injective and surjective.
If $f$ is bijective, it has an inverse $f^{-1}: Y \to X$ such that $f^{-1} \circ f$ and $f \circ f^{-1}$ are the identity mappings on $X$ and $Y$, respectively.
If $X \subset X$, we denote by $f|A$ the restriction of $f$ to $A$, $(f|A): A \to Y$, $(f|A)(x) = f(x)$ for $x \in A$.

A sequence in a set $X$ is a mapping from $N$ into $X$.

\section{Orderings}

\begin{definition}[Partial Ordering]
    A partial ordering on a nonempty set $X$ is a relation $R$ on $X$ with the following properties:
    \begin{enumerate}
        \item if $x R y$ and $y R z$ then $x R z$;
        \item if $x R y$ and $y R x$ then $x = y$;
        \item $x R x$ for all $x$.
    \end{enumerate}
\end{definition}

\begin{example}
    If $E$ is any set, then $\mathcal{P}(E)$ is partially ordered by inclusion.
\end{example}

\begin{definition}[Linear Ordering]
    If a partial ordering R also satisfies
    \begin{enumerate}
        \item if $x, y \in X$, then either $x R y$ or $y R x$,
    \end{enumerate}
    then $R$ is called a linear (or total) ordering.
\end{definition}

\begin{definition}[Order Isomorphism]
    \labdef{order_isomorphic}
    Two partially ordered sets $X$ and $Y$ are said to be order isomorphic if there is a bijection $f: X \to Y$ such that $x_1 \le x_2$ iff $f(x_1) \le f(x_2)$.
\end{definition}

\begin{definition}[Well Ordering]
    If $X$ is linearly ordered by $\le$ and every nonempty subset of $X$ has a minimal element, X is said to be well ordered by $\le$, and $\le$ is called a well ordering on $X$.   
\end{definition}

\begin{theorem}[The Hausdorff Maximal Principle]
    \labthm{Hausdorff}
    Every partially ordered set has a maximal linearly ordered subset.
\end{theorem}

This means that if $X$ is partially ordered by $\le$, there is a set $E \subset X$ that is linearly ordered by $\le$, such that no subset of $X$ that properly includes $E$ is linearly ordered by $\le$.

\begin{lemma}[Zorn's Lemma]
    \lablemma{Zorn}
    If $X$ is a partially ordered set and every linearly ordered subset of $X$ has an upper bound, then $X$ has a maximal element.
\end{lemma}

The Hausdorff Maximal Principle and Zorn's Lemma are equivalent.

Clearly, the \refthm{Hausdorff} implies \reflemma{Zorn}.
An upper bound for a maximal linearly ordered subset of $X$ is a maximal element of $X$.

It is also not difficult to see \reflemma{Zorn} implies \refthm{Hausdorff}. 
Let $X$ be an arbitrary partially ordered set and let $\mathcal{X}$ be the set of all linearly ordered subsets in $X$, ordered by inclusion.
Then $\mathcal{X}$ is also partially ordered.
We then need to show $\mathcal{X}$ satisfies the hypothesis of Zorn's Lemma.
Because suppose $\mathcal{C} \subset \mathcal{X}$ is linearly ordered, then $\bigcup_{C \in \mathcal{C}} C$ is linearly ordered in $X$ and upper bounded.
Thus, for any linearly ordered $\mathcal{C} \subset \mathcal{X}$, $\mathcal{C}$ is upper bounded.
Then, according to Zorn's Lemma, $\mathcal{X}$ has a maximal element which is a maximal linearly ordered subset of $X$.


\begin{theorem}[The well Ordering Principle]
    \labthm{well_ordering}
    Every nonempty set $X$ can be well ordered.
\end{theorem}

\begin{theorem}[The Axiom of Choice]
    \labthm{axiom_of_choice}
    If $\{ X_{\alpha} \}_{\alpha \in A}$ is a nonempty collection of nonempty sets, then $\prod_{\alpha \in A} X_{\alpha}$ is nonempty.
\end{theorem}

\begin{proof}
    Let $X = \bigcup_{\alpha \in A} X_\alpha$.
    So $X$ is nonempty and well ordered by \refthm{axiom_of_choice}, so we can pick a well ordering on $X$.
    For $\alpha \in A$, let $f(\alpha)$ be the minimal element of $X_\alpha$.
    Then $f \in \prod_{\alpha \in A} X_\alpha$.
\end{proof}

\begin{corollary}
    If $\{ X_{\alpha} \}_{\alpha \in A}$ is a disjoint collection of nonempty sets, there is a set $Y \subset \cup_{\alpha \in A} X_{\alpha}$ such that $Y \cap X_{\alpha}$ contains precisely one element for each $\alpha \in A$.
\end{corollary}

\begin{proof}
    Take $Y = f(A)$ where $f \in \prod_{\alpha \in A} X_\alpha$.
\end{proof}

\section{Cardinality}

\begin{definition}
    If $X$ and $Y$ are nonempty sets, we define $\card (X) \le \card (Y)$, $\card (X) = \card (Y)$, $\card (X) \ge \card (Y)$ to mean that there exists $f: X \to Y$ which is injective, bijective, or surjective, respectively.
\end{definition}

\begin{definition}
    A set $X$ is called countable (or denumerable) if $\card (X) \le \card (\N)$.
    And $\card (X) = n$ iff $\card (X) = \card (\{ 1, \dots, n \})$.
\end{definition}

\begin{proposition}
    If $X$ and $Y$ are countable, so is $X \times Y$.
\end{proposition}

\begin{proposition}
    If $A$ is countable and $X_{\alpha}$ is countable for every $\alpha \in A$, then $\cup_{\alpha \in A} X_{\alpha}$ is countable.
\end{proposition}

\begin{proposition}
    If $X$ is countably infinite, then $\card(X) = \card(\N)$.
\end{proposition}

\begin{proposition}
    $\Z$ and $\Q$ are countable.
\end{proposition}

\begin{definition}
    A set $X$ is said to have the carinality of the continuum if $\card(X) = \card(\R) = \mathfrak{c}$.
\end{definition}


\begin{proposition}
    $\card(\mathcal{P}(\N)) = \mathfrak{c}$.
\end{proposition}

\begin{corollary}
    If $\card(X) \ge \mathfrak{c}$, then $X$ is uncountable.
\end{corollary}

\section{More about well ordered set}

\begin{definition}
    If $x \in X$, the initial segment of $x$ is $I_x = \{ y \in X : y < x \}$.
\end{definition}

The principle of mathematical induction is equivlent to the fact that $\N$ is well ordered.
It can be extended to arbitrary well ordered sets as follows:

\begin{theorem}[The Principle of Transfinite Induction]
    Let $X$ be a well ordered set. 
    If $A$ is a subset of $X$ such that $x \in A$ whenever $I_x \subset A$, then $A=X$.
\end{theorem}

\begin{proposition}
    If $X$ is well ordered and $A \subset X$, then $\bigcup_{x \in A} I_x$ is either an initial segment or $X$ itself. 
\end{proposition}

\begin{proposition}
    If $X$ and $Y$ are well ordered, then either $X$ is order isomorphic to $Y$, or $X$ is order isomorphic to an initial segment in $Y$, or an initial segment in $X$ is order isomorphic to $Y$.
\end{proposition}

\begin{proposition}
    There is an uncountable well ordered set $\Omega$ such that $I_x$ is countable for each $x \in \Omega$.
    If $\Omega'$ is another set with the same properties, then $\Omega$ and $\Omega'$ are order isomorphic.
\end{proposition}

\begin{proposition}
    Every countable subset of $\Omega$ has an upper bound.
\end{proposition}

\section{The extended real number system}

\begin{definition}[Extended Real Number System]
    The extended real number system is $\bar{\R} = \R \cup \{ -\infty, \infty \}$, and to extend the usual ordering on $\R$ by declaring that $-\infty < x < \infty$ for all $x \in \R$.
    The arithmetical operations on $\R$ can be partially extended to $\bar{\R}$:
    \begin{align}
        x \pm \infty &= \pm \infty \quad (x \in \R), \\
        \infty + \infty &= \infty, \\
        -\infty -\infty &= -\infty, \\
        x \cdot (\pm \infty) &= \pm \infty \quad (x > 0), \\
        x \cdot (\pm \infty) &= \mp \infty \quad (x < 0).
    \end{align}
    We make no attempt to define $\infty - \infty$.
    And unless otherwise stated, 
    \begin{align}
        0 \cdot (\pm \infty) = 0.
    \end{align}
\end{definition}

\begin{definition}
    If $X$ is an arbitrary set (may be uncountable), and $f: X \to [0, \infty]$, we define $\sum_{x \in X} f(x)$ to be the supremum of its finite partial sums:
    \begin{align}
        \sum_{x \in X} f(x) = \sup \left\{ \sum_{x \in F} f(x): F \subset X, F \text{ is finite} \right\}.
    \end{align} 
\end{definition}

\begin{proposition}
    Given $f: X \to [0, \infty]$, let $A = {x: f(x) > 0}$.
    If $A$ is uncountable, then $\sum_{x \in X} f(x) = \infty$.
    If $A$ is countably infinite, then $\sum_{x \in X} f(x) = \sum_{n=1}^{\infty} f(g(n))$ where $g: \N \to A$ is any bijection and the sum on the right is an ordinary infinite series.
\end{proposition}

\begin{proof}
    Let $A = \bigcup_{n=1}^{\infty} A_n$ where $A_n = \{ f(x) > 1/n \}$.
    If $A$ is uncountable, then some $A_n$ must be uncountable by monotonity of $A_n$.
    And $\sum_{x \in F} f(x) > \frac{\card (F)}{n}$ for a finite subset $F$ of $A_n$.
    It follows that $\sum_{x \in X} f(x) = \infty$.

    If $A$ is countably infinite, $g: \N \to A$ is a bijection, and $B_N = g(\{ 1, \dots, N \})$, then every finite subset $F$ of $A$ is contained in some $B_N$.
    Hence, 
    \begin{align}
        \sum_{x \in F} f(x) \le \sum_{n=1}^{N} f(g(n)) \le \sum_{x \in X} f(x).
    \end{align}
    Taking supremum over $N$, we find
    \begin{align}
        \sum_{x \in F} f(x) \le \sum_{n=1}^{\infty} f(g(n)) \le \sum_{x \in X} f(x),
    \end{align}
    and then takng the supremum over $F$, we obtain the desired result. 
\end{proof}

\begin{proposition}
    Every open set in $\R$ is a countable disjoint union of open intervals.
\end{proposition}

\begin{proof}
    TODO
\end{proof}

\section{Metric spaces}

\begin{definition}[Metric space]
    \labdef{metric_space}
    A metric on a set $X$ is a function $\rho : X \times X \to [0, \infty)$ such that
    \begin{enumerate}
        \item $\rho(x, y) = 0$ iff $x = y$;
        \item $\rho(x, y) = \rho(y, x)$ for all $x, y \in X$;
        \item $\rho(x, z) \le \rho (x, y) + \rho(y, z)$ for all $x, y, z \in X$.
    \end{enumerate}
    A set equipped with a metric is called a metric space, denoted as $(X, \rho)$
\end{definition}

\begin{definition}
    Let $(X, \rho)$ be a metric space.
    If $x \in X$ and $r > 0$, then (open) ball of radius $r$ about $x$ is $B(r, x) = \left\{ y \in X: \rho(x, y) < r \right\}$.
\end{definition}

\begin{definition}[Open and Closed]
    Let $(X, \rho)$ be a metric space.
    A set $E \subset X$ is open if for every $x \in E$ there exists $r > 0$ such that $B(r, x) \subset E$.
    A set $E \subset X$ is closed if its complement is open. 
\end{definition}

\begin{definition}
    If $E \subset X$, the union of all open sets $U \subset E$ is the largest open set contained in $E$; it is called the interior of $E$ and denoted by $E^o$.
    The intersection of all closed sets $F \supset E$ is the smallest closed set containing $E$; it is called the closure of $E$ and is denoted by $\bar{E}$.
\end{definition}

\begin{definition}
    A set $E$ is dense in $X$ if $\bar{E} = X$, and nowhere dense if $E$ has empty interior. 
\end{definition}

\begin{definition}
    $X$ is called separable if it has a countable dense subset.
\end{definition}

\begin{example}
    $\Q^n$ is a countable dense subset of $\R^n$.
\end{example}

\begin{definition}
    A sequence $\{ x_n \}$ in $X$ converges to $x \in X$ (symbolically, $x_n \to x$ or $\lim x_n = x$) if $\lim _{n \to \infty} \rho(x_n, x) = 0$.
\end{definition}

\begin{definition}[Continuity]
    \labdef{continuity}
    If $(X_1, \rho_1)$ and $(X_2, \rho_2)$ are metric spaces, a map $f: X_1 \to X_2$ is called continuous at $x \in X$ if for every $\epsilon > 0$ there exists $\delta > 0$ such that $\rho_2(f(x), f(y)) < \epsilon$ whenever $\rho_1(x, y) < \delta$. 
    In other words, such that $f^{-1} \left( B(\epsilon, f(x)) \right) \supset B(\delta, x)$.
    The map $f$ is called continuous if it is continuous at each $x \in X_1$ and uniformly continuous if, in addition, the $\delta$ in the definition of continuity can be chosen independent of $x$.
\end{definition}

\begin{proposition}
    $f: X_1 \to X_2$ is continuous iff $f^{-1}(U)$ is open in $X_1$ for every open $U \subset X_2$.
\end{proposition}

\begin{proof}
    If the latter condition holds, then for every $x \in X$ and $\epsilon > 0$, then $f^{-1} \left( B(\epsilon, f(x)) \right)$ is open and contains $x$, so ti contains some ball about $x$.
    This means we can pick a $\delta > 0$ such that $B(\delta, x) \subset f^{-1}\left( B(\epsilon, f(x)) \right)$.
    So $f$ is continuous by \refdef{continuity}.

    Conversely, suppose $f$ is continuous, and $U$ is open in $X_2$.
    For each $y \in U$, there exists $\epsilon_y > 0$ such that $B(\epsilon_y, y) \subset U$, and for each $x \in f^{-1}(\{ y \})$, there exhists $\delta_x > 0$ such that 
    \begin{align}
        B(\delta_x, x) \subset f^{-1}\left( B(\epsilon, f(x)) \right) \subset f^{-1}(U).
    \end{align}
    Thus $f^{-1}(U) = \bigcup_{x \in f^{-1}(U)} B(\delta_x, x)$ is open.
\end{proof}

\begin{proposition}
    A closed subset of a complete metric space is complete, and a complete subset of an arbitrary metric space is closed.
\end{proposition}

\begin{proof}
    
\end{proof}

\begin{definition}
    A sequence $\{x_n\}$ in a metric space $(X, \rho)$ is called Cauchy if $\rho(x_n, x_m) \to 0$ as $n, m\to \infty$.
\end{definition}

\begin{theorem}
    If $E$ is a subset of the metric space $(X, \rho)$, the following are equivalent:
    \begin{enumerate}
        \item $E$ is complete and totally bounded.
        \item (The Bolzano-Weierstrass Property) Every sequence in $E$ has a subsequence that converges to a point of $E$.
        \item (The Heine-Borel Property) If $\{ V_{\alpha} \}_{\alpha \in A}$ is a cover of $E$ by open sets, there is a finite set $F \subset A$ such that $\{ V_{\alpha} \}_{\alpha \in F}$ covers $E$.
    \end{enumerate}
\end{theorem}

\begin{theorem}
    Every closed and bounded subset of $\R^n$ is compact.
\end{theorem}