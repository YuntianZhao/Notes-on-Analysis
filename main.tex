% kaobook
% LaTeX Template
% Version 1.3 (December 9, 2021)
%
% This template originates from:
% https://www.LaTeXTemplates.com
%
% For the latest template development version and to make contributions:
% https://github.com/fmarotta/kaobook
%
% Authors:
% Federico Marotta (federicomarotta@mail.com)
% Based on the doctoral thesis of Ken Arroyo Ohori (https://3d.bk.tudelft.nl/ken/en)
% and on the Tufte-LaTeX class.
% Modified for LaTeX Templates by Vel (vel@latextemplates.com)
%
% License:
% CC0 1.0 Universal (see included MANIFEST.md file)

% Notes on Optimal Control
% Yuntian Zhao

%----------------------------------------------------------------------------------------
%	PACKAGES AND OTHER DOCUMENT CONFIGURATIONS
%----------------------------------------------------------------------------------------

\documentclass[
	a4paper, % Page size (e.g., b5paper)
	fontsize=10pt, % Base font size
	twoside=false, % Use different layouts for even and odd pages (in particular, if twoside=true, the margin column will be always on the outside)
	%open=any, % If twoside=true, uncomment this to force new chapters to start on any page, not only on right (odd) pages
	%chapterentrydots=true, % Uncomment to output dots from the chapter name to the page number in the table of contents
	numbers=noenddot, % Comment to output dots after chapter numbers; the most common values for this option are: enddot, noenddot and auto (see the KOMAScript documentation for an in-depth explanation)
]{kaobook}

% Choose the language
\ifxetexorluatex
	\usepackage{polyglossia}
	\setmainlanguage{english}
\else
	\usepackage[english]{babel} % Load characters and hyphenation
\fi
\usepackage[english=american]{csquotes}	% English quotes
% Load packages for testing
\usepackage{blindtext}
% \usepackage{showframe} % Uncomment to show boxes around the text area, margin, header and footer
% \usepackage{showlabels} % Uncomment to output the content of \label commands to the document where they are used

% Load the bibliography package
\usepackage{kaobiblio}
\addbibresource{main.bib} % Bibliography file

% Load mathematical packages for theorems and related environments
\usepackage[framed=true]{kaotheorems}

% Load the package for hyperreferences
\usepackage{kaorefs}

\usepackage{snippets}

\graphicspath{{examples/documentation/images/}{images/}} % Paths in which to look for images

\makeindex[columns=2, title=Alphabetical Index, intoc] % Make LaTeX produce the files required to compile the index

\makeglossaries % Make LaTeX produce the files required to compile the glossary
% \newglossaryentry{computer}{
% 	name=computer,
% 	description={is a programmable machine that receives input, stores and manipulates data, and provides output in a useful format}
% }

% % Glossary entries (used in text with e.g. \acrfull{fpsLabel} or \acrshort{fpsLabel})
% \newacronym[longplural={Frames per Second}]{fpsLabel}{FPS}{Frame per Second}
% \newacronym[longplural={Tables of Contents}]{tocLabel}{TOC}{Table of Contents}

 % Include the glossary definitions

\makenomenclature % Make LaTeX produce the files required to compile the nomenclature

% Reset sidenote counter at chapters
%\counterwithin*{sidenote}{chapter}

%----------------------------------------------------------------------------------------

\begin{document}

%----------------------------------------------------------------------------------------
%	BOOK INFORMATION
%----------------------------------------------------------------------------------------

% \titlehead{The \texttt{kaobook} class}
% \subject{Use this document as a template}

\title[Optimization]{Notes on Optimization}
\subtitle{Static and Dynamic Optimization}

\author[Yuntian Zhao]{Yuntian Zhao\thanks{email: \texttt{1710313@i.pkuschool.edu.cn}}}

\date{\today}

% \publishers{An Awesome Publisher}

%----------------------------------------------------------------------------------------

\frontmatter % Denotes the start of the pre-document content, uses roman numerals

%----------------------------------------------------------------------------------------
%	OPENING PAGE
%----------------------------------------------------------------------------------------

% \makeatletter
% \extratitle{
% 	% In the title page, the title is vspaced by 9.5\baselineskip
% 	\vspace*{9\baselineskip}
% 	\vspace*{\parskip}
% 	\begin{center}
% 		% In the title page, \huge is set after the komafont for title
% 		\usekomafont{title}\huge\@title
% 	\end{center}
% }
% \makeatother

%----------------------------------------------------------------------------------------
%	COPYRIGHT PAGE
%----------------------------------------------------------------------------------------

% \makeatletter
% \uppertitleback{\@titlehead} % Header

% \lowertitleback{
% 	\textbf{Disclaimer}\\
% 	You can edit this page to suit your needs. For instance, here we have a no copyright statement, a colophon and some other information. This page is based on the corresponding page of Ken Arroyo Ohori's thesis, with minimal changes.
	
% 	\medskip
	
% 	\textbf{No copyright}\\
% 	\cczero\ This book is released into the public domain using the CC0 code. To the extent possible under law, I waive all copyright and related or neighbouring rights to this work.
	
% 	To view a copy of the CC0 code, visit: \\\url{http://creativecommons.org/publicdomain/zero/1.0/}
	
% 	\medskip
	
% 	\textbf{Colophon} \\
% 	This document was typeset with the help of \href{https://sourceforge.net/projects/koma-script/}{\KOMAScript} and \href{https://www.latex-project.org/}{\LaTeX} using the \href{https://github.com/fmarotta/kaobook/}{kaobook} class.
	
% 	The source code of this book is available at:\\\url{https://github.com/fmarotta/kaobook}
	
% 	(You are welcome to contribute!)
	
% 	\medskip
	
% 	\textbf{Publisher} \\
% 	First printed in May 2019 by \@publishers
% }
% \makeatother

%----------------------------------------------------------------------------------------
%	DEDICATION
%----------------------------------------------------------------------------------------

% \dedication{
% 	The harmony of the world is made manifest in Form and Number, and the heart and soul and all the poetry of Natural Philosophy are embodied in the concept of mathematical beauty.\\
% 	\flushright -- D'Arcy Wentworth Thompson
% }

%----------------------------------------------------------------------------------------
%	OUTPUT TITLE PAGE AND PREVIOUS
%----------------------------------------------------------------------------------------

% Note that \maketitle outputs the pages before here

\maketitle

%----------------------------------------------------------------------------------------
%	PREFACE
%----------------------------------------------------------------------------------------

\chapter*{Preface}
\addcontentsline{toc}{chapter}{Preface} % Add the preface to the table of contents as a chapter

Nothing to say here.

\begin{flushright}
\textit{Yuntian Zhao}
\end{flushright}

\index{preface}

%----------------------------------------------------------------------------------------
%	TABLE OF CONTENTS & LIST OF FIGURES/TABLES
%----------------------------------------------------------------------------------------

\begingroup % Local scope for the following commands

% Define the style for the TOC, LOF, and LOT
%\setstretch{1} % Uncomment to modify line spacing in the ToC
%\hypersetup{linkcolor=blue} % Uncomment to set the colour of links in the ToC
\setlength{\textheight}{230\hscale} % Manually adjust the height of the ToC pages

% Turn on compatibility mode for the etoc package
\etocstandarddisplaystyle % "toc display" as if etoc was not loaded
\etocstandardlines % "toc lines" as if etoc was not loaded

\tableofcontents % Output the table of contents

\listoffigures % Output the list of figures

% Comment both of the following lines to have the LOF and the LOT on different pages
\let\cleardoublepage\bigskip
\let\clearpage\bigskip

\listoftables % Output the list of tables

\endgroup

%----------------------------------------------------------------------------------------
%	MAIN BODY
%----------------------------------------------------------------------------------------

\mainmatter % Denotes the start of the main document content, resets page numbering and uses arabic numbers
\setchapterstyle{kao} % Choose the default chapter heading style

% \input{chapters/introduction.tex}

\pagelayout{wide} % No margins
\addpart{Real analysis}
\pagelayout{margin} % Restore margins

\setchapterstyle{kao}

\setchapterpreamble[u]{\margintoc}
\chapter{Prologue}
\labch{prologue}

\section{The language of set theory}

\begin{definition}[Number systems]
    Our notation of the fundamental number systems is as follows: \\
    $\N$ the set of positive integers (not including zero) \\
    $\Z$ the set of integers \\
    $\Q$ the set of rational numbers \\
    $\R$ the set of real numbers \\
    $\C$ the set of complex numbers
\end{definition}

The family of all subsets of a set $X$ is denoted by:
\begin{align}
    \mathcal{P}(x) = \{ E : E \subset X \}
\end{align}

\begin{align}
    \bigcup_{E \in \mathcal{E}} = \{ x : x\in E \text{ for some } E \in \mathcal{E} \} \\
    \bigcap_{E \in \mathcal{E}} = \{ x : x\in E \text{ for all } E \in \mathcal{E} \}
\end{align}

\begin{align}
    \mathcal{E} = \{ E_{\alpha} : \alpha \in A \} = \{E_{\alpha}\}_{\alpha \in A}
\end{align}

If $E_{\alpha} \cap E_{\beta} = \empty$ whenever $\alpha \ne \beta$, the sets $E_{\alpha}$ are called disjoint.

The terms ``disjoint collection of sets'' and ``collection of disjoint sets'' are used interchangeably,

\begin{align}
    \limsup E_n = \bigcap_{k=1}^{\infty} \bigcup_{n=k}^{\infty} E_n \\
    \liminf E_n = \bigcup_{k=1}^{\infty} \bigcap_{n=k}^{\infty} E_n
\end{align}

\begin{align}
    E \setminus F = \{ x : x \in E \text{ and } x \notin F \}
\end{align}

\begin{align}
    E \triangle F = (E \setminus F) \cup (F \setminus E)
\end{align}

\begin{align}
    E^c = X \setminus E
\end{align}

deMorgan's Laws:
\begin{align}
    (\bigcup_{\alpha \in A} E_{\alpha})^{c} = \bigcap_{\alpha \in A} E_{\alpha}^{c} \\
    (\bigcap_{\alpha \in A} E_{\alpha})^{c} = \bigcup_{\alpha \in A} E_{\alpha}^{c}
\end{align}

If $X$ and $Y$ are sets, their Cartesian product $X \times Y$ is the set of all ordered pairs $(x, y)$ such that $x \in X$ and $y \in Y$.

A relation from $X$ to $Y$ is a subset of $X \times Y$.
If $X=Y$, we speak of a relation on $X$.

If $f: X \to Y$ and $g: Y \to Z$ are mappings, we denote by $g \circ f$ their composition $g \circ f: X \to Z$, $g \circ f (x) = g(f(x))$.

\begin{align}
    f(D) = \{ f(x) : x \in D \} \\
    f^{-1}(E) = \{ x : f(x) \in E \}
\end{align}

The map $f^{-1}: \mathcal{P}(Y) \to \mathcal{P}(X)$ commutes with unions, intersections, and complements:
\begin{align}
    f^{-1} (\bigcup_{\alpha \in A} E_{\alpha}) = \bigcup_{\alpha \in A} f^{-1}(E_{\alpha}) \\
    f^{-1} (\bigcap_{\alpha \in A} E_{\alpha}) = \bigcap_{\alpha \in A} f^{-1}(E_{\alpha}) \\
    f^{-1} (E^{c}) = (f^{-1}(E))^{c}
\end{align}

The direct image mapping $f: \mathcal{P}(X) \to \mathcal{P}(Y)$ commutes with unions, but in general not with intersections or complements.

If $f: X \to Y$ is a mapping, $X$ is the domain of $f$, and $f(X)$ is the range of $f$.
$f$ is injective if $f(x_1) = f(x_2)$ only when $x_1 = x_2$, surjective if $f(X) = Y$, and bijective if it is both injective and surjective.
If $f$ is bijective, it has an inverse $f^{-1}: Y \to X$ such that $f^{-1} \circ f$ and $f \circ f^{-1}$ are the identity mappings on $X$ and $Y$, respectively.
If $X \subset X$, we denote by $f|A$ the restriction of $f$ to $A$, $(f|A): A \to Y$, $(f|A)(x) = f(x)$ for $x \in A$.

A sequence in a set $X$ is a mapping from $N$ into $X$.

\section{Orderings}

\begin{definition}[Partial Ordering]
    A partial ordering on a nonempty set $X$ is a relation $R$ on $X$ with the following properties:
    \begin{enumerate}
        \item if $x R y$ and $y R z$ then $x R z$;
        \item if $x R y$ and $y R x$ then $x = y$;
        \item $x R x$ for all $x$.
    \end{enumerate}
\end{definition}

\begin{example}
    If $E$ is any set, then $\mathcal{P}(E)$ is partially ordered by inclusion.
\end{example}

\begin{definition}[Linear Ordering]
    If a partial ordering R also satisfies
    \begin{enumerate}
        \item if $x, y \in X$, then either $x R y$ or $y R x$,
    \end{enumerate}
    then $R$ is called a linear (or total) ordering.
\end{definition}

\begin{definition}[Order Isomorphism]
    \labdef{order_isomorphic}
    Two partially ordered sets $X$ and $Y$ are said to be order isomorphic if there is a bijection $f: X \to Y$ such that $x_1 \le x_2$ iff $f(x_1) \le f(x_2)$.
\end{definition}

\begin{definition}[Well Ordering]
    If $X$ is linearly ordered by $\le$ and every nonempty subset of $X$ has a minimal element, X is said to be well ordered by $\le$, and $\le$ is called a well ordering on $X$.   
\end{definition}

\begin{theorem}[The Hausdorff Maximal Principle]
    \labthm{Hausdorff}
    Every partially ordered set has a maximal linearly ordered subset.
\end{theorem}

This means that if $X$ is partially ordered by $\le$, there is a set $E \subset X$ that is linearly ordered by $\le$, such that no subset of $X$ that properly includes $E$ is linearly ordered by $\le$.

\begin{lemma}[Zorn's Lemma]
    \lablemma{Zorn}
    If $X$ is a partially ordered set and every linearly ordered subset of $X$ has an upper bound, then $X$ has a maximal element.
\end{lemma}

The Hausdorff Maximal Principle and Zorn's Lemma are equivalent.

Clearly, the \refthm{Hausdorff} implies \reflemma{Zorn}.
An upper bound for a maximal linearly ordered subset of $X$ is a maximal element of $X$.

It is also not difficult to see \reflemma{Zorn} implies \refthm{Hausdorff}. 
Let $X$ be an arbitrary partially ordered set and let $\mathcal{X}$ be the set of all linearly ordered subsets in $X$, ordered by inclusion.
Then $\mathcal{X}$ is also partially ordered.
We then need to show $\mathcal{X}$ satisfies the hypothesis of Zorn's Lemma.
Because suppose $\mathcal{C} \subset \mathcal{X}$ is linearly ordered, then $\bigcup_{C \in \mathcal{C}} C$ is linearly ordered in $X$ and upper bounded.
Thus, for any linearly ordered $\mathcal{C} \subset \mathcal{X}$, $\mathcal{C}$ is upper bounded.
Then, according to Zorn's Lemma, $\mathcal{X}$ has a maximal element which is a maximal linearly ordered subset of $X$.


\begin{theorem}[The well Ordering Principle]
    \labthm{well_ordering}
    Every nonempty set $X$ can be well ordered.
\end{theorem}

\begin{theorem}[The Axiom of Choice]
    \labthm{axiom_of_choice}
    If $\{ X_{\alpha} \}_{\alpha \in A}$ is a nonempty collection of nonempty sets, then $\prod_{\alpha \in A} X_{\alpha}$ is nonempty.
\end{theorem}

\begin{proof}
    Let $X = \bigcup_{\alpha \in A} X_\alpha$.
    So $X$ is nonempty and well ordered by \refthm{axiom_of_choice}, so we can pick a well ordering on $X$.
    For $\alpha \in A$, let $f(\alpha)$ be the minimal element of $X_\alpha$.
    Then $f \in \prod_{\alpha \in A} X_\alpha$.
\end{proof}

\begin{corollary}
    If $\{ X_{\alpha} \}_{\alpha \in A}$ is a disjoint collection of nonempty sets, there is a set $Y \subset \cup_{\alpha \in A} X_{\alpha}$ such that $Y \cap X_{\alpha}$ contains precisely one element for each $\alpha \in A$.
\end{corollary}

\begin{proof}
    Take $Y = f(A)$ where $f \in \prod_{\alpha \in A} X_\alpha$.
\end{proof}

\section{Cardinality}

\begin{definition}
    If $X$ and $Y$ are nonempty sets, we define $\card (X) \le \card (Y)$, $\card (X) = \card (Y)$, $\card (X) \ge \card (Y)$ to mean that there exists $f: X \to Y$ which is injective, bijective, or surjective, respectively.
\end{definition}

\begin{definition}
    A set $X$ is called countable (or denumerable) if $\card (X) \le \card (\N)$.
    And $\card (X) = n$ iff $\card (X) = \card (\{ 1, \dots, n \})$.
\end{definition}

\begin{proposition}
    If $X$ and $Y$ are countable, so is $X \times Y$.
\end{proposition}

\begin{proposition}
    If $A$ is countable and $X_{\alpha}$ is countable for every $\alpha \in A$, then $\cup_{\alpha \in A} X_{\alpha}$ is countable.
\end{proposition}

\begin{proposition}
    If $X$ is countably infinite, then $\card(X) = \card(\N)$.
\end{proposition}

\begin{proposition}
    $\Z$ and $\Q$ are countable.
\end{proposition}

\begin{definition}
    A set $X$ is said to have the carinality of the continuum if $\card(X) = \card(\R) = \mathfrak{c}$.
\end{definition}


\begin{proposition}
    $\card(\mathcal{P}(\N)) = \mathfrak{c}$.
\end{proposition}

\begin{corollary}
    If $\card(X) \ge \mathfrak{c}$, then $X$ is uncountable.
\end{corollary}

\section{More about well ordered set}

\begin{definition}
    If $x \in X$, the initial segment of $x$ is $I_x = \{ y \in X : y < x \}$.
\end{definition}

The principle of mathematical induction is equivlent to the fact that $\N$ is well ordered.
It can be extended to arbitrary well ordered sets as follows:

\begin{theorem}[The Principle of Transfinite Induction]
    Let $X$ be a well ordered set. 
    If $A$ is a subset of $X$ such that $x \in A$ whenever $I_x \subset A$, then $A=X$.
\end{theorem}

\begin{proposition}
    If $X$ is well ordered and $A \subset X$, then $\bigcup_{x \in A} I_x$ is either an initial segment or $X$ itself. 
\end{proposition}

\begin{proposition}
    If $X$ and $Y$ are well ordered, then either $X$ is order isomorphic to $Y$, or $X$ is order isomorphic to an initial segment in $Y$, or an initial segment in $X$ is order isomorphic to $Y$.
\end{proposition}

\begin{proposition}
    There is an uncountable well ordered set $\Omega$ such that $I_x$ is countable for each $x \in \Omega$.
    If $\Omega'$ is another set with the same properties, then $\Omega$ and $\Omega'$ are order isomorphic.
\end{proposition}

\begin{proposition}
    Every countable subset of $\Omega$ has an upper bound.
\end{proposition}

\section{The extended real number system}

\begin{definition}[Extended Real Number System]
    The extended real number system is $\bar{\R} = \R \cup \{ -\infty, \infty \}$, and to extend the usual ordering on $\R$ by declaring that $-\infty < x < \infty$ for all $x \in \R$.
    The arithmetical operations on $\R$ can be partially extended to $\bar{\R}$:
    \begin{align}
        x \pm \infty &= \pm \infty \quad (x \in \R), \\
        \infty + \infty &= \infty, \\
        -\infty -\infty &= -\infty, \\
        x \cdot (\pm \infty) &= \pm \infty \quad (x > 0), \\
        x \cdot (\pm \infty) &= \mp \infty \quad (x < 0).
    \end{align}
    We make no attempt to define $\infty - \infty$.
    And unless otherwise stated, 
    \begin{align}
        0 \cdot (\pm \infty) = 0.
    \end{align}
\end{definition}

\begin{definition}
    If $X$ is an arbitrary set (may be uncountable), and $f: X \to [0, \infty]$, we define $\sum_{x \in X} f(x)$ to be the supremum of its finite partial sums:
    \begin{align}
        \sum_{x \in X} f(x) = \sup \left\{ \sum_{x \in F} f(x): F \subset X, F \text{ is finite} \right\}.
    \end{align} 
\end{definition}

\begin{proposition}
    Given $f: X \to [0, \infty]$, let $A = {x: f(x) > 0}$.
    If $A$ is uncountable, then $\sum_{x \in X} f(x) = \infty$.
    If $A$ is countably infinite, then $\sum_{x \in X} f(x) = \sum_{n=1}^{\infty} f(g(n))$ where $g: \N \to A$ is any bijection and the sum on the right is an ordinary infinite series.
\end{proposition}

\begin{proof}
    Let $A = \bigcup_{n=1}^{\infty} A_n$ where $A_n = \{ f(x) > 1/n \}$.
    If $A$ is uncountable, then some $A_n$ must be uncountable by monotonity of $A_n$.
    And $\sum_{x \in F} f(x) > \frac{\card (F)}{n}$ for a finite subset $F$ of $A_n$.
    It follows that $\sum_{x \in X} f(x) = \infty$.

    If $A$ is countably infinite, $g: \N \to A$ is a bijection, and $B_N = g(\{ 1, \dots, N \})$, then every finite subset $F$ of $A$ is contained in some $B_N$.
    Hence, 
    \begin{align}
        \sum_{x \in F} f(x) \le \sum_{n=1}^{N} f(g(n)) \le \sum_{x \in X} f(x).
    \end{align}
    Taking supremum over $N$, we find
    \begin{align}
        \sum_{x \in F} f(x) \le \sum_{n=1}^{\infty} f(g(n)) \le \sum_{x \in X} f(x),
    \end{align}
    and then takng the supremum over $F$, we obtain the desired result. 
\end{proof}

\begin{proposition}
    \labprop{open_set_in_R_countable_disjoint_union_of_open_intervals}
    Every open set in $\R$ is a countable disjoint union of open intervals.
\end{proposition}

\begin{proof}
    TODO
\end{proof}

\section{Metric spaces}

\begin{definition}[Metric space]
    \labdef{metric_space}
    A metric on a set $X$ is a function $\rho : X \times X \to [0, \infty)$ such that
    \begin{enumerate}
        \item $\rho(x, y) = 0$ iff $x = y$;
        \item $\rho(x, y) = \rho(y, x)$ for all $x, y \in X$;
        \item $\rho(x, z) \le \rho (x, y) + \rho(y, z)$ for all $x, y, z \in X$.
    \end{enumerate}
    A set equipped with a metric is called a metric space, denoted as $(X, \rho)$
\end{definition}

\begin{definition}
    Let $(X, \rho)$ be a metric space.
    If $x \in X$ and $r > 0$, then (open) ball of radius $r$ about $x$ is $B(r, x) = \left\{ y \in X: \rho(x, y) < r \right\}$.
\end{definition}

\begin{definition}[Open and Closed]
    Let $(X, \rho)$ be a metric space.
    A set $E \subset X$ is open if for every $x \in E$ there exists $r > 0$ such that $B(r, x) \subset E$.
    A set $E \subset X$ is closed if its complement is open. 
\end{definition}

\begin{definition}
    If $E \subset X$, the union of all open sets $U \subset E$ is the largest open set contained in $E$; it is called the interior of $E$ and denoted by $E^o$.
    The intersection of all closed sets $F \supset E$ is the smallest closed set containing $E$; it is called the closure of $E$ and is denoted by $\bar{E}$.
\end{definition}

\begin{definition}
    A set $E$ is dense in $X$ if $\bar{E} = X$, and nowhere dense if $E$ has empty interior. 
\end{definition}

\begin{definition}
    $X$ is called separable if it has a countable dense subset.
\end{definition}

\begin{example}
    $\Q^n$ is a countable dense subset of $\R^n$.
\end{example}

\begin{definition}
    A sequence $\{ x_n \}$ in $X$ converges to $x \in X$ (symbolically, $x_n \to x$ or $\lim x_n = x$) if $\lim _{n \to \infty} \rho(x_n, x) = 0$.
\end{definition}

\begin{definition}[Continuity]
    \labdef{continuity}
    If $(X_1, \rho_1)$ and $(X_2, \rho_2)$ are metric spaces, a map $f: X_1 \to X_2$ is called continuous at $x \in X$ if for every $\epsilon > 0$ there exists $\delta > 0$ such that $\rho_2(f(x), f(y)) < \epsilon$ whenever $\rho_1(x, y) < \delta$. 
    In other words, such that $f^{-1} \left( B(\epsilon, f(x)) \right) \supset B(\delta, x)$.
    The map $f$ is called continuous if it is continuous at each $x \in X_1$ and uniformly continuous if, in addition, the $\delta$ in the definition of continuity can be chosen independent of $x$.
\end{definition}

\begin{proposition}
    \labprop{continuous_open}
    $f: X_1 \to X_2$ is continuous iff $f^{-1}(U)$ is open in $X_1$ for every open $U \subset X_2$.
\end{proposition}

\begin{proof}
    If the latter condition holds, then for every $x \in X$ and $\epsilon > 0$, then $f^{-1} \left( B(\epsilon, f(x)) \right)$ is open and contains $x$, so ti contains some ball about $x$.
    This means we can pick a $\delta > 0$ such that $B(\delta, x) \subset f^{-1}\left( B(\epsilon, f(x)) \right)$.
    So $f$ is continuous by \refdef{continuity}.

    Conversely, suppose $f$ is continuous, and $U$ is open in $X_2$.
    For each $y \in U$, there exists $\epsilon_y > 0$ such that $B(\epsilon_y, y) \subset U$, and for each $x \in f^{-1}(\{ y \})$, there exhists $\delta_x > 0$ such that 
    \begin{align}
        B(\delta_x, x) \subset f^{-1}\left( B(\epsilon, f(x)) \right) \subset f^{-1}(U).
    \end{align}
    Thus $f^{-1}(U) = \bigcup_{x \in f^{-1}(U)} B(\delta_x, x)$ is open.
\end{proof}

\begin{proposition}
    A closed subset of a complete metric space is complete, and a complete subset of an arbitrary metric space is closed.
\end{proposition}

\begin{proof}
    
\end{proof}

\begin{definition}[Cauchy]
    A sequence $\{x_n\}$ in a metric space $(X, \rho)$ is called Cauchy if $\rho(x_n, x_m) \to 0$ as $n, m\to \infty$.
\end{definition}

\begin{definition}[Complete]
    A subset $E$ of $X$ is called complete if every Cauchy sequence in $E$ converges and its limit is in $E$.
\end{definition}

\begin{example}
    $\R^n$ is complete, whereas $\Q^n$ is not.
\end{example}

\begin{theorem}
    If $E$ is a subset of the metric space $(X, \rho)$, the following are equivalent:
    \begin{enumerate}
        \item $E$ is complete and totally bounded.
        \item (The Bolzano-Weierstrass Property) Every sequence in $E$ has a subsequence that converges to a point of $E$.
        \item (The Heine-Borel Property) If $\{ V_{\alpha} \}_{\alpha \in A}$ is a cover of $E$ by open sets, there is a finite set $F \subset A$ such that $\{ V_{\alpha} \}_{\alpha \in F}$ covers $E$.
    \end{enumerate}
\end{theorem}

\begin{definition}[Compact]
    \labdef{compact}
    A set $E$ is a subset of the metric space $(X, \rho)$.
    $E$ is called compact if $E$ is complete and totally bounded.
\end{definition}

\begin{theorem}
    Every closed and bounded subset of $\R^n$ is compact.
\end{theorem}
\setchapterstyle{kao}
\setchapterpreamble[u]{\margintoc}
\chapter{Measures}
\labch{measures}

\section{\texorpdfstring{$\sigma$}{sigma}-algebras}

\begin{definition}[Algebra]
    Let $X$ be a nonempty set.
    An algebra of sets on $X$ is a nonempty collection $\mathcal{A}$ of subsets of $X$ that is closed under finite unions and complements.
\end{definition}

\begin{definition}[\texorpdfstring{$\sigma$}{sigma}-algebra]
    A $\sigma$-algebra is an algebra that is closed under countable unions.
\end{definition}

Since $\bigcap_{j} E_j = (\bigcup_{j} E_j^c)^c$, algebras (respectively, $\sigma$-algebra) are also closed under finite (respectively, countable) intersections.

\begin{example}
    If $X$ is any set, $\mathcal{P}(X)$ and $\{ \emptyset, X \}$ are $\sigma$-algebra.
\end{example}

\begin{example}
    If $X$ is uncountable, then
    \begin{align}
        \mathcal{A} = \{ E \subset X : E \text{ is countable or } E^{c} \text{ is countable} \}
    \end{align}
    is a $\sigma$-algebra, called the $\sigma$-algebra of countable or co-countable sets.
\end{example}

\begin{definition}
    If $\mathcal{E}$ is any subset of $\mathcal{P}(X)$, there is a unique smallest $\sigma$-algebra $\mathcal{M}(\mathcal{E})$ containing $\mathcal{E}$, namely, the intersection of all $\sigma$-algebras containing $\mathcal{E}$.
    $\mathcal{M}(\mathcal{E})$ is called the $\sigma$-algebra generated by $\mathcal{E}$.
\end{definition}

\begin{lemma}
    If $\mathcal{E} \subset \mathcal{M}(\mathcal{F})$, then $\mathcal{M}(\mathcal{E}) \subset \mathcal{M}(\mathcal{F})$.
\end{lemma}

\begin{proof}
    Because $\mathcal{M}(\mathcal{F})$ is a $\sigma$-algebra containing $\mathcal{E}$, it contains $\mathcal{M}(\mathcal{E})$.
\end{proof}

\begin{definition}[Borel \texorpdfstring{$\sigma$}{sigma}-algebra]
    If $X$ is any metric space, or more generally any topological space, the $\sigma$-algebra generated by the family of open sets in $X$ (or equivalently the family of closed sets in $X$) is called Borel $\sigma$-algebra on $X$ and is denoted by $\mathcal{B}_{X}$.
    Its members are called Borel sets.
\end{definition}

\begin{definition}
    Let $\{ X_{\alpha} \}_{\alpha \in A}$ be an indexed collection nonempty sets, $X = \prod_{\alpha \in A} X_{\alpha}$, and $\pi_{\alpha}: X \to X_{\alpha}$ the coordinated maps.
    If $\mathcal{M}_{\alpha}$ is a $\sigma$-algebra on $X_{\alpha}$ for each $\alpha$, the product $\sigma$-algebra on $X$ is the $\sigma$-algebra generated by
    \begin{align}
        \{ \pi_{\alpha}^{-1} (E_\alpha) : E_{\alpha} \in \mathcal{M}_{\alpha}, \alpha \in A \}.
    \end{align}
    We denote this $\sigma$-algebra by $\bigotimes_{\alpha \in A} \mathcal{M}_{\alpha}$.
\end{definition}

\begin{proposition}
    If $A$ is countable, then $\bigotimes_{\alpha \in A} \mathcal{M}_{\alpha}$ is the $\sigma$-algebra generated by $\{ \prod_{\alpha \in A} E_{\alpha} : E_{\alpha} \in \mathcal{M}_{\alpha} \}$.
\end{proposition}

\begin{proof}

\end{proof}

\begin{proposition}
    Suppose that $\mathcal{M}_{\alpha}$ is generated by $\mathcal{E}_{\alpha}$, $\alpha \in A$.
    Then $\bigotimes_{\alpha \in A} \mathcal{M}_{\alpha}$ is generated by $\mathcal{F}_1 = \{ \pi_{\alpha}^{-1} (E_\alpha) : E_{\alpha} \in \mathcal{E}_{\alpha}, \alpha \in A \}$.
    If $A$ is countable and $X_\alpha \in \mathcal{E}_{\alpha}$ for all $\alpha$, then $\bigotimes_{\alpha \in A} \mathcal{M}_{\alpha}$ is generated by $\mathcal{F}_2 = \{ \prod_{\alpha \in A} E_{\alpha} : E_{\alpha} \in \mathcal{E}_{\alpha}  \}$.
\end{proposition}

\begin{proof}

\end{proof}

\begin{proposition}
    Let $X_1, \dots, X_n$ be metric spaces and let $X = \prod_{j=1}{n} X_j$ equipped with the product metric.
    Then $\bigotimes_{j=1}^{n} \mathcal{B}_{X_j} \subset \mathcal{B}_{X}$.
    If the $X_j$'s are separable, then $\bigotimes_{j=1}^{n} \mathcal{B}_{X_j} = \mathcal{B}_{X}$.
\end{proposition}

\begin{proof}

\end{proof}

\begin{corollary}
    $\bigotimes_{j=1}^{n} \mathcal{B}_{\R} = \mathcal{B}_{\R^n}$.
\end{corollary}

\begin{proof}

\end{proof}

\begin{definition}[Elementary family]
    An elementary family is a collection $\mathcal{E}$ of subsets of $X$ such that
    \begin{enumerate}
        \item $\emptyset \in \mathcal{E}$;
        \item if $E, F \in \mathcal{E}$, then $E \cap F \in \mathcal{E}$;
        \item if $E \in \mathcal{E}$, then $E^c$ is a finite disjoint union of members of $\mathcal{E}$.
    \end{enumerate}
\end{definition}

\begin{proposition}
    If $\mathcal{E}$ is an elementary family, the collection $\mathcal{A}$ of finite disjoint unions of members of $\mathcal{E}$ is an algebra.
\end{proposition}

\section{Measures}

\begin{definition}[Measure]
    Let $X$ be a set equipped with a $\sigma$-algebra $\mathcal{M}$.
    A measure on $\mathcal{M}$ (or on $(X, \mathcal{M})$, or simply on $X$ if $\mathcal{M}$ is understood) is a function $\mu: \mathcal{M} \to [0, \infty]$ such that
    \begin{enumerate}
        \item $\mu(\emptyset) = 0$;
        \item if $\{ E_j \}_{j=1}^{\infty}$ is a sequence of disjoint sets in $\mathcal{M}$, then $\mu(\bigcup_{j=1}^{\infty} E_j) = \sum _{j=1}^{\infty} \mu(E_j)$.
    \end{enumerate}
\end{definition}

\begin{definition}[Finite measure]
    Let $(X, \mathcal{M}, \mu)$ be a measure space.
    If $\mu(X) < \infty$, $\mu$ is called finite.
\end{definition}

\begin{definition}[$\sigma$-finite measure]
    Let $(X, \mathcal{M}, \mu)$ be a measure space.
    If $X = \bigcup_{j=1}^{\infty} E_j$ where $E_j \in \mathcal{M}$ and $\mu(E_j) < \infty$ for all $j$, $\mu$ is called $\sigma$-finite.
\end{definition}

\begin{definition}[Semifinite measure]
    Let $(X, \mathcal{M}, \mu)$ be a measure space.
    If for each $E \in \mathcal{M}$ with $\mu(E) = \infty$ there exists $F \in \mathcal{M}$ with $F \subset E$ and $0 < \mu(F) < \infty$, $\mu$ is called semifinite.
\end{definition}

\begin{example}
    Let $x$ be an infinite set and $\mathcal{M} = \mathcal{P}(X)$.
    Define $\mu(E) = 0$ if $E$ is finite, $\mu(E) = \infty$ if $E$ is infinite.
    Then, $\mu$ is a finitely additive measure but not a measure.
\end{example}

\begin{theorem}
    Let $(X, \mathcal{M}, \mu)$ be a measure space.
    \begin{enumerate}
        \item If $E, F \in \mathcal{M}$ and $E \subset F$, then $\mu(E) \le \mu(F)$.
        \item If $\{ E_{j} \}_{j=1}^{\infty} \subset \mathcal{M}$, then $\mu(\bigcup_{j=1}^{\infty} E_j) \le \sum _{j=1}^{\infty} \mu(E_j)$.
        \item If $\{ E_{j} \}_{j=1}^{\infty} \subset \mathcal{M}$ and $E_1 \subset E_2 \subset \cdots$, then $\mu(\bigcup_{j=1}^{\infty} E_j) = \lim_{j\to \infty} \mu(E_j)$.
        \item If $\{ E_{j} \}_{j=1}^{\infty} \subset \mathcal{M}$, $E_1 \supset E_2 \supset \cdots$, and $\mu(E_1) < \infty$, then $\mu(\bigcap_{j=1}^{\infty} E_j) = \lim_{j\to \infty} \mu(E_j)$.
    \end{enumerate}
\end{theorem}

\begin{proof}
    If $E \subset F$, then $\mu(F) = \mu(E) + \mu(F \setminus E) \ge \mu(E)$.

    Let $F_1 = E_1$, $F_j = E_j \setminus (\bigcup_{k=1}^{j-1} E_k)$.
    Then ${F_j}_{j=1}^{\infty}$ is disjoint and $\bigcup_{j=1}^{\infty} F_j = \bigcup_{j=1}^{\infty} E_j$.
    \begin{align}
        \mu( \bigcup_{j=1}^{\infty} E_j) = \mu( \bigcup_{j=1}^{\infty} F_j ) = \sum_{j=1}^{\infty} \mu(F_j) \le \sum_{j=1}^{\infty} \mu(E_j).
    \end{align}

    Setting $E_0 = \emptyset$, we have
    \begin{align}
        \mu( \bigcup_{j=1}^{\infty} E_j) & = \mu( \bigcup_{j=1}^{\infty} E_j \setminus E_{j-1}) = \sum_{j=1}^{\infty} \mu(E_j \setminus E_{j-1}) \\
                                         & = \lim_{n \to \infty} \sum_{j=1}^{n} \mu(E_j \setminus E_{j-1}) = \lim_{n \to \infty} \mu(E_n).
    \end{align}

    Let $F_j = E_1 \setminus E_j$, then $F_1 \subset F_2 \subset \cdots$ and $\mu(\bigcup_{j=1}^{\infty} F_j) = \lim_{j\to \infty} \mu(F_j)$.
    Also notice that $\mu(E_1) = \mu(F_j) + \mu(E_j)$ and $\bigcup_{j=1}^{\infty} F_j = E_1 \setminus \left( \bigcap_{j=1}^{\infty} E_j \right)$.
    \begin{align}
        \mu(E_1) & = \mu( \bigcap_{j=1}^{\infty} E_j ) + \mu(\bigcup_{j=1}^{\infty} F_j)           \\
                 & = \mu( \bigcap_{j=1}^{\infty} E_j ) + \lim_{j\to \infty} \mu(F_j)               \\
                 & = \mu( \bigcap_{j=1}^{\infty} E_j ) + \lim_{j\to \infty} [\mu(E_1) - \mu(E_j)].
    \end{align}
    Since $\mu(E_1) < \infty$, we may subtract it from both sides to yield the desired result.
\end{proof}

\begin{definition}
    If $(X, \mathcal{M}, \mu)$ is a measure space, a set $E \in \mathcal{M}$ such that $\mu(E) = 0$ is called a null set.
\end{definition}

\begin{definition}[Almost everywhere]
    If a statement about points $x \in X$ is true except for $x$ in some null set, we say it is true almost everywhere (abbreviated a.e.) or for almost every $x$.
\end{definition}

\begin{definition}
    A measure whose domain includes all subsets of null sets is called complete.
\end{definition}

\begin{theorem}
    Suppose $(X, \mathcal{M}, \mu)$ is a measure space.
    Let $\mathcal{N} = \{ N \in \mathcal{M}: \mu(N) = 0 \}$ and $\Bar{\mathcal{M}} = \{ E \cup F: E \in \mathcal{M} \text{ and } F \subset N \text{ for some } N \in \mathcal{N} \}$.
    Then $\bar{\mathcal{M}}$ is a $\sigma$-algebra, and there is a unique extension $\bar{\mu}$ of $\mu$ to a complete measure on $\bar{\mathcal{M}}$.
\end{theorem}

\begin{proof}
    Since $\mathcal{M}$ and $\mathcal{N}$ are closed under countable unions, so is $\bar{\mathcal{M}}$.
    Next, we need to show that $\bar{\mathcal{M}}$ is closed under complements.
    If $E \cup F \in \bar{\mathcal{M}}$ where $E \in \mathcal{M}$ and $F \subset N \in \mathcal{N}$, we can assume that $E \cap N = \emptyset$.
    Then $E \cup F = (E \cup N) \cap (N^c \cup F)$, so $(E \cup F)^c = (E \cup N)^c \cup (N \setminus F)$ where $(E \cup N)^c \in \mathcal{M}$ and $(N \setminus F) \subset N$.
    Therefore, $(E \cup F)^c \in \bar{\mathcal{M}}$, and $\bar{\mathcal{M}}$ is a $\sigma$-algebra.

    TODO: Show that there is a unique extension $\bar{\mu}$ of $\mu$ to a complete measure on $\bar{\mathcal{M}}$.
\end{proof}

\begin{proposition}
    Every $\sigma$-finite measure is semifinite.
\end{proposition}

\begin{proof}
    Let $(X, \mathcal{M}, \mu)$ be a measure space.
    Suppose a $\sigma$-finite measure $\mu$ is not semifinite.
    So there exists $E \in \mathcal{M}$ such taht $\mu(E) = \infty$, and for any $F \in \mathcal{M}$ with $F \subset E$, we have $mu(F) \in \{ 0, \infty \}$.

    Since $\mu$ is $\sigma$-finite, there are $\{ E_j \}_{j=1}^{\infty} \subset \mathcal{M}$ such that $X = \bigcup_{j=1}^{\infty} E_j$ and $\mu(E_j) < \infty$, for any $j$.
    Then we have $E = \bigcup_{j=1}^{\infty} (E \cap E_j)$ and $\mu(E) \le \sum_{j=1}^{\infty} \mu(E \cap E_j)$ by subadditivity.
    Because $\mu(E \cap E_j) \le \mu(E_j) < \infty$ and $\mu(E \cap E_j) \in \{ 0, \infty \}$ as $(E \cap E_j) \subset E$, $\mu(E \cap E_j) = 0$.
    But then $\mu(E) = 0$, which leads to a contradiction.
\end{proof}

\begin{proposition}
    If $\mu_1, \dots, \mu_n$ are measures on $(X, \mathcal{M})$ and $a_1, \dots, a_n \in [0, \infty)$, then $\mu = \sum_{k=1}^{n} a_k \mu_k$ is a measure on $(X, \mathcal{M})$.
\end{proposition}

\begin{proposition}
    If $(X, \mathcal{M}, \mu)$ is a measure space and $E \in \mathcal{M}$, define $\mu_E(A) = \mu(A \cap E)$ for $A \in \mathcal{M}$.
    Then $\mu_E$ is a measure.
\end{proposition}

\begin{proposition}
    If $(X, \mathcal{M}, \mu)$ is a measure space and $\{ E_j \}_{j=1}^{\infty} \subset \mathcal{M}$, then $\mu(\liminf E_j) \le \liminf \mu(E_j)$.
    Also, $\mu(\limsup E_j) \ge \limsup \mu(E_j)$ provided that $\mu(\bigcup_{j=1}^{\infty} E_j) < \infty$.
\end{proposition}

\begin{proposition}
    If $\mu$ is a semifinite measure and $\mu(E) = \infty$, for any $C > 0$ there exists $F \subset E$ with $C < \mu(F) < \infty$.
\end{proposition}

\section{Outer measures}

\begin{definition}[Outer measure]
    \labdef{outer_measure}
    An outer measure on a nonempty set $X$ is a function $\mu^*: \mathcal{P}(X) \to [0, \infty]$ that satisfies
    \begin{enumerate}
        \item $\mu^*(\emptyset) = 0$;
        \item $\mu^*(A) \le \mu^*(B)$ if $A \subset B$;
        \item $\mu^*(\bigcup_{j=1}^{\infty} A_j) \le \sum_{j=1}^{\infty} \mu^*(A_j)$.
    \end{enumerate}
\end{definition}

\begin{proposition}
    \labthm{covering_outer_measure}
    Let $\mathcal{E} \subset \mathcal{P}(X)$ and $\rho : \mathcal{E} \to [0, \infty]$ be such that $\emptyset \in \mathcal{E}$, $X \in \mathcal{E}$, and $\rho(\emptyset) = 0$.
    For any $A \subset X$, define
    \begin{align}
        \mu^*(A) = \inf \left\{ \sum_{j=1}^{\infty} \rho(E_j): E_j \in \mathcal{E} \text{ and } A \subset \bigcup _{j=1}^{\infty} E_j \right\}.
    \end{align}
    Then, $\mu^*$ is an outer measure.
\end{proposition}

\begin{proof}

\end{proof}

\begin{definition}
    If $\mu^*$ is an outer measure on $X$, a set $A \subset X$ is called $\mu^*$-measurable if $\mu^*(E) = \mu^*(E \cap A) + \mu^*(E \cap A^c)$ for all $E \subset X$.
\end{definition}

The inequality $\mu^*(E) \le \mu^*(E \cap A) + \mu^*(E \cap A^c)$ holds for any $A$ and $E$ because of the subadditivity in \refdef{outer_measure}.
So to prove that $A$ is $\mu^*$-measurable, it suffices to prove the reverse inequality.
The latter is trivial if $\mu^*(E) = \infty$, so we see that $A$ is is $\mu^*$-measurable iff $\mu^*(E) \ge \mu^*(E \cap A) + \mu^*(E \cap A^c)$ for all $E \subset X$ such that $\mu^*(E) < \infty$.

\begin{theorem}[Caratheodory's Theorem]
    \labthm{caratheodory}
    If $\mu^*$ is an outer measure on $X$, the collection $\mathcal{M}$ of $\mu^*$-measurable sets is a $\sigma$-algebra, and the restriction of $\mu^*$ to $\mathcal{M}$ is a complete measure.
\end{theorem}

\begin{proof}
    TODO

    Finally, if $\mu^*(A) = 0$, for any $E \subset X$ we have
    \begin{align}
        \mu^*(E) \le \mu^*(E \cap A) + \mu^*(E \cap A^c) = \mu^*(E \cap A^c) \le \mu^*(E),
    \end{align}
    so that $A \in \mathcal{M}$.
    Therefore, $\mu^*|\mathcal{M}$ is a complete measure.
\end{proof}

Our first applications of \refthm{caratheodory} will be in the context of extending measures from algebras to $\sigma$-algebras.

\begin{definition}[Premeasure]
    \labdef{premeasure}
    If $\mathcal{A} \subset \mathcal{P}(X)$ is an algebra, a function $\mu_0: \mathcal{A} \to [0, \infty]$ will be called a premeasure if
    \begin{enumerate}
        \item $\mu_0(\emptyset) = 0$;
        \item if $\{ A_j \}_{j=1}^{\infty}$ is a sequence of disjoint sets in $\mathcal{A}$ such that $\bigcup_{j=1}^{\infty} A_j \in \mathcal{A}$, then $\mu_0(\bigcup_{j=1}^{\infty} A_j) = \sum _{j=1}^{\infty} \mu_0(A_j)$.
    \end{enumerate}
\end{definition}

In particular, a premeasure is finitely additive since one can take $A_j = \emptyset$ for $j$ large.
The notions of finite and $\sigma$-finite premeasure are defined just as for measures.
If $\mu_0$ is a premeasure on $\mathcal{A} \subset \mathcal{P}(X)$, it induces an outer measure on $X$ in accordance with \refthm{covering_outer_measure}, namely

\begin{align}
    \mu^*(E) = \inf \left\{ \sum_{j=1}^{\infty} \mu_0(A_j) : A_j \in \mathcal{A}, E \subset \bigcup_{j=1}^{\infty} A_j \right\} \labeq{premeasure_outer_measure}
\end{align}

\begin{proposition}
    If $\mu_0$ is a premeasure on $\mathcal{A}$ and $\mu^*$ is defined by \refeq{premeasure_outer_measure}, then
    \begin{enumerate}
        \item $\mu^*|\mathcal{A} = \mu_0$;
        \item every set in $\mathcal{A}$ is $\mu^*$-measurable.
    \end{enumerate}
\end{proposition}

\begin{proof}

\end{proof}

\begin{theorem}
    Let $\mathcal{A} \subset \mathcal{P}(X)$ be an algebra, $\mu_0$ a premeasure on $\mathcal{A}$, and $\mathcal{M}$ the $\sigma$-algebra generated by $\mathcal{A}$.
    There exists a measure $\mu$ on $\mathcal{M}$ whose restriction to $\mathcal{A}$ is $\mu_0$ --- namely, $\mu = \mu^* | \mathcal{M}$ where $\mu^*$ is given by \refeq{premeasure_outer_measure}.
    If $\nu$ is another measure on $\mathcal{M}$ that extends $\mu_0$, then $\nu(E) \le \mu(E)$ for all $E in \mathcal{M}$, with equality when $\mu(E) < \infty$.
    If $\mu_0$ is $\sigma$-finite, then $\mu$ is the unique extension of $\mu_0$ to a measure on $\mathcal{M}$.
\end{theorem}

\begin{proof}

\end{proof}

\section{Borel measures on the real line}

In this section, we will construct a definitive theory for measuring subsets of $\R$ based on the idea that the measure of an interval is its length.
We begin with a more general construction that yields a large family of measures on $\R$ whose domain is the Borel $\sigma$-algebra $\mathcal{B}_{\R}$; such measures are called Borel measures on $\R$.

\begin{proposition}
    Let $F:\R \to \R$ be increasing and right continuous.
    If $(a_j, b_j]$ are disjoint h-intervals, let
    \begin{align}
        \mu_0 \left(\bigcup_{j=1}^{n} (a_j, b_j] \right) = \sum_{j=1}^{n} [F(b_j) - F(a_j)],
    \end{align}
    and let $\mu_0(\emptyset) = 0$.
    Then $\mu_0$ is a premeasure on the algebra $\mathcal{A}$.
\end{proposition}

\begin{proof}

\end{proof}

\begin{theorem}
    If $F: \R \to \R$ is any increasing, right continuous function, there is a unique Borel measure $\mu_F$ on $\R$ such that $\mu_F ((a,b]) = F(b) - F(A)$ for all $a, b$.
    If $G$ is another such function, we have $\mu_F = \mu_G$ iff $F-G$ is constant.
    Conversely, if $\mu$ is Borel measure on $\R$ that is finite on all bounded Borel sets and we define
    \begin{align}
        F(x) = \begin{cases}
            \mu((0, x]), \text{ if } x > 0, \\
            0, \text{ if } x = 0,\\
            -\mu((-x, 0]), \text{ if } x < 0, \\
        \end{cases}
    \end{align}
    then $F$ is increasing and right continuous, and $\mu = \mu_F$.
\end{theorem}

\begin{proof}
    
\end{proof}


% \input{chapters/convexity.tex}
% \input{chapters/nonlinear_optimization.tex}

% \pagelayout{wide} % No margins
% \addpart{Dynamic Optimization}
% \pagelayout{margin} % Restore margins

% \input{chapters/layout.tex}
% \setchapterstyle{kao}
\setchapterpreamble[u]{\margintoc}
\chapter{Mathematics and Boxes}
\labch{mathematics}

\section{Theorems}

Despite most people complain at the sight of a book full of equations, 
mathematics is an important part of many books. Here, we shall 
illustrate some of the possibilities. We believe that theorems, 
definitions, remarks and examples should be emphasised with a shaded 
background; however, the colour should not be to heavy on the eyes, so 
we have chosen a sort of light yellow.\sidenote[][*10]{The boxes are all of the 
same colour here, because we did not want our document to look like 
\href{https://en.wikipedia.org/wiki/Harlequin}{Harlequin}.}

\begin{definition}
\labdef{openset}
Let $(X, d)$ be a metric space. A subset $U \subset X$ is an open set 
if, for any $x \in U$ there exists $r > 0$ such that $B(x, r) \subset 
U$. We call the topology associated to d the set $\tau\textsubscript{d}$ 
of all the open subsets of $(X, d).$
\end{definition}

\refdef{openset} is very important. I am not joking, but I have inserted 
this phrase only to show how to reference definitions. The following 
statement is repeated over and over in different environments.

\begin{theorem}
A finite intersection of open sets of (X, d) is an open set of (X, d), 
i.e $\tau\textsubscript{d}$ is closed under finite intersections. Any 
union of open sets of (X, d) is an open set of (X, d).
\end{theorem}

\begin{proposition}
A finite intersection of open sets of (X, d) is an open set of (X, d), 
i.e $\tau\textsubscript{d}$ is closed under finite intersections. Any 
union of open sets of (X, d) is an open set of (X, d).
\end{proposition}

\marginnote{You can even insert footnotes inside the theorem 
environments; they will be displayed at the bottom of the box.}

\begin{lemma}
A finite intersection\footnote{I'm a footnote} of open sets of (X, d) is 
an open set of (X, d), i.e $\tau\textsubscript{d}$ is closed under 
finite intersections. Any union of open sets of (X, d) is an open set of 
(X, d).
\end{lemma}

You can safely ignore the content of the theorems\ldots I assume that if 
you are interested in having theorems in your book, you already know 
something about the classical way to add them. These example should just 
showcase all the things you can do within this class.

\begin{corollary}[Finite Intersection, Countable Union]
\labcorollary{finite_intersection}
A finite intersection of open sets of (X, d) is an open set of (X, d), 
i.e $\tau\textsubscript{d}$ is closed under finite intersections. Any 
union of open sets of (X, d) is an open set of (X, d).
\end{corollary}

\begin{proof}
The proof is left to the reader as a trivial exercise. Hint: \blindtext
\end{proof}

\begin{definition}
Let $(X, d)$ be a metric space. A subset $U \subset X$ is an open set 
if, for any $x \in U$ there exists $r > 0$ such that $B(x, r) \subset 
U$. We call the topology associated to d the set $\tau\textsubscript{d}$ 
of all the open subsets of $(X, d).$
\end{definition}

\marginnote{
	Here is a random equation, just because we can:
	\begin{equation*}
  x = a_0 + \cfrac{1}{a_1
          + \cfrac{1}{a_2
          + \cfrac{1}{a_3 + \cfrac{1}{a_4} } } }
	\end{equation*}
}

\begin{example}
Let $(X, d)$ be a metric space. A subset $U \subset X$ is an open set 
if, for any $x \in U$ there exists $r > 0$ such that $B(x, r) \subset 
U$. We call the topology associated to d the set $\tau\textsubscript{d}$ 
of all the open subsets of $(X, d).$
\end{example}

\begin{remark}
Let $(X, d)$ be a metric space. A subset $U \subset X$ is an open set 
if, for any $x \in U$ there exists $r > 0$ such that $B(x, r) \subset 
U$. We call the topology associated to d the set $\tau\textsubscript{d}$ 
of all the open subsets of $(X, d).$
\end{remark}

As you may have noticed, definitions, example and remarks have independent counters; theorems, propositions, lemmas and corollaries share the same counter. \sidenote{\refdef{openset}, \refcorollary{finite_intersection}}

\begin{remark}
Here is how an integral looks like inline: $\int_{a}^{b} x^2 dx$, and 
here is the same integral displayed in its own paragraph:
\[\int_{a}^{b} x^2 dx\]
\end{remark}


There is also an environment for exercises.
\begin{align}
\int _{\C} \frac{\pp y}{\pp z}\dd x
\end{align}

\begin{exercise}
Prove (or disprove) the Riemann hypothesis.
\end{exercise}

We provide one package for the theorem styles: 
\href{kaotheorems.sty}{kaotheorems.sty}, to which you can pass the 
\Option{framed} option you do want coloured boxes around theorems, like 
in this document.\sidenote{The styles without \Option{framed} are not 
showed, but actually the only difference is that they don't have the 
yellow boxes.} You may want to edit this files according to your taste 
and the general style of the book. However, there is an option to 
customise the background colour of the boxes if you use the 
\Option{framed} option: when you load this package, you can pass it the 
\Option{background=mycolour} option (replace \enquote{mycolour} with the 
actual colour, for instance, \enquote{red!35!white}). This will change 
the colour of all the boxes, but it is also possible to override the 
default colour only for some elements. For instance, the 
\Option{propositionbackground=mycolour} option will change the colour 
for propositions only. There are similar options for theorem, 
definition, lemma, corollary, remark, and example.

\section[Boxes \& Environments]{Boxes \& Custom Environments}
\sidenote[][*1.8]{Notice that in the table of contents and in the 
	header, the name of this section is \enquote{Boxes \& Environments}; 
	we achieved this with the optional argument of the \texttt{section} 
	command.}

Say you want to insert a special section, an optional content or just 
something you want to emphasise. We think that nothing works better than 
a box in these cases. We used \Package{mdframed} to construct the ones 
shown below. You can create and modify such environments by editing the 
provided file \href{style/environments.sty}{environments.sty}.

\begin{kaobox}[frametitle=Title of the box]
\blindtext
\end{kaobox}

If you set up a counter, you can even create your own numbered 
environment.

\begin{kaocounter}
	Asshole.
\end{kaocounter}

\section{Experiments}

It is possible to wrap marginnotes inside boxes, too. Audacious readers 
are encouraged to try their own experiments and let me know the 
outcomes.

\marginnote[-2.2cm]{
	\begin{kaobox}[frametitle=title of margin note]
		Margin note inside a kaobox.\\
		(Actually, kaobox inside a marginnote!)
	\end{kaobox}
}

I believe that many other special things are possible with the 
\Class{kaobook} class. During its development, I struggled to keep it as 
flexible as possible, so that new features could be added without too 
great an effort. Therefore, I hope that you can find the optimal way to 
express yourselves in writing a book, report or thesis with this class, 
and I am eager to see the outcomes of any experiment that you may try.

% \begin{margintable}
% 	\captionsetup{type=table,position=above}
% 	\begin{kaobox}
% 		\caption{caption}
% 		\begin{tabular}{ |c|c|c|c| }
% 			\hline
% 			col1 & col2 & col3 \\
% 			\hline
% 			\multirow{3}{4em}{Multiple row} & cell2 & cell3 \\ & cell5 
% 			%& cell6 \\ 
% 			& cell8 & cell9 \\
% 			\hline
% 		\end{tabular}
% 	\end{kaobox}
% \end{margintable}


% \appendix % From here onwards, chapters are numbered with letters, as is the appendix convention

% \pagelayout{wide} % No margins
% \addpart{Appendix}
% \pagelayout{margin} % Restore margins

% \input{chapters/appendix.tex}

%----------------------------------------------------------------------------------------

\backmatter % Denotes the end of the main document content
\setchapterstyle{plain} % Output plain chapters from this point onwards

%----------------------------------------------------------------------------------------
%	BIBLIOGRAPHY
%----------------------------------------------------------------------------------------

% The bibliography needs to be compiled with biber using your LaTeX editor, or on the command line with 'biber main' from the template directory

\defbibnote{bibnote}{References are listed in citation order.\par\bigskip} % Prepend this text to the bibliography
\printbibliography[heading=bibintoc, title=Bibliography, prenote=bibnote] % Add the bibliography heading to the ToC, set the title of the bibliography and output the bibliography note

%----------------------------------------------------------------------------------------
%	NOMENCLATURE
%----------------------------------------------------------------------------------------

% The nomenclature needs to be compiled on the command line with 'makeindex main.nlo -s nomencl.ist -o main.nls' from the template directory

% \nomenclature{$c$}{Speed of light in a vacuum inertial frame}
% \nomenclature{$h$}{Planck constant}

% \renewcommand{\nomname}{Notation} % Rename the default 'Nomenclature'
% \renewcommand{\nompreamble}{The next list describes several symbols that will be later used within the body of the document.} % Prepend this text to the nomenclature

% \printnomenclature % Output the nomenclature

%----------------------------------------------------------------------------------------
%	GREEK ALPHABET
% 	Originally from https://gitlab.com/jim.hefferon/linear-algebra
%----------------------------------------------------------------------------------------

% \vspace{1cm}

% {\usekomafont{chapter}Greek Letters with Pronunciations} \\[2ex]
% \begin{center}
% 	\newcommand{\pronounced}[1]{\hspace*{.2em}\small\textit{#1}}
% 	\begin{tabular}{l l @{\hspace*{3em}} l l}
% 		\toprule
% 		Character & Name & Character & Name \\ 
% 		\midrule
% 		$\alpha$ & alpha \pronounced{AL-fuh} & $\nu$ & nu \pronounced{NEW} \\
% 		$\beta$ & beta \pronounced{BAY-tuh} & $\xi$, $\Xi$ & xi \pronounced{KSIGH} \\ 
% 		$\gamma$, $\Gamma$ & gamma \pronounced{GAM-muh} & o & omicron \pronounced{OM-uh-CRON} \\
% 		$\delta$, $\Delta$ & delta \pronounced{DEL-tuh} & $\pi$, $\Pi$ & pi \pronounced{PIE} \\
% 		$\epsilon$ & epsilon \pronounced{EP-suh-lon} & $\rho$ & rho \pronounced{ROW} \\
% 		$\zeta$ & zeta \pronounced{ZAY-tuh} & $\sigma$, $\Sigma$ & sigma \pronounced{SIG-muh} \\
% 		$\eta$ & eta \pronounced{AY-tuh} & $\tau$ & tau \pronounced{TOW (as in cow)} \\
% 		$\theta$, $\Theta$ & theta \pronounced{THAY-tuh} & $\upsilon$, $\Upsilon$ & upsilon \pronounced{OOP-suh-LON} \\
% 		$\iota$ & iota \pronounced{eye-OH-tuh} & $\phi$, $\Phi$ & phi \pronounced{FEE, or FI (as in hi)} \\
% 		$\kappa$ & kappa \pronounced{KAP-uh} & $\chi$ & chi \pronounced{KI (as in hi)} \\
% 		$\lambda$, $\Lambda$ & lambda \pronounced{LAM-duh} & $\psi$, $\Psi$ & psi \pronounced{SIGH, or PSIGH} \\
% 		$\mu$ & mu \pronounced{MEW} & $\omega$, $\Omega$ & omega \pronounced{oh-MAY-guh} \\
% 		\bottomrule
% 	\end{tabular} \\[1.5ex]
% 	Capitals shown are the ones that differ from Roman capitals.
% \end{center}

%----------------------------------------------------------------------------------------
%	GLOSSARY
%----------------------------------------------------------------------------------------

% The glossary needs to be compiled on the command line with 'makeglossaries main' from the template directory

\setglossarystyle{listgroup} % Set the style of the glossary (see https://en.wikibooks.org/wiki/LaTeX/Glossary for a reference)
\printglossary[title=Special Terms, toctitle=List of Terms] % Output the glossary, 'title' is the chapter heading for the glossary, toctitle is the table of contents heading

%----------------------------------------------------------------------------------------
%	INDEX
%----------------------------------------------------------------------------------------

% The index needs to be compiled on the command line with 'makeindex main' from the template directory

\printindex % Output the index

%----------------------------------------------------------------------------------------
%	BACK COVER
%----------------------------------------------------------------------------------------

% If you have a PDF/image file that you want to use as a back cover, uncomment the following lines

%\clearpage
%\thispagestyle{empty}
%\null%
%\clearpage
%\includepdf{cover-back.pdf}

%----------------------------------------------------------------------------------------

\end{document}
